\textbf{An OFDM transceiver has 85 MHz of available RF bandwidth, of which 80 MHz can be occupied for data transmission, leaving $2.5$ MHz at each side as guard bands. The subcarrier spacing is $20$ kHz, and the delay spread of the channel almost never exceeds 6 $\mu$s.
}

\textbf{Question: What is the smallest possible CP overhead of this system, in percentage?}
\vspace{0.5cm}

$O{_CP} = \frac{L_cT}{NT}$

$L_cT= \Delta_{Tmax} = 6 \mu$s

$\Delta_c = \frac{1}{NT} \rightarrow NT = \frac{1}{\Delta_c}= \frac{1}{20 \times 10^3} = 50 \mu s$

$O{_CP} = \frac{6\mu s}{50 \mu s} = 0.12$

\textbf{The smallest possible CP overhead of this system is 12\%}

\vspace{1cm}
\textbf{Question: Explain why it is not feasible to use a value of $2^{10}$ for the IFFT size.}
\vspace{0.5cm}

The occupied bandwidth of an OFDM signal is approximately
\[
    B \approx N\,\Delta c,
\]
where $N$ is the IFFT size (number of subcarriers) and $\Delta c$ is the subcarrier spacing.

Given:
\[
    B = 80~\text{MHz}, \quad \Delta c = 20~\text{kHz},
\]
we can estimate the required IFFT size as
\[
    N \approx \frac{B}{\Delta c} = \frac{80\times10^6}{20\times10^3} = 4000.
\]

If we choose $N = 2^{10} = 1024$, the occupied bandwidth would be
\[
    B \approx 1024 \times 20~\text{kHz} = 20.48~\text{MHz},
\]
which is much smaller than the required 80~MHz.

\textbf{Therefore, using $N = 2^{10}$ is not feasible because it would not cover the full 80~MHz data bandwidth.}
To span the available bandwidth, we would need an IFFT size close to $N \approx 4000$ (i.e., the next power of two $N = 4096$).

\vspace{1cm}
\textbf{Question: Assume an IFFT size of $2^{13}$. What would be the value of the sampling rate at the IFFT output?}
\vspace{0.5cm}

In an OFDM system, the sampling rate $F_s$ (at the IFFT output) is related to the IFFT size $N$ and the subcarrier spacing $\Delta c$ by:
\[
    F_s = N \, \Delta c.
\]

Given:
\[
    N = 2^{13} = 8192, \quad \Delta c = 20~\text{kHz},
\]
then
\[
    F_s = 8192 \times 20 \times 10^3 = 163.84 \times 10^6~\text{Hz} = \textbf{163.84~MHz}.
\]

Therefore, the sampling rate at the IFFT output is \textbf{163.84~MHz}.


\vspace{1cm}
\textbf{Question: What is the data rate (in Mbits/s) of this system, when all active subcarriers are used to carry data and a QPSK constellation is used?}
\vspace{0.5cm}

% Data Rate= $\frac{N \times Bits_s}{T_{OFDM}}$ 

% Since we have a QPSK constellation, we have $\log_2(4)= 2$ bits.

% The duration of a OFDM Symbol is $T_{OFDM} = (N + L_c)T$.
% $L_c T \ge \Delta_T$ $$L_c \ge \frac{\Delta\tau}{T} = \frac{6 \times 10^{-6} \text{ s}}{12.207 \times 10^{-9} \text{ s}} \approx 491.5 = 492$$

% $$T = \frac{1}{N \cdot \Delta_c} = \frac{1}{4096 \cdot 20 \times 10^3 \text{ Hz}} \approx 12.207 \times 10^{-9} \text{ s}$$
% $$T_{OFDM} = (4096 + 492) \cdot 12.207 \times 10^{-9} \text{ s} = 4588 \cdot 12.207 \times 10^{-9} \text{ s} \approx 55.99 \text{ µs}$$

% Whith this values and with $N = \frac{B}{\Delta_c} = \frac{80 \times10^6}{20\times 10^3}= 4000$, we have a data rate of $\frac{4000 \times 2}{55.99\times 10^{-6}} = 142.89 \text{ Mbits/s} $ 

Data Rate = $\frac{N \times \text{Bits}_s}{T_{OFDM}}$ 

Since we have a QPSK constellation, we have $\log_2(4) = 2$ bits per subcarrier.

The duration of an OFDM symbol is 
\[
T_{OFDM} = T_\text{symbol} + T_{CP},
\] 
where the useful symbol duration is 
\[
T_\text{symbol} = \frac{1}{\Delta c} = \frac{1}{20 \times 10^3 \text{ Hz}} \approx 50 \, \mu\text{s}.
\] 

The cyclic prefix must be at least as long as the maximum delay spread of the channel:
\[
T_{CP} \ge \Delta \tau = 6 \, \mu\text{s}.
\] 

Hence, the total OFDM symbol duration is 
\[
T_{OFDM} = T_\text{symbol} + T_{CP} \approx 50 + 6 = 56 \, \mu\text{s}.
\]

The number of active subcarriers is
\[
N = \frac{B}{\Delta c} = \frac{80 \times 10^6}{20 \times 10^3} = 4000.
\] 

Finally, the data rate is
\[
R = \frac{N \times \text{Bits}_s}{T_{OFDM}} = \frac{4000 \times 2}{56 \times 10^{-6}} \approx \textbf{142.86 Mbits/s}.
\]


\vspace{1cm}
\textbf{Question:  Repeat the previous point for a 16-QAM constellation when 100 of the active subcarriers are used to send training sequences to aid synchronization and channel estimation at the receiver.}
\vspace{0.5cm}

As its a 16-QAM constellation we have $\log_2(16) = 4 \text{bits per symbol}$.

The active subcarriers are the same, but we have 100 subcarriers used to send training data, so we have 3900 active subcarriers sending data.

\[
R = \frac{N \times \text{Bits}_s}{T_{OFDM}} = \frac{3900 \times 4}{56 \times 10^{-6}} \approx \textbf{278.57 Mbits/s}.
\]
