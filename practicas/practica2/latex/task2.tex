\textbf{An OFDM transceiver has 85 MHz of available RF bandwidth, of which 80 MHz can be occupied for data transmission, leaving $2.5$ MHz at each side as guard bands. The subcarrier spacing is $20$ kHz, and the delay spread of the channel almost never exceeds 6 $\mu$s.
}

\textbf{Question: What is the smallest possible CP overhead of this system, in percentage?}
\vspace{0.5cm}

$O{_CP} = \frac{L_cT}{NT} = \frac{L_c}{N}$

$\Delta_c = \frac{1}{NT} \rightarrow NT = \frac{1}{\Delta_c}= \frac{1}{20 \times 10^3} = 50 \mu s$

$O{_CP} = \frac{6\mu s}{50 \mu s} = 0.12$

\vspace{1cm}
\textbf{Question: Explain why it is not feasible to use a value of $2^{10}$ for the IFFT size.}
\vspace{0.5cm}

The occupied bandwidth of an OFDM signal is approximately
\[
    B \approx N\,\Delta f,
\]
where $N$ is the IFFT size (number of subcarriers) and $\Delta f$ is the subcarrier spacing.

Given:
\[
    B = 80~\text{MHz}, \quad \Delta f = 20~\text{kHz},
\]
we can estimate the required IFFT size as
\[
    N \approx \frac{B}{\Delta f} = \frac{80\times10^6}{20\times10^3} = 4000.
\]

If we choose $N = 2^{10} = 1024$, the occupied bandwidth would be
\[
    B \approx 1024 \times 20~\text{kHz} = 20.48~\text{MHz},
\]
which is much smaller than the required 80~MHz.

\textbf{Therefore, using $N = 2^{10}$ is not feasible because it would not cover the full 80~MHz data bandwidth.}
To span the available bandwidth, we would need an IFFT size close to $N \approx 4000$ (i.e., the next power of two $N = 4096$).

\vspace{1cm}
\textbf{Question: Assume an IFFT size of $2^{13}$. What would be the value of the sampling rate at the IFFT output?}
\vspace{0.5cm}

In an OFDM system, the sampling rate $F_s$ (at the IFFT output) is related to the IFFT size $N$ and the subcarrier spacing $\Delta f$ by:
\[
    F_s = N \, \Delta f.
\]

Given:
\[
    N = 2^{13} = 8192, \quad \Delta f = 20~\text{kHz},
\]
then
\[
    F_s = 8192 \times 20 \times 10^3 = 163.84 \times 10^6~\text{Hz} = \textbf{163.84~MHz}.
\]

Therefore, the sampling rate at the IFFT output is \textbf{163.84~MHz}.


\vspace{1cm}
\textbf{Question: What is the data rate (in Mbits/s) of this system, when all active subcarriers are used to carry data and a QPSK constellation is used?}
\vspace{0.5cm}

COMPLETAR

\vspace{1cm}
\textbf{Question:  Now assume that we decide to implement an IFFT with size $4 096$, and redo the previous two points.}
\vspace{0.5cm}
In an OFDM system, the sampling rate $F_s$ (at the IFFT output) is related to the IFFT size $N$ and the subcarrier spacing $\Delta f$ by:
\[
    F_s = N \, \Delta f.
\]

Given:
\[
    N = 4096, \quad \Delta f = 20~\text{kHz},
\]
then
\[
    F_s = 4096 \times 20 \times 10^3 = 81,92 \times 10^6~\text{Hz} = \textbf{81,92~MHz}.
\]

Therefore, the sampling rate at the IFFT output is \textbf{81,92~MHz}.

COMPLETAR