\textbf{Question: Is there any loss of information in the process described by (9)?}
\vspace{0.5cm}

No, there is no loss of information in the process described by (9).
The upsampling operation in \eqref{eq:upsample} simply inserts $(L-1)$ zeros between consecutive samples of $w[n]$.
Therefore, the original sequence $w[n]$ can be perfectly recovered from the upsampled sequence $\bar{w}[n]$
by taking one sample out of every $L$ samples (i.e., by discarding the inserted zeros).

\begin{equation} \label{eq:upsample}
    \bar{w}[n] =
    \begin{cases}
        w[n/L], & \text{if } n \text{ is an integer multiple of } L, \\[4pt]
        0,      & \text{otherwise.}
    \end{cases}
\end{equation}
\vspace{1cm}
\textbf{Question: Show that if the Discrete-Time Fourier Transform (DTFT) of $w[n]$ is $W(e^{jw})$, then the DTFT of..?
}
\vspace{0.5cm}

\vspace{1cm}
\noindent\textbf{Question:} Let $A>0$ and $\omega_0 = \frac{\pi}{2}$, and consider
\begin{equation}
    W(e^{j\omega}) = \begin{cases}
        A\left(1-\frac{\omega}{\omega_0}\right) & \text{if } |\omega| \leq \omega_0          \\
        0                                       & \text{if } \omega_0 \leq |\omega| \leq \pi
    \end{cases}
\end{equation}
Sketch $W(e^{j\omega})$, and also $\overline{W(e^{j\omega})}$ for $L=2$ and $L=3$.
\vspace{0.5cm}
\begin{figure}[H]
    \centering
    \includegraphics[width=.8\linewidth]{img/task1_3.png}
\end{figure}

\vspace{1cm}
\textbf{Question: For $L=2$ and $L=3$, sketch the transfer function $G_{\rm TX}(e^{j\omega})$ of the discrete-time filter $g_{\rm TX}[n]$.?
}
\vspace{0.5cm}
