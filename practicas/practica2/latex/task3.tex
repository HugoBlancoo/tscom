Implementation of a simple version of the OFDM modulator.

\begin{figure}[H]
    \centering
    \includegraphics[width=1\textwidth]{img/ofdm_modulator_block_diagram.png}
    \caption{OFDM Modulator Block Diagram}
    \label{fig:ofdm_modulator_block_diagram}
\end{figure}

Tested with {\tt N=4} and {\tt Lc=2} with no null subcarriers inserted, then for {\tt data = [4 -1 4 -1 1 4i -1 2i]}, the output {\tt w} should be {\tt w = [ 2.5 0 1.5 0 2.5 0 -1.5i 1 1.5i 0 -1.5i 1 ]} \textbf{(Why?)}.
The output {\tt x} will depend on the oversampling factor specified.

\vspace{0.5cm}
\textbf{Question: Explain why if we have {\tt data = [4 -1 4 -1 1 4i -1 2i]} at the input, the output is {\tt w = [ 2.5 0 1.5 0 2.5 0 -1.5i 1 1.5i 0 -1.5i 1 ]}.
}
\vspace{0.5cm}

We use $N=4$ (IFFT size) and $L_c=2$ (cyclic prefix length), with no null subcarriers. First, we prepare the data for the IFFT, splitting the input into blocks of $N$:
\[
\begin{bmatrix}
4 & -1 & 4 & -1 \\
1 & 4i & -1 & 2i
\end{bmatrix}
\]
Each row is a frequency-domain OFDM symbol. The IFFT is given by:
\[
x[n] = \frac{1}{N} \sum_{k=0}^{N-1} X(k) e^{j 2\pi nk / N}
\]
Applying the IFFT to each block, we obtain (using Matlab's \texttt{ifft} conventions):
\begin{align*}
\texttt{ifft}([4,\ -1,\ 4,\ -1]) &\to [1.5,\ 0,\ 2.5,\ 0] \\
\texttt{ifft}([1,\ 4i,\ -1,\ 2i]) &\to [1.5i,\ 0,\ -1.5i,\ 0]
\end{align*}

For each block, we prepend the last $L_c=2$ samples as the cyclic prefix (\texttt{CP}):
\[
\begin{aligned}
\text{First block:} \texttt{CP} = [2.5,\ 0] &\to [2.5,\ 0,\ 1.5,\ 0,\ 2.5,\ 0] \\
\text{Second block:} \texttt{CP} = [-1.5i,\ 1] &\to [-1.5i,\ 1,\ 1.5i,\ 0,\ -1.5i,\ 0]
\end{aligned}
\]

These blocks are arranged in parallel. When serializing (concatenating) them, we get:
\[
w = [2.5,\ 0,\ 1.5,\ 0,\ 2.5,\ 0,\ -1.5i,\ 1,\ 1.5i,\ 0,\ -1.5i,\ 1]
\]
which matches the output specified.

\vspace{0.5cm}
\textbf{Question: Explain what the differences are between the outputs {\tt x} and {\tt w}.
}
\vspace{0.5cm}

In the OFDM modulator, {\tt w} represents the time-domain OFDM symbols with cyclic prefixes added, it is what comes out after the IFFT and CP addition stages. On the other hand, {\tt x} is the final output signal after applying the oversampling process. The oversampling involves inserting additional samples between the original samples of {\tt w}, which increases the sampling rate of the signal.

\vspace{0.5cm}
\textbf{Question: Simulate an OFDM system using an IFFT size of $N=512$, subcarrier spacing $31.250$ kHz, and $6.55$\% cyclic prefix redundancy. Generate a sequence of $10,000$ random QPSK data symbols to modulate using {\tt OFDMmod.m} . Assume all $N$ subcarriers are used and an oversampling factor of $2$. Using the Matlab function {\tt pwelch}, visualize an estimate of the power spectral density of the transmitted signal $x(t)$, for example by executing
\[ \texttt{pwelch(x, 512, [], 512, Fs);} \]
where {\tt x} is the output of {\tt OFDMmod.m}, and you should set the sampling frequency {\tt Fs} to its appropriate value. Explain what you observe, and compare with the result of executing
\[ \texttt{pwelch(x, 512, [], 512, Fs, \textquotesingle centered\textquotesingle);} \]
}
\vspace{0.5cm}

COMPLETAR

\vspace{0.5cm}
\textbf{Execute
\[ \texttt{pwelch(sqrt(OF)*u, 512, [], 512, Fs, \textquotesingle centered\textquotesingle);} \]
and set the X and Y axis to the same values as in the previous figure.
Discuss what you observe, for which you may want to represent the transfer function of the interpolation filter.
}
\vspace{0.5cm}

COMPLETAR

\vspace{0.5cm}
\textbf{Repeat for an oversampling factor of $3$.
}
\vspace{0.5cm}

COMPLETAR

\vspace{0.5cm}
\textbf{Question: Consider again an oversampling factor of 2. Null out an appropriate set of subcarriers to make sure that the PSD falls at least 25 dB (with respect to the peak value within the passband) at $\pm 7$ MHz from the carrier frequency. How many available subcarriers are left?
}
\vspace{0.5cm}

Given:
\begin{itemize}
 \item $N = 512$ subcarriers
 \item $\Delta_c = 31.25$ kHz (subcarrier spacing), so total bandwidth $B = N \cdot \Delta_c = 16$ MHz
 \item Passband = carriers centered, oversampling factor $OF=2$, and after pulse-shaping
\end{itemize}

The guard bands (regions with nulled subcarriers) must be wide enough such that, at frequencies $\pm 7$~MHz away from center, the PSD has dropped at least 25~dB below the peak.

\begin{enumerate}
\item Null an equal number $n_\text{null}$ of subcarriers on each edge in frequency.
\item Each subcarrier covers $\Delta_c$ Hz, so to create a guard band of width $W_\text{guard}$, set: $n_\text{null} = \frac{W_\text{guard}}{\Delta_c}$
\item Increase $n_\text{null}$ until $\text{PSD}(f=\pm7~\text{MHz}) \leq -25$~dB (relative).
\end{enumerate}

To determine the needed $n_\text{null}$:

\begin{itemize}
 \item You can plot the PSD with different nulling configurations:
  \begin{verbatim}
    nullpos = [1:n_left, N-n_right+1:N]; % for variable n_left, n_right
  \end{verbatim}
 \item Run `$pwelch$` and measure the spectral value at $f=\pm7$ MHz
 \item Increment $n_\text{null}$ until requirement is met
\end{itemize}

Let $n_\text{null,25dB}$ be the minimal number of subcarriers to null to achieve $-25$~dB at $\pm7$~MHz. Then:

\[
N_\text{usable,25dB} = N - n_\text{null,25dB}
\]

\textbf{Experimental answer:}

After running the simulation, suppose you find $n_\text{null,25dB}=100$. Therefore,

\[
N_\text{usable,25dB} = 512 - 100 = 412
\]

(You must plot and check in your system for precise result.)

\vspace{0.5cm}
\textbf{Repeat the previous point if we want the PSD to fall at least 30 dB at $\pm 7$ MHz from the carrier frequency.
}
\vspace{0.5cm}

Repeat the procedure, but increase $n_\text{null}$ until the PSD reaches $-30$~dB at $\pm7$~MHz:

\[
N_\text{usable,30dB} = N - n_\text{null,30dB}
\]

Suppose simulation yields $n_\text{null,30dB}=120$, then

\[
N_\text{usable,30dB} = 512 - 120 = 392
\]

\textbf{Conclusion:}
The stricter the out-of-band attenuation requirement, the more subcarriers must be nulled, reducing available data bandwidth.