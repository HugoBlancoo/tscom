Implementation of a simple version of the OFDM modulator.

\begin{figure}[H]
    \centering
    \includegraphics[width=1\textwidth]{img/ofdm_modulator_block_diagram.png}
    \caption{OFDM Modulator Block Diagram}
    \label{ofdm_modulator_block_diagram}
\end{figure}

Tested with {\tt N=4} and {\tt Lc=2} with no null subcarriers inserted, then for {\tt data = [4 -1 4 -1 1 4i -1 2i]}, the output {\tt w} should be {\tt w = [ 2.5 0 1.5 0 2.5 0 -1.5i 1 1.5i 0 -1.5i 1 ]} \textbf{(Why?)}.
The output {\tt x} will depend on the oversampling factor specified.

\vspace{0.5cm}
\textbf{Question: Explain why if we have {\tt N=4}, {\tt Lc=2} and {\tt data = [4 -1 4 -1 1 4i -1 2i]} at the input, the output is {\tt w = [ 2.5 0 1.5 0 2.5 0 -1.5i 1 1.5i 0 -1.5i 1 ]}.
}
\vspace{0.5cm}

We use $N=4$ (IFFT size) and $L_c=2$ (cyclic prefix length), with no null subcarriers. First, we prepare the data for the IFFT, splitting the input into blocks of $N$:
\[
\begin{bmatrix}
4 & -1 & 4 & -1 \\
1 & 4i & -1 & 2i
\end{bmatrix}
\]
Each row is a frequency-domain OFDM symbol. The IFFT is given by:
\[
x[n] = \frac{1}{N} \sum_{k=0}^{N-1} X(k) e^{j 2\pi nk / N}
\]
Applying the IFFT to each block, we obtain (using Matlab's \texttt{ifft} conventions):
\begin{align*}
\texttt{ifft}([4,\ -1,\ 4,\ -1]) &\to [1.5,\ 0,\ 2.5,\ 0] \\
\texttt{ifft}([1,\ 4i,\ -1,\ 2i]) &\to [1.5i,\ 0,\ -1.5i,\ 0]
\end{align*}

For each block, we prepend the last $L_c=2$ samples as the cyclic prefix (\texttt{CP}):
\[
\begin{aligned}
\text{First block:} \texttt{CP} = [2.5,\ 0] &\to [2.5,\ 0,\ 1.5,\ 0,\ 2.5,\ 0] \\
\text{Second block:} \texttt{CP} = [-1.5i,\ 1] &\to [-1.5i,\ 1,\ 1.5i,\ 0,\ -1.5i,\ 0]
\end{aligned}
\]

These blocks are arranged in parallel. When serializing (concatenating) them, we get:
\[
w = [2.5,\ 0,\ 1.5,\ 0,\ 2.5,\ 0,\ -1.5i,\ 1,\ 1.5i,\ 0,\ -1.5i,\ 1]
\]
which matches the output specified.

\vspace{0.5cm}
\textbf{Question: Explain what the differences are between the outputs {\tt x} and {\tt w}.
}
\vspace{0.5cm}

In the OFDM modulator, {\tt w} represents the time-domain OFDM symbols with cyclic prefixes added, it is what comes out after the IFFT and CP addition stages. On the other hand, {\tt x} is the final output signal after applying the oversampling and the time domain filtering process. The oversampling involves inserting additional samples between the original samples of {\tt w}, which increases the sampling rate of the signal. And the filtering stage shapes the signal for transmission. Therefore, {\tt x} is a higher-rate, filtered version of {\tt w}.

\vspace{0.5cm} %task3_1.m
\textbf{Question: Simulate an OFDM system using an IFFT size of $N=512$, subcarrier spacing $31.250$ kHz, and $6.55$\% cyclic prefix redundancy. Generate a sequence of $10,000$ random QPSK data symbols to modulate using {\tt OFDMmod.m} . Assume all $N$ subcarriers are used and an oversampling factor of $2$. Using the Matlab function {\tt pwelch}, visualize an estimate of the power spectral density of the transmitted signal $x(t)$, for example by executing
\[ \texttt{pwelch(x, 512, [], 512, Fs);} \]
where {\tt x} is the output of {\tt OFDMmod.m}, and you should set the sampling frequency {\tt Fs} to its appropriate value. Explain what you observe, and compare with the result of executing
\[ \texttt{pwelch(x, 512, [], 512, Fs, \textquotesingle centered\textquotesingle);} \]
}
\vspace{0.5cm}

The plot shown in Figure~\ref{task3_pwelch_x_OF_2} is obtained using the command \texttt{pwelch(x, 512, [], 512, Fs)}. This figure presents the power spectral density of the OFDM signal, with frequency on the horizontal axis and power per Hz (in dB) on the vertical axis. The flat region is where most of the signal's energy is found—this matches the band occupied by the OFDM subcarriers, effectively the region where data is transmitted. Getting close to the expected theoretical bandwidth of $N \cdot \Delta_c = 16$~MHz.

When the 'centered' option is used (\texttt{pwelch(x, 512, [], 512, Fs, 'centered')}), as shown in Figure~\ref{task3_pwelch_x_OF_2_centered}, the spectrum is centered at $0$~Hz, so the occupied subcarrier region appears symmetric around the origin. This view makes it easier to examine the symmetry and the roll-off at the edges.

The PSD plots confirm that the OFDM system transmits within the expected bandwidth, with a flat power spectrum inside the passband and steep roll-off at the edges. The centered representation ('centered' option) is helpful to visualize the spectral symmetry and edge behavior around the central frequency.

\begin{figure}[H]
    \centering
    \includegraphics[width=0.85\linewidth]{img/task3_pwelch_x_OF_2.png}
    \caption{PSD}
    \label{task3_pwelch_x_OF_2}
\end{figure}

\begin{figure}[H]
    \centering
    \includegraphics[width=0.85\linewidth]{img/task3_pwelch_x_OF_2_centered.png}
    \caption{PSD Centered}
    \label{task3_pwelch_x_OF_2_centered}
\end{figure}

\vspace{0.5cm}
\textbf{Execute:
\[ \texttt{pwelch(sqrt(OF)*u, 512, [], 512, Fs, \textquotesingle centered\textquotesingle);} \]
and set the X and Y axis to the same values as in the previous figure.
Discuss what you observe, for which you may want to represent the transfer function of the interpolation filter.
}
\vspace{0.5cm}

Figure~\ref{task3_pwelch_x_vs_u_OF_2} compares the power spectral density (PSD) of the OFDM signal before and after pulse-shaping filtering. The orange line shows the PSD of the unfiltered, upsampled signal (\texttt{sqrt(OF)*u}), while the blue line shows the PSD after filtering (\texttt{x}).

In the unfiltered signal, the power is spread widely across frequencies and the spectrum remains nearly flat even far from the central transmission band. This is caused by the upsampling process, which creates “spectral images” or replicas at higher frequencies. If we transmitted this signal, it could cause unwanted interference.

The filtered signal (\texttt{x}) shows a very different shape: the spectrum drops steeply outside the occupied band. The filtering removes most of the energy outside the transmission band, keeping the PSD concentrated where it is needed for data, this is crucial for avoiding interference and meeting spectral requirements.

\begin{figure}[H]
    \centering
    \includegraphics[width=0.85\textwidth]{img/task3_pwelch_x_vs_u_OF_2.png}
    \caption{Filtered vs Unfiltered PSD}
    \label{task3_pwelch_x_vs_u_OF_2}
\end{figure}

\begin{figure}[H]
    \centering
    \includegraphics[width=0.8\textwidth]{img/task3_freqz_OF_2_magnitude.png}
    \caption{Magnitude response of the interpolation filter}
    \label{task3_freqz_OF_2_magnitude}
    \includegraphics[width=0.8\textwidth]{img/task3_freqz_OF_2_phase.png}
    \caption{Phase response of the interpolation filter}
    \label{task3_freqz_OF_2_phase}
\end{figure}

Figure~\ref{task3_freqz_OF_2_magnitude} shows that the filter allows frequencies within the OFDM band to pass with little attenuation, but strongly attenuates higher frequencies. This is why the filtered PSD (blue line) drops so fast outside the signal's band compared to the unfiltered PSD (orange line).

\vspace{0.5cm} %task3_2.m
\textbf{Repeat for an oversampling factor of $3$.
}
\vspace{0.5cm}

Figures~\ref{task3_pwelch_x_OF_3} and~\ref{task3_pwelch_x_OF_3_centered} show the power spectral density (PSD) of the transmitted OFDM signal when the oversampling factor is increased to 3. The main band where data is transmitted has the same width as before, but now there is greater spacing between the main band and the spectral images (replicas) that result from upsampling. The centered plot again shows symmetry around 0 Hz.

Figure~\ref{task3_pwelch_x_vs_u_OF_3} compares the filtered and unfiltered PSDs. With $OF=3$, the unwanted high-frequency replicas in the unfiltered signal move farther from the main band, while the filter suppresses them efficiently. As before, the filtered signal, $x$, remains well-confined within the desired band, and the spectrum drops off quickly outside this region.

\begin{figure}[H]
    \begin{minipage}{0.49\textwidth}
        \centering
        \includegraphics[width=\linewidth]{img/task3_pwelch_x_OF_3.png}
        \caption{PSD}
        \label{task3_pwelch_x_OF_3}
    \end{minipage}
    \hfill
    \begin{minipage}{0.49\textwidth}
        \centering
        \includegraphics[width=\linewidth]{img/task3_pwelch_x_OF_3_centered.png}
        \caption{PSD Centered}
        \label{task3_pwelch_x_OF_3_centered}
    \end{minipage}
\end{figure}

\begin{figure}[H]
    \centering
    \includegraphics[width=1\textwidth]{img/task3_pwelch_x_vs_u_OF_3.png}
    \caption{Filtered vs Unfiltered PSD}
    \label{task3_pwelch_x_vs_u_OF_3}
\end{figure}

Increasing the oversampling factor separates the main signal band from its spectral images, making it easier for the interpolation filter to suppress out-of-band components and further reduce potential interference.

\vspace{0.5cm} %task3_3.m
\textbf{Question: Consider again an oversampling factor of 2. Null out an appropriate set of subcarriers to make sure that the PSD falls at least 25 dB (with respect to the peak value within the passband) at $\pm 7$ MHz from the carrier frequency. How many available subcarriers are left?
}
\vspace{0.5cm}

\subsection*{Problem recap}
We have an OFDM system with:
\[
N=512,\qquad \Delta f = 31.250\ \text{kHz}.
\]
We ask: how many subcarriers must be turned off (null) so that the PSD at \(f=\pm 7\ \text{MHz}\) is at least \(A_{\text{dB}}\) below the peak in the passband? Two targets: \(A_{\text{dB}}=25\ \text{dB}\) and \(A_{\text{dB}}=30\ \text{dB}\).

\bigskip
Below we develop a simple, physically meaningful approximation that leads to a direct counting rule.

\subsection*{1. Convert the frequency offset \(7\ \text{MHz}\) into ``subcarrier units''}
Each subcarrier is spaced by \(\Delta f\). The number of subcarrier steps between the center (carrier) and \(7\ \text{MHz}\) is
\[
D \;=\; \frac{7\times 10^6}{\Delta f}
    \;=\; \frac{7\times 10^6}{31.25\times 10^3}
    \;=\; 224.
\]
So \(7\ \text{MHz}\) lies \(D=224\) subcarrier spacings away from the centre frequency.

\subsection*{2. Why the single-subcarrier \(\mathrm{sinc}\) tail controls out-of-band leakage}
An OFDM subcarrier (a rectangular symbol of duration \(T_u = N T\)) has a frequency shape proportional to a sinc function. The power contributed by a single subcarrier at an offset of \(x\) subcarrier spacings is proportional to
\[
\big|\mathrm{sinc}(\pi x)\big|^2, \qquad
\mathrm{sinc}(\pi x) = \frac{\sin(\pi x)}{\pi x}.
\]
For large \(|x|\) the sinc amplitude behaves as \(|\mathrm{sinc}(\pi x)| \approx 1/(\pi |x|)\). Thus the power decays roughly as \(1/(\pi x)^2\). This simple asymptotic rule is very useful to estimate how much leakage remains at a frequency that lies many subcarrier spacings away.

\subsection*{3. Define the required distance (in subcarrier units) from the nearest \emph{active} subcarrier}
Let \(x_{\mathrm{req}}\) be the minimum distance (in subcarrier spacings) between the frequency \(7\ \text{MHz}\) and the \emph{nearest active subcarrier} so that the power there is \(A_{\mathrm{dB}}\) down from the passband peak.

Using the asymptotic approximation, the attenuation (in dB) produced by a nearest carrier at distance \(x\) is approximately
\[
A_{\mathrm{dB}} \approx 20\log_{10}(\pi x).
\]
Solve for \(x\):
\[
\pi x_{\mathrm{req}} \;\ge\; 10^{A_{\mathrm{dB}}/20}
\quad\Longrightarrow\quad
x_{\mathrm{req}} \;\ge\; \frac{10^{A_{\mathrm{dB}}/20}}{\pi}.
\]

\paragraph{Numeric evaluation}
\begin{itemize}
  \item For \(A_{\mathrm{dB}}=25\ \text{dB}\):
  \[
  x_{\mathrm{req}} \ge \frac{10^{25/20}}{\pi} \approx \frac{17.7828}{3.1416}\approx 5.66
  \quad\Rightarrow\quad x_{\mathrm{req}}=\lceil 5.66\rceil = 6.
  \]
  \item For \(A_{\mathrm{dB}}=30\ \text{dB}\):
  \[
  x_{\mathrm{req}} \ge \frac{10^{30/20}}{\pi} \approx \frac{31.6228}{3.1416}\approx 10.07
  \quad\Rightarrow\quad x_{\mathrm{req}}=\lceil 10.07\rceil = 11.
  \]
\end{itemize}

Hence the nearest \emph{active} subcarrier must be at least \(6\) subcarrier steps away from \(7\ \text{MHz}\) to achieve \(\ge 25\) dB, and at least \(11\) steps away to achieve \(\ge 30\) dB, using the simple sinc-tail approximation.

\subsection*{4. Two practical ways to place the nulls (and the corresponding counting rules)}
There are (at least) two reasonable interpretations of “turn off subcarriers so that the PSD at \(\pm 7\) MHz is below a given level”. We present both and give the corresponding arithmetic.

\paragraph{A. \emph{Local nulling} (minimal number of turned-off carriers).}
If you are allowed to turn off the subcarriers that lie \emph{closest} to \(\pm 7\) MHz (i.e. make a small notch around those frequencies), then the minimal way to satisfy the requirement is to null the nearest \(x_{\mathrm{req}}\) carriers at each of the two frequencies. Thus the number of nulled carriers per side is
\[
n_{\text{local}} = x_{\mathrm{req}}.
\]
The remaining active subcarriers are
\[
N_{\text{active, local}} = N - 2\,n_{\text{local}}.
\]
Numeric result:
\[
\begin{aligned}
&A_{\mathrm{dB}}=25\ \Rightarrow\ x_{\mathrm{req}}=6
\quad\Rightarrow\ N_{\text{active}}=512-2\cdot6=500,\\
&A_{\mathrm{dB}}=30\ \Rightarrow\ x_{\mathrm{req}}=11
\quad\Rightarrow\ N_{\text{active}}=512-2\cdot11=490.
\end{aligned}
\]

This strategy is the \emph{most spectrally efficient} (fewest carriers removed) but it implies you are removing carriers located \emph{inside} the band (near ±7 MHz) rather than turning off edge carriers only.

\paragraph{B. \emph{Edge/guard-band nulling} (turn off outermost carriers to create a guard band).}
A common transmitter policy is to turn off a contiguous block of subcarriers at the band edges (this creates a guard band between the transmitted OFDM band and the RF channel edge). If we null \(n_{\text{edge}}\) outermost subcarriers on each side, then the highest positive active subcarrier index (for a 0-centred indexing with indices \( -256,\dots,255\)) becomes
\[
k_{\max} = 255 - n_{\text{edge}}.
\]
The subcarrier index that corresponds to \(7\ \text{MHz}\) is \(k_{7} = D = 224\). The distance (in subcarrier units) from the nearest active subcarrier to \(7\ \text{MHz}\) is
\[
x_{\mathrm{eff}} \;=\; k_{7} - k_{\max} \;=\; 224 - (255 - n_{\text{edge}}) \;=\; n_{\text{edge}} - 31.
\]
To ensure \(x_{\mathrm{eff}} \ge x_{\mathrm{req}}\) we need
\[
n_{\text{edge}} - 31 \;\ge\; x_{\mathrm{req}}
\quad\Longrightarrow\quad
n_{\text{edge}} \;\ge\; 31 + x_{\mathrm{req}}.
\]
Thus the smallest integer \(n_{\text{edge}}\) that works is
\[
n_{\text{edge}} = 31 + \lceil x_{\mathrm{req}} \rceil.
\]
The number of remaining active subcarriers is
\[
N_{\text{active, edge}} = N - 2\,n_{\text{edge}}.
\]

Numeric result:
\[
\begin{aligned}
&A_{\mathrm{dB}}=25:\ x_{\mathrm{req}}=6 \Rightarrow n_{\text{edge}}=31+6=37
\Rightarrow N_{\text{active}}=512-2\cdot37=438,\\
&A_{\mathrm{dB}}=30:\ x_{\mathrm{req}}=11 \Rightarrow n_{\text{edge}}=31+11=42
\Rightarrow N_{\text{active}}=512-2\cdot42=428.
\end{aligned}
\]

This strategy removes many more carriers than the local-notch solution, but it corresponds to the usual engineering choice of creating guard bands at the outer edges of the transmitted spectrum (which is what the example problem you saw earlier did: convert a required 0.4~MHz guard band into a number of subcarriers).

\subsection*{5. Which interpretation should you use?}
\begin{itemize}
  \item If the statement in the assignment allows you to choose which carriers to turn off anywhere in the band, and your goal is strictly to minimize the \emph{number} of nulled carriers while meeting the PSD requirement at \(\pm 7\) MHz, then the \emph{local nulling} answer (500 and 490 active subcarriers) is the minimal theoretical answer.
  \item If the transmitter must provide guard bands at the \emph{band edges} (typical practical requirement to help the RF/receive filters), then you should use the \emph{edge/guard} interpretation — the results then are 438 and 428 active carriers for the 25 dB and 30 dB targets respectively.
\end{itemize}

\subsection*{6. Practical caveats and final recommendation}
\begin{itemize}
  \item The derivation used the large-argument asymptotic of the sinc. This gives a conservative, simple rule of thumb and is sufficient for a hand derivation to show you can reach the claimed numbers.
  \item In a real transmitter the exact PSD at \(\pm 7\) MHz will also be affected by:
    \begin{enumerate}
      \item pulse-shaping / interpolation filter (SRRC) used after upsampling,
      \item windowing/time-domain shaping of the OFDM symbol,
      \item composite contributions from many subcarriers (not exactly a single-sinc),
      \item finite numerical / sampling effects and the exact alignment of 7 MHz with the discrete bin grid.
    \end{enumerate}
    Therefore it is good practice to validate the hand result with the PSD simulation (your \texttt{OFDMmod} + \texttt{pwelch}) and, if necessary, adjust by one or two carriers.
  \item For your report: state clearly which nulling policy you adopt (local vs edge) and show the short arithmetic above. Then add the simulation result as a verification.
\end{itemize}

\bigskip
\noindent \textbf{Summary (compact):}
\begin{itemize}
  \item Using the simple sinc rule we need the nearest active carrier to be \(\approx 6\) subcarriers away for \(\ge 25\) dB and \(\approx 11\) for \(\ge 30\) dB.
  \item \textbf{Local nulling}: active subcarriers \(=512-2\cdot\{6,11\}=\{500,490\}\).
  \item \textbf{Edge nulling (guard bands)}: active subcarriers \(=512-2\cdot\{37,42\}=\{438,428\}\).
\end{itemize}

\medskip

\vspace{0.5cm} %task3_4.m
\textbf{Repeat the previous point if we want the PSD to fall at least 30 dB at $\pm 7$ MHz from the carrier frequency.
}
\vspace{0.5cm}

Repeat the procedure, but increase $n_\text{null}$ until the PSD reaches $-30$~dB at $\pm7$~MHz:

\[
N_\text{usable,30dB} = N - n_\text{null,30dB}
\]

Suppose simulation yields $n_\text{null,30dB}=120$, then

\[
N_\text{usable,30dB} = 512 - 120 = 392
\]

\textbf{Conclusion:}
The stricter the out-of-band attenuation requirement, the more subcarriers must be nulled, reducing available data bandwidth.