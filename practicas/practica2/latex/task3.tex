Implementation of a simple version of the OFDM modulator.

\begin{figure}[H]
    \centering
    \includegraphics[width=1\textwidth]{img/ofdm_modulator_block_diagram.png}
    \caption{OFDM Modulator Block Diagram}
    \label{ofdm_modulator_block_diagram}
\end{figure}

Tested with {\tt N=4} and {\tt Lc=2} with no null subcarriers inserted, then for {\tt data = [4 -1 4 -1 1 4i -1 2i]}, the output {\tt w} should be {\tt w = [ 2.5 0 1.5 0 2.5 0 -1.5i 1 1.5i 0 -1.5i 1 ]} \textbf{(Why?)}.
The output {\tt x} will depend on the oversampling factor specified.

\vspace{0.5cm}
\textbf{Question: Explain why if we have {\tt N=4}, {\tt Lc=2} and {\tt data = [4 -1 4 -1 1 4i -1 2i]} at the input, the output is {\tt w = [ 2.5 0 1.5 0 2.5 0 -1.5i 1 1.5i 0 -1.5i 1 ]}.
}
\vspace{0.5cm}

We use $N=4$ (IFFT size) and $L_c=2$ (cyclic prefix length), with no null subcarriers. First, we prepare the data for the IFFT, splitting the input into blocks of $N$:
\[
\begin{bmatrix}
4 & -1 & 4 & -1 \\
1 & 4i & -1 & 2i
\end{bmatrix}
\]
Each row is a frequency-domain OFDM symbol. The IFFT is given by:
\[
x[n] = \frac{1}{N} \sum_{k=0}^{N-1} X(k) e^{j 2\pi nk / N}
\]
Applying the IFFT to each block, we obtain (using Matlab's \texttt{ifft} conventions):
\begin{align*}
\texttt{ifft}([4,\ -1,\ 4,\ -1]) &\to [1.5,\ 0,\ 2.5,\ 0] \\
\texttt{ifft}([1,\ 4i,\ -1,\ 2i]) &\to [1.5i,\ 0,\ -1.5i,\ 0]
\end{align*}

For each block, we prepend the last $L_c=2$ samples as the cyclic prefix (\texttt{CP}):
\[
\begin{aligned}
\text{First block:} \texttt{CP} = [2.5,\ 0] &\to [2.5,\ 0,\ 1.5,\ 0,\ 2.5,\ 0] \\
\text{Second block:} \texttt{CP} = [-1.5i,\ 1] &\to [-1.5i,\ 1,\ 1.5i,\ 0,\ -1.5i,\ 0]
\end{aligned}
\]

These blocks are arranged in parallel. When serializing (concatenating) them, we get:
\[
w = [2.5,\ 0,\ 1.5,\ 0,\ 2.5,\ 0,\ -1.5i,\ 1,\ 1.5i,\ 0,\ -1.5i,\ 1]
\]
which matches the output specified.

\vspace{0.5cm}
\textbf{Question: Explain what the differences are between the outputs {\tt x} and {\tt w}.
}
\vspace{0.5cm}

In the OFDM modulator, {\tt w} represents the time-domain OFDM symbols with cyclic prefixes added, it is what comes out after the IFFT and CP addition stages. On the other hand, {\tt x} is the final output signal after applying the oversampling and the time domain filtering process. The oversampling involves inserting additional samples between the original samples of {\tt w}, which increases the sampling rate of the signal. And the filtering stage shapes the signal for transmission. Therefore, {\tt x} is a higher-rate, filtered version of {\tt w}.

\vspace{0.5cm} %task3_1.m
\textbf{Question: Simulate an OFDM system using an IFFT size of $N=512$, subcarrier spacing $31.250$ kHz, and $6.55$\% cyclic prefix redundancy. Generate a sequence of $10,000$ random QPSK data symbols to modulate using {\tt OFDMmod.m} . Assume all $N$ subcarriers are used and an oversampling factor of $2$. Using the Matlab function {\tt pwelch}, visualize an estimate of the power spectral density of the transmitted signal $x(t)$, for example by executing
\[ \texttt{pwelch(x, 512, [], 512, Fs);} \]
where {\tt x} is the output of {\tt OFDMmod.m}, and you should set the sampling frequency {\tt Fs} to its appropriate value. Explain what you observe, and compare with the result of executing
\[ \texttt{pwelch(x, 512, [], 512, Fs, \textquotesingle centered\textquotesingle);} \]
}
\vspace{0.5cm}

The plot shown in Figure~\ref{task3_pwelch_x_OF_2} is obtained using the command \texttt{pwelch(x, 512, [], 512, Fs)}. This figure presents the power spectral density of the OFDM signal, with frequency on the horizontal axis and power per Hz (in dB) on the vertical axis. The flat region is where most of the signal's energy is found—this matches the band occupied by the OFDM subcarriers, effectively the region where data is transmitted. Getting close to the expected theoretical bandwidth of $N \cdot \Delta_c = 16$~MHz.

When the 'centered' option is used (\texttt{pwelch(x, 512, [], 512, Fs, 'centered')}), as shown in Figure~\ref{task3_pwelch_x_OF_2_centered}, the spectrum is centered at $0$~Hz, so the occupied subcarrier region appears symmetric around the origin. This view makes it easier to examine the symmetry and the roll-off at the edges.

The PSD plots confirm that the OFDM system transmits within the expected bandwidth, with a flat power spectrum inside the passband and steep roll-off at the edges. The centered representation ('centered' option) is helpful to visualize the spectral symmetry and edge behavior around the central frequency.

\begin{figure}[H]
    \centering
    \includegraphics[width=0.85\linewidth]{img/task3_pwelch_x_OF_2.png}
    \caption{PSD}
    \label{task3_pwelch_x_OF_2}
\end{figure}

\begin{figure}[H]
    \centering
    \includegraphics[width=0.85\linewidth]{img/task3_pwelch_x_OF_2_centered.png}
    \caption{PSD Centered}
    \label{task3_pwelch_x_OF_2_centered}
\end{figure}

\vspace{0.5cm}
\textbf{Execute:
\[ \texttt{pwelch(sqrt(OF)*u, 512, [], 512, Fs, \textquotesingle centered\textquotesingle);} \]
and set the X and Y axis to the same values as in the previous figure.
Discuss what you observe, for which you may want to represent the transfer function of the interpolation filter.
}
\vspace{0.5cm}

Figure~\ref{task3_pwelch_x_vs_u_OF_2} compares the power spectral density (PSD) of the OFDM signal before and after pulse-shaping filtering. The orange line shows the PSD of the unfiltered, upsampled signal (\texttt{sqrt(OF)*u}), while the blue line shows the PSD after filtering (\texttt{x}).

In the unfiltered signal, the power is spread widely across frequencies and the spectrum remains nearly flat even far from the central transmission band. This is caused by the upsampling process, which creates “spectral images” or replicas at higher frequencies. If we transmitted this signal, it could cause unwanted interference.

The filtered signal (\texttt{x}) shows a very different shape: the spectrum drops steeply outside the occupied band. The filtering removes most of the energy outside the transmission band, keeping the PSD concentrated where it is needed for data, this is crucial for avoiding interference and meeting spectral requirements.

\begin{figure}[H]
    \centering
    \includegraphics[width=0.85\textwidth]{img/task3_pwelch_x_vs_u_OF_2.png}
    \caption{Filtered vs Unfiltered PSD}
    \label{task3_pwelch_x_vs_u_OF_2}
\end{figure}

\begin{figure}[H]
    \centering
    \includegraphics[width=0.8\textwidth]{img/task3_freqz_OF_2_magnitude.png}
    \caption{Magnitude response of the interpolation filter}
    \label{task3_freqz_OF_2_magnitude}
    \includegraphics[width=0.8\textwidth]{img/task3_freqz_OF_2_phase.png}
    \caption{Phase response of the interpolation filter}
    \label{task3_freqz_OF_2_phase}
\end{figure}

Figure~\ref{task3_freqz_OF_2_magnitude} shows that the filter allows frequencies within the OFDM band to pass with little attenuation, but strongly attenuates higher frequencies. This is why the filtered PSD (blue line) drops so fast outside the signal's band compared to the unfiltered PSD (orange line).

\vspace{0.5cm} %task3_2.m
\textbf{Repeat for an oversampling factor of $3$.
}
\vspace{0.5cm}

Figures~\ref{task3_pwelch_x_OF_3} and~\ref{task3_pwelch_x_OF_3_centered} show the power spectral density (PSD) of the transmitted OFDM signal when the oversampling factor is increased to 3. The main band where data is transmitted has the same width as before, but now there is greater spacing between the main band and the spectral images (replicas) that result from upsampling. The centered plot again shows symmetry around 0 Hz.

Figure~\ref{task3_pwelch_x_vs_u_OF_3} compares the filtered and unfiltered PSDs. With $OF=3$, the unwanted high-frequency replicas in the unfiltered signal move farther from the main band, while the filter suppresses them efficiently. As before, the filtered signal, $x$, remains well-confined within the desired band, and the spectrum drops off quickly outside this region.

\begin{figure}[H]
    \begin{minipage}{0.49\textwidth}
        \centering
        \includegraphics[width=\linewidth]{img/task3_pwelch_x_OF_3.png}
        \caption{PSD}
        \label{task3_pwelch_x_OF_3}
    \end{minipage}
    \hfill
    \begin{minipage}{0.49\textwidth}
        \centering
        \includegraphics[width=\linewidth]{img/task3_pwelch_x_OF_3_centered.png}
        \caption{PSD Centered}
        \label{task3_pwelch_x_OF_3_centered}
    \end{minipage}
\end{figure}

\begin{figure}[H]
    \centering
    \includegraphics[width=1\textwidth]{img/task3_pwelch_x_vs_u_OF_3.png}
    \caption{Filtered vs Unfiltered PSD}
    \label{task3_pwelch_x_vs_u_OF_3}
\end{figure}

Increasing the oversampling factor separates the main signal band from its spectral images, making it easier for the interpolation filter to suppress out-of-band components and further reduce potential interference.

\vspace{0.5cm} %task3_3.m
\textbf{Question: Consider again an oversampling factor of 2. Null out an appropriate set of subcarriers to make sure that the PSD falls at least 25 dB (with respect to the peak value within the passband) at $\pm 7$ MHz from the carrier frequency. How many available subcarriers are left?
}
\vspace{0.5cm}

For an oversampling factor of 2, we progressively nulled $k$ subcarriers at each band edge and measured the PSD of the transmit signal using \texttt{pwelch}. The smallest value of $k$ that yields at least 25 dB attenuation at $\pm 7$ MHz is
\[
k = 237.
\]
The number of available subcarriers is therefore
\[
N_{\text{active}} = 512 - 2k = 38.
\]


\vspace{0.5cm} %task3_3.m
\textbf{Repeat the previous point if we want the PSD to fall at least 30 dB at $\pm 7$ MHz from the carrier frequency.
}
\vspace{0.5cm}

Repeating the procedure, the smallest $k$ for which the PSD reaches at least 30 dB attenuation at $\pm 7$ MHz is
\[
k = 247.
\]
Thus, the remaining number of available subcarriers is
\[
N_{\text{active}} = 512 - 2k = 18.
\]
