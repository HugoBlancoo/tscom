\textbf{Assume a full-scale sinusoidal input with $f_0 = 37.1094 MHz$, and let the FFT size be M = 1024.
    Generate $15 \cdot M$ samples of $x(t)$ (at fs = 100 MHz) and quantize them to N = 12 bits. Break
    the vector xq of quantized samples into 15 size-M blocks using, e.g., the command reshape:}

\vspace{0.5cm}

\begin{lstlisting}[language=Matlab]
    xqblocks = reshape(xq, M, 15);
\end{lstlisting}

\textbf{so that each column of the $M \times 15$ matrix xqblocks will contain the corresponding block of size
    M . Now, since the fft command computes the FFT columnwise, in order to apply an M -point
    FFT to each block, we simply make
}
\begin{lstlisting}[language=Matlab]
    X = fft(xqblocks, M);
\end{lstlisting}

\textbf{Average the squared magnitude of the DFT coefficients over the 15 blocks and plot the results
    between 0 and fs/2, in dBFS.
    Observe the location and peak value of the principal frequency component, as well as the value
    of the noise floor. Do your observations agree (quantitatively) with what you would expect?
}