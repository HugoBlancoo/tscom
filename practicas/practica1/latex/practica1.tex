\documentclass[11pt,a4paper]{article}
\usepackage[utf8]{inputenc}
\usepackage[spanish]{babel}
\usepackage{graphicx}
\usepackage{subcaption}
\usepackage{amsmath,amssymb,amsfonts}
\usepackage{mathtools}
\usepackage{geometry}
\usepackage{xcolor}
\usepackage{hyperref}
\usepackage{booktabs}
\usepackage{enumitem}
\usepackage{fancyhdr}
\usepackage{titlesec}
\usepackage{microtype}
\usepackage{float}
\usepackage{listings}
\usepackage{color}

% Establecer márgenes
\geometry{a4paper, margin=1in, top=1.2in, headheight=15pt}

% Configurar hipervínculos
\hypersetup{
    colorlinks=true,
    linkcolor=blue,
    filecolor=blue,
    citecolor=blue,
    urlcolor=blue,
    pdftitle={Práctica 1},
    pdfauthor={Renato Bedriñana Cárdenas, Hugo Blanco Demelo},
    pdfsubject={Práctica 1 - Sampling and Quantization},
}

% Configuración de encabezado y pie de página
\pagestyle{fancy}
\fancyhf{}
\fancyhead[L]{\footnotesize Práctica 1}
\fancyhead[R]{\footnotesize Sampling and Quantization}
\fancyfoot[C]{\thepage}
\renewcommand{\headrulewidth}{0.4pt}
\renewcommand{\footrulewidth}{0.4pt}

% Configuración para código fuente
\definecolor{codegreen}{rgb}{0,0.6,0}
\definecolor{codegray}{rgb}{0.5,0.5,0.5}
\definecolor{codepurple}{rgb}{0.58,0,0.82}
\definecolor{backcolour}{rgb}{0.95,0.95,0.95}

\lstset{
    backgroundcolor=\color{backcolour},
    commentstyle=\color{codegreen},
    keywordstyle=\color{magenta},
    stringstyle=\color{codepurple},
    basicstyle=\ttfamily\small,
    breakatwhitespace=false,
    breaklines=true,
    captionpos=b,
    keepspaces=true,
    numbersep=5pt,
    showspaces=false,
    showstringspaces=false,
    showtabs=false,
    tabsize=2
}

% Formato de títulos
\titleformat{\section}
  {\normalfont\large\bfseries\color{blue!70!black}}
  {\thesection}{1em}{}
\titleformat{\subsection}
  {\normalfont\normalsize\bfseries\color{blue!60!black}}
  {\thesubsection}{1em}{}

% Espacio después de secciones
\titlespacing*{\section}{0pt}{3.5ex plus 1ex minus .2ex}{2.3ex plus .2ex}
\titlespacing*{\subsection}{0pt}{3.25ex plus 1ex minus .2ex}{1.5ex plus .2ex}

% Definición de comandos para matemáticas
\newcommand{\dB}{\text{dB}}
\newcommand{\dBFS}{\text{dBFS}}

% Información del documento
\title{\vspace{-1.5cm}\Large\textbf{Práctica 1: Sampling and Quantization}}
\author{\normalsize Renato Bedriñana Cárdenas \and \normalsize Hugo Blanco Demelo}
\date{\normalsize\today}
\usepackage[T1]{fontenc}
\usepackage{lmodern}

\begin{document}

\maketitle
\section{Task 1}
\textbf{Question: Give your interpretation of the resulting graphs. Do the quantization levels correspond with the values you had expected?}

\vspace{0.5cm}

In Figure~\ref{fig:task1_2}, we can observe the continuous signal x (blue) and the 2 quantized version using 2 (red) and 4 (yellow) bits.
As expected, the 2 bit quantization produces fewer discrete levels than the 4 bits quantization.
Increase the number of bits decreases $\Delta$, resulting in smaller steps and a quantized signal that follows the input more closely.

\begin{figure}[H]  % H is fot NOT letting latex insert the image in another place
    \centering
    \includegraphics[width=1\textwidth]{img/task1_2.png}
    \caption{Quantization error for N=2 (red) and N=4 (yellow)}
    \label{fig:task1_2}
\end{figure}

\vspace{1cm}
\textbf{Question: For both cases, represent the quantization error as a function of input amplitude in the range $[-7, +7]$ and comment on your results. Is this error always within the $[-\Delta /2, +\Delta /2]$ interval?}

\vspace{0.5cm}
The magnitude of the error decreases as the number of bits increases, since a smaller quantization step
$\Delta$ reduces the maximum deviation between the input and its quantized version.
The $[-\Delta /2, +\Delta /2]$ in each case is as follows:
\begin{itemize}
    \item For N = 2 the $\Delta$ value we get is $\Delta = 3.5$, so the interval should be $[-1.75, 1.75]$.
    \item For N = 4 the $\Delta$ value we get is $\Delta = 0.875$, so the interval should be $[-0.4375, 0.4375]$.
\end{itemize}
In both cases, the error remains bounded within the theoretical interval $[-\Delta /2, +\Delta /2]$.

\begin{figure}
    \centering
    \includegraphics[width=1\textwidth]{img/task1_2_2.png}
    % \caption{Description of your image}
    \label{fig:task1_2_2}
\end{figure}

\section{Task 2}
\textbf{Question: Assume a full-scale sinusoidal input and plot the histogram of the quantization error. Do you observe what you expected, or not?}

\vspace{0.5cm}
Due we have an amplitude equal to FS we can expect clipping. We have $\Delta= \frac{2*FS}{2^N} = 0.0098$, the $[-\frac{\Delta}{2}, +\frac{\Delta}{2}]$ interval should be uniformly distributed (while the input does not get clipped) between $[-0.0049, +0.0049]$.
In the histogram we can see that in that interval the error is uniformly distributed, but there is an error tail in the positive extreme.
It means that there is \textbf{clipping} in the positive.

\begin{figure}[H]
    \centering
    \includegraphics[width=1\textwidth]{img/task2_1.png}
    % \caption{Description of your image}
    \label{fig:task2_1}
\end{figure}

\vspace{1cm}
\textbf{Question: Explain the operation of the Matlab command var. Estimate the variance of the quantization error using var,
    and compare it to its theoretical value. Estimate the value (in dB) of the Signal to-Quantization Noise Ratio (SQNR)
    and compare it to its theoretical value.
}

\vspace{0.5cm}
The MATLAB command \texttt{var} computes the variance of a set of values. For a vector $x$, it calculates:
$\text{var}(x) = \frac{1}{n-1}\sum_{i=1}^n (x_i - \bar{x})^2$, where $\bar{x}$ is the mean of the values in $x$ and $n$ is the total number of samples.
We used \texttt{var(x,1)} to compute the population variance (divide by n).

The empiral value of the variance of the quantization error we got is $8.72123e-06$, and the theoretical value is $\frac{\Delta^2}{12} = 7.94729e-06$.

The estimated value of the SQNR in dB we got is $61.5633$ dB, and the theoretical value is $61.9597$ dB.

\vspace{1cm}
\textbf{Question: Repeat the previous steps for sinusoids with different amplitudes, and with decreasing resolutions
    of 12, 10, 8, 6 and 4 bits, in order to fill Table 1, rounding the SQNR values (in dB) to two
    decimal places. Comment on your results.
}

\vspace{0.5cm}
\begin{table}[H]
    \centering
    \begin{tabular}{|c||c|c|c|c|c|c|c|c|}
        \hline
                    & \multicolumn{2}{c|}{$A=0.5\cdot\text{\tt FS}$} & \multicolumn{2}{c|}{$A=0.75\cdot\text{\tt FS}$} & \multicolumn{2}{c|}{$A=\text{\tt FS}$} & \multicolumn{2}{c|}{$A=1.03\cdot\text{\tt FS}$}                                          \\
        \cline{2-9} & \multicolumn{2}{ |c| }{SQNR (dB)}              & \multicolumn{2}{ |c| }{SQNR (dB)}               & \multicolumn{2}{ |c| }{SQNR (dB)}      & \multicolumn{2}{ |c| }{SQNR (dB)}                                                        \\
        \hline
        $N$         & theory                                         & measured                                        & theory                                 & measured                                        & theory  & measured & theory & measured \\
        \hline\hline
        12          & 67.98                                          & 68.03                                           & 71.5                                   & 71.54                                           & 74.00   & 73.7     & 74.26  & 38.47    \\
        \hline
        10          & 55.94                                          & 56.01                                           & 59.46                                  & 59.53                                           & 61.96   & 61.56    & 62.22  & 38.19    \\
        \hline
        8           & 43.90                                          & 44.04                                           & 47.42                                  & 47.53                                           & 49.92   & 49.15    & 50.18  & 36.97    \\
        \hline
        6           & 31.86                                          & 32.13                                           & 35.38                                  & 35.6                                            & 37.88   & 36.52    & 38.14  & 32.27    \\
        \hline
        4           & 19.82                                          & 20.37                                           & 23.34                                  & 23.78                                           & 25.8397 & 23.63    & 26.1   & 22.53    \\
        \hline
    \end{tabular}
    \caption{Pertaining to Task 2.}
    \label{tab:task2}
\end{table}

For amplitudes below FS (0.5*FS and 0.75*FS) the empirical SQNR values closely match the theoretical predictions.
For an amplitude equal to FS, the empirical values still align well with theory, indicating minimal clipping effects.
However, as the amplitude exceeds FS (1.03*FS), discrepancies arise due to clipping effects, SQNR collapses and even adding more bits does not solve the problem.

As the number of bits decreases the variance of the error grows roughly as expected and SQNR drops approximately 6 dB/bit.
For moderate amplitudes the theory remains a good approximation down to mid-low N (but deviations increases as N gets smaller).

\section{Task 3}
\textbf{Question: Suppose that you have an N-bit A/D converter with tunable FS, and you know that your input samples follow
    a symmetric triangular pdf in some interval $[-x_0,x_0]$. Intuitively, how would you set the FS value of your converter?
    What would the resulting rms value $\sigma_x$ in dBFS be?
}

\vspace{0.5cm}
If you set $FS < x_0$ any imput $|x|$ greater than FS will be clipped. If $FS > x_0$,we would be wasting the converter's since the signal would never
reach the limits. Therefore, the value of FS should be $x_0$.

To reach the variance of a symmetric triangular distribution we need to make some calculations:
\[
    Var(x) = E[x^2] - (E[x])^2 = E[x^2] + 0 = \int_{-x_0}^{x_0} x^2 f(x) dx = x_0^2/6
\]

$\sigma_x = \sqrt{var(x)} = \frac{x_0}{\sqrt{6}}$ and in dBFS (with $x_0$ = FS) would be $20\log_{10}(1/\sqrt{6}) = -7,78$ dBFS.

\vspace{1cm}
\textbf{Question: Explain how to generate in Matlab samples of a random variable following a symmetric triangular pdf with
    zero mean and rms value $\sigma_x$. Check the histogram and use the commands mean and var to validate your approach
}

\vspace{0.5cm}
We have two options to do it:
\begin{itemize}
    % uso task3_2.m
    \item Option 1: The easiest way to generate a random variable with triangular pdf is using the function \textit{makedist} from Matlab.
          The function spects the parameters A, B and C that define the triangular distribution.
          So we set the function parameters to get a symmetric triangular distribution centered at 0: \texttt{makedist('Triangular','A',-x0,'B',x0,'C',0)}

          The result of the mean and var commands are as follows:
          \begin{itemize}
              \item Empirical mean: -4.58885e-05, wich is near to 0, very close to our target mean.
              \item Empirical rms value [dBFS]: -7.78072, wich is very close to our target variance.
          \end{itemize}

          \begin{figure}[H]
              \centering
              \includegraphics[width=1\textwidth]{img/task3_tri_mkdist.png}
              \label{fig:task3_tri_mkdist}
          \end{figure}

          To do it, we can use the following code:
          \begin{lstlisting}[language=Matlab]
            x0 = 2;
            A = -x0; B = 0; C = +x0; % simetria = media 0

            pd = makedist('Triangular','A',A,'B',B,'C',C);
            N = 100000;
            samples = random(pd, N, 1);

            % comprobaciones rapidas
            emp_mean = mean(samples);
            emp_var = var(samples);
            emp_desv_std = std(samples);

            % valores teoricos
            % theo_mean = 0; % simetria centrado en 0
            theo_var = (A^2 + B^2 + C^2 - A*B - A*C - B*C)/18;
            rms = 20*log10(sqrt(theo_var)/x0);

            fprintf('Theorical mean: 0; emp mean: %.2f\n',emp_mean);
            fprintf('Theorical var: %.2f; emp var: %.2f\n',theo_var,emp_var);
            fprintf('Sigma value: %.2f\n',sqrt(theo_var));
            fprintf('rms value in dBFS: %.2f\n',rms)

            % ver histograma y pdf teorica
            xgrid = linspace(A,C,500)';
            figure
            histogram(samples,100,'Normalization','pdf')
            hold on
            plot(xgrid, pdf(pd,xgrid), 'LineWidth',1.5)
            title('Triangular (media 0) -- muestras vs PDF')
            hold off
        \end{lstlisting}

          % uso task3_tri.m
    \item Option 2: Another way we can generate samples of a random variable following a symmetric triangular pdf as the sum of two independent random variables $X_1$ and $X_2$ from a uniform distribution.
          When two independent random variables with uniform distributions are added, the resulting probability density function (PDF) becomes triangular. This can be understood both intuitively and mathematically.

          Intuitively, if each variable is uniform on \([-a,a]\), there are many pairs that sum near zero but only a few that produce sums near the extremes \(\pm 2a\). Hence the PDF peaks at zero and decreases linearly towards the edges.

          Mathematically, let \(Z = X_1 + X_2\) with \(X_1,X_2\) independent and uniform on \([-a,a]\). The PDF of \(Z\) is the convolution of the two uniform PDFs:
          \[
              f_Z(z) = (f_{X_1} * f_{X_2})(z) = \int_{-\infty}^{\infty} f_{X_1}(t)\, f_{X_2}(-t + z)\, dt.
          \]
          Carrying out the convolution yields the triangular PDF supported on \([-2a,2a]\):
          \[
              f_Z(z) = \frac{2a - |z|}{4a^{2}}, \qquad |z|\le 2a.
          \]
          If we want the triangular distribution to have support \([-x_0,x_0]\), we must choose \(a = x_0/2\). In that case the PDF simplifies to
          \[
              f_Z(z)=\frac{x_0 - |z|}{x_0^2}, \qquad |z|\le x_0,
          \]
          Adding two uniform random variables with 0 mean, results in another random variable with 0 mean.

          \[
              E[Z] = E[X_1 + X_2] = E[X_1] + E[X_2] = 0 + 0 = 0.
          \]

          The variance of the sum of two independent random variables is the sum of their variances.
          So if we want a triangular distribution with variance $\sigma_x$ (in dBFS), we need to set the variance of each uniform variable to $\sigma_x/2$.

          \[
              Var(Z) = Var(X_1 + X_2) = Var(X_1) + Var(X_2) = \sigma_x/2 + \sigma_x/2 = \sigma_x.
          \]

          For a uniform on \([-a,a]\) we have \(\mathrm{Var}(X_i)=a^{2}/3\). Taking \(a=x_0/2\) gives \(\mathrm{Var}(Z)=2\cdot (x_0/2)^2/3 = x_0^2/6 = \sigma_x\), as required.

          The result of the mean and var commands are as follows:
          \begin{itemize}
              \item Empirical mean: 0.00003, wich is close to 0, very close to our target mean.
              \item Empirical variance [dB]: -7.80701, wich is very close to our target variance.
          \end{itemize}

          \begin{figure}[H]
              \centering
              \includegraphics[width=1\textwidth]{img/task3_tri_sum2uniform.png}
              \label{fig:task3_tri_sum2uniform}
          \end{figure}

          we can doit as follows: REVISAR!!
          \begin{lstlisting}[language=Matlab]
            x0=2;
            sigma0 = x0/sqrt(2);
            N = 100000;
            
            c = sigma0 * sqrt(3/2);

            x1 = (2 * rand(N, 1) - 1) * c;
            x2 = (2 * rand(N, 1) - 1) * c;

            y = x1 + x2;

            sample_mean = mean(y);
            sample_var = var(y);
            sample_rms = std(y);

            fprintf('--- Validation ---\n');
            fprintf('Target Mean: 0.0\n');
            fprintf('Sample Mean: %f\n\n', sample_mean);

            fprintf('Target Variance (sigma0^2): %f\n', sigma0^2);
            fprintf('Sample Variance: %f\n\n', sample_var);

            fprintf('Target RMS (sigma0): %f\n', sigma0);
            fprintf('Sample RMS: %f\n\n', sample_rms);

            figure;
            histogram(y, 100, 'Normalization', 'pdf', 'DisplayName', 'Generated Samples');
            grid on;
            hold on;

            a = 2*c;
            x_pdf = linspace(-a, a, 400);
            y_pdf = (1/a) * (1 - abs(x_pdf)/a);
            plot(x_pdf, y_pdf, 'r-', 'LineWidth', 2.5, 'DisplayName', 'Theoretical PDF');

            title('Symmetric Triangular Distribution');
            xlabel('Random Variable Value');
            ylabel('Probability Density Function (PDF)');
            legend;
            hold off;
        \end{lstlisting}
\end{itemize}

\vspace{1cm}
\textbf{Question: Take $10\cdot 2^{10}$ of these triangularly distributed samples, quantize them, and estimate the SQNR empirically
    for $N=$ 3, 4, 5 and 6 bits. Do this for $\sigma_x$ varying in the range $[-50, 0]$ dBFS and in steps of $0.1$ dBFS. Plot the
    resulting curves (SQNR in dB vs. $\sigma_x$ in dBFS) along with the theoretical expression
    \begin{equation}\label{eq:sqnr}
        {\rm SQNR} = 6.02 N + 4.77-20\log_{10}\frac{\rm FS}{\sigma_x} \qquad \mbox{(dB).}
    \end{equation}
    Are there any differences between the theoretical and empirical curves? If so, how do you explain them?}
\vspace{0.5cm}

\begin{figure}[H]
    \begin{subfigure}[t]{.4\textwidth}
        \centering
        \includegraphics[width=\linewidth]{img/task3_tri_n3.png}
        \caption{N=3 bits - $\sigma_{opt}$=-6.60 dBFS, SQNR=15.36 dB}
    \end{subfigure}
    \hfill
    \begin{subfigure}[t]{.4\textwidth}
        \centering
        \includegraphics[width=\linewidth]{img/task3_tri_n4.png}
        \caption{N=4 bits - $\sigma_{opt}$=-7.20 dBFS, SQNR=21.56 dB}
    \end{subfigure}

    \medskip

    \begin{subfigure}[t]{.4\textwidth}
        \centering
        \includegraphics[width=\linewidth]{img/task3_tri_n5.png}
        \caption{N=5 bits - $\sigma_{opt}$=-7.30 dBFS, SQNR=27.52 dB}
    \end{subfigure}
    \hfill
    \begin{subfigure}[t]{.4\textwidth}
        \centering
        \includegraphics[width=\linewidth]{img/task3_tri_n6.png}
        \caption{N=6 bits - $\sigma_{opt}$=-7.30 dBFS, SQNR=33.47 dB}
    \end{subfigure}

    \caption{SQNR vs $\sigma_x$ (dBFS) for triangularly distributed input at different quantization resolutions.}
    \label{fig:task3_tri_sqnr_vs_sigma}
\end{figure}

The comparison between theoretical and empirical SQNR curves for triangularly distributed inputs is shown in Figure~\ref{fig:task3_tri_sqnr_vs_sigma}.
The red line represents the theoretical SQNR curve, while the blue line represents the empirical SQNR values obtained from quantizing the triangularly distributed samples.

Empirical curves deviate from the straight theoretical line for two practical reasons. At very small $\sigma_x$ values, quantization noise is no longer uniformly distributed and we have no enough bits for quantization.
At large $\sigma_x$ values, clipping occurs, distorting the signal and reducing SQNR below theoretical predictions.
Otherweise, in the mid-range of $\sigma_x$ values, empirical results closely follow theoretical expectations and reaches the maximum SQNR (marked in the description of each image).

When we increase the number of bits N, the curve starts to follow the theorical curve with smaller values of $\sigma_x$.
We reach a point where optimum $\sigma_x$ value (where SQNR is maximized) approaches the theoretical value of -7.78 dBFS calculated earlier.

\vspace{1cm}
\textbf{Question: In view of your results, what are the optimum values (regarding SQNR) of $\sigma_x$ (in dBFS), and for the different resolutions analyzed (3 to 6 bits)?
    Does this agree with your intuition (see first point above)?
}
\vspace{0.5cm}

We can see in the previous plots~\ref{fig:task3_tri_sqnr_vs_sigma} (see the green lines) that the optimum values of $\sigma_x$ (where SQNR is maximized) for different resolutions are:
\begin{itemize}
    \item N=3 bits: $\sigma_{opt}$ = -6.60 dBFS
    \item N=4 bits: $\sigma_{opt}$ = -7.20 dBFS
    \item N=5 bits: $\sigma_{opt}$ = -7.30 dBFS
    \item N=6 bits: $\sigma_{opt}$ = -7.30 dBFS
\end{itemize}
Those are the points where the empirical SQNR reaches its maximum value and then starts to decrease.

The value we calculated in the first point was -7.78 dBFS, which is close to the optimum values we obtained empirically.
If we increase the number of bits, the optimum value gets closer.

\vspace{1cm}
\textbf{Question: Repeat the previous points, but now using normally distributed input samples with zero mean and standard deviation $\sigma_x$.
}
\vspace{0.5cm}

In a Gaussian distribution, we can not set the $[-x_0,x_0]$ limit as in the triangular distribution. So it is a parameter that we can not control and clipping will always occur for some samples no matter how we set FS.
But, we can still set FS to optimize SQNR. Following the definition of dBFS, we can set FS to be some multiple of $\sigma_x$.

\[
    \sigma_x = 20\log_{10}\frac{\sigma_x}{\rm FS} = -20\log_{10}{k} \implies FS = k \cdot \sigma_x
\]

To decide which k value is the best, we have to see the clip probability for each k.
To express the clipping probability in terms of the standard normal CDF, we normalize the Gaussian variable as $Z = X / \sigma_x$, so that $Z \sim \mathcal{N}(0,1)$. Then:
\[
    P(X > FS) = P\!\left(Z > \frac{FS}{\sigma_x}\right) = 1 - \Phi\!\left(k\right),
\]
where $\Phi(k)$ is the cumulative distribution function of the standard normal distribution. Therefore, considering both tails, the total clipping probability is:
\[
    p_{\text{clip}} = 2(1 - \Phi(k)), \quad \text{with } k = \frac{FS}{\sigma_x}.
\]

So we have the next posible values for k:

\begin{itemize}
    \item k = 1: $\sigma_x$ = 0 dBFS (clipping prob = 31.73\%)
    \item k = 2: $\sigma_x$ = -6.02 dBFS (clipping prob = 4.55\%)
    \item k = 3: $\sigma_x$ = -9.54 dBFS (clipping prob = 0.27\%)
    \item k = 4: $\sigma_x$ = -12.04 dBFS (clipping prob = 0.000063\%)
\end{itemize}

A good trade-off between clipping probability and SQNR can be achieved with k = 3, which gives a good compromise between dynamic range usage and distortion.

To generate normally distributed samples we can use the Matlab command \texttt{randn}, which generates samples from a standard normal distribution (mean 0, variance 1).
And then we scale the samples to get the desired standard deviation $\sigma_x$. The output plot is visible in the Figure~\ref{fig:task3_normal_dist}.

% use el código task3_normal_dist_set.m
The result of the mean and var commands are as follows:
\begin{itemize}
    \item Empirical mean: -0.000265784, close to our target mean.
    \item Empirical variance [dB]: -9.56729, again, close to the value we expected.
\end{itemize}

Analyzing the SQNR vs $\sigma_x$ curves in different N levels, we obtain similar outputs as the triangular distribution.
We can check them in the Figure~\ref{fig:task3_normal_sqnr_vs_sigma}, we observe that the empirical SQNR values deviate from the theoretical predictions, especially at higher $\sigma_x$ values.

But we need more level of $\sigma_x$ to reach the optimum point, because of the clipping that occurs in the Gaussian distribution, and we get a fewer maximum SQNR value than in the triangular distribution.
This is produced because the Gaussian distribution has heavier tails, leading to more frequent clipping events at higher $\sigma_x$ levels.

\begin{figure}[H]
    \centering
    \includegraphics[width=1\textwidth]{img/task3_normal_dist.png}
    \caption{Gaussian distribution (normalized)}
    \label{fig:task3_normal_dist}
\end{figure}

\begin{figure}[H]
    \begin{subfigure}[t]{.4\textwidth}
        \centering
        \includegraphics[width=\linewidth]{img/task3_normal_n3.png}
        \caption{N=3 bits - $\sigma_{opt}$=-7.90 dBFS, SQNR=14.03 dB}
    \end{subfigure}
    \hfill
    \begin{subfigure}[t]{.4\textwidth}
        \centering
        \includegraphics[width=\linewidth]{img/task3_normal_n4.png}
        \caption{N=4 bits - $\sigma_{opt}$=-8.50 dBFS, SQNR=19.44 dB}
    \end{subfigure}

    \medskip

    \begin{subfigure}[t]{.4\textwidth}
        \centering
        \includegraphics[width=\linewidth]{img/task3_normal_n5.png}
        \caption{N=5 bits - $\sigma_{opt}$=-9.20 dBFS, SQNR=24.79 dB}
    \end{subfigure}
    \hfill
    \begin{subfigure}[t]{.4\textwidth}
        \centering
        \includegraphics[width=\linewidth]{img/task3_normal_n6.png}
        \caption{N=6 bits - $\sigma_{opt}$=-10.50 dBFS, SQNR=30.16 dB}
    \end{subfigure}

    \caption{SQNR vs $\sigma_x$ (dBFS) for normal distribution input at different quantization resolutions.}
    \label{fig:task3_normal_sqnr_vs_sigma}
\end{figure}

\vspace{0.5cm}
\section{Task 4}
\textbf{Question: Assume a full-scale sinusoidal input with $f_0 = 37.1094 MHz$, and let the FFT size be M = 1024.
    Generate $15 \cdot M$ samples of $x(t)$ (at fs = 100 MHz) and quantize them to N = 12 bits. Break
    the vector xq of quantized samples into 15 size-M blocks using, e.g., the command reshape:}

\vspace{0.5cm}

\begin{lstlisting}[language=Matlab]
    xqblocks = reshape(xq, M, 15);
\end{lstlisting}

\textbf{so that each column of the $M \times 15$ matrix xqblocks will contain the corresponding block of size
    M . Now, since the fft command computes the FFT columnwise, in order to apply an M -point
    FFT to each block, we simply make
}
\begin{lstlisting}[language=Matlab]
    X = fft(xqblocks, M);
\end{lstlisting}

\textbf{Average the squared magnitude of the DFT coefficients over the 15 blocks and plot the results
    between 0 and fs/2, in dBFS.
    Observe the location and peak value of the principal frequency component, as well as the value
    of the noise floor. Do your observations agree (quantitatively) with what you would expect?
}
\vspace{0.5cm}

\subsection{Theoretical values}
First, we need to calculate the expected theoretical values for the signal peak and for the noise floor value.
\subsubsection{Signal Peak}
We have $f_0 = 37.1094 MHz$ and $f_s = 100 MHz$. As $f_0 < f_s/2$ we don't have aliasing.
Terefore, we expect a signal peak at $f_0$, with a value of 0 DBFS, as it is a full-scale signal.

\subsubsection{Noise Floor}

To calculate the theoretical SQNR we have the formula
$SQNR = 6.02N +4.77 -20log_{10}(FS/\sigma_x)$
As we have a full scale sinusoid we have $\sigma_x = A/\sqrt{2} = FS/\sqrt{2}$
So, for $N=12$ and $\sigma_x = FS/\sqrt{2}$ we have $SQNR = 73.99 DBFS$

We have to calculate the processing gain, with the formula $10log_{10}(M/2)$.
For $M=1024$, we have a gain of 27.09 DBFS.

The noise floor will be $-(73.99+27.09)=101.08 DBFS$

\subsection{Matlab execution}

Executing the task\_4\_1.m matlab script we can see the next figure.

\begin{figure}[H]
    \centering
    \includegraphics[width=1\textwidth]{img/task4_1.png}
    \label{fig:task4_1}
\end{figure}

On the figure we can see a signal peak at 37.1094 MHz, whith a value of 0 DBFS.

This agrees quantitatively with the theory, which predicts a peak at the input frequency.

$f_0 = 37.1094 MHz$ and a level of 0 dBFS due to the normalization used for a full-scale signal.

We can also see that the noise floor is around the 100 DBFS, which agrees with the theoretical value.
\vspace{1cm}

\textbf{Question: Repeat the previous steps for an FFT size M = 256.
}
\subsection{Theoretical values}
As the frequency $f_0$ is the same, we would also have a signal peak on that point.
Since is full-scale too, thew value of the peak would also be 0 DBFS.

The SQNR will be the same, because we have the same number of bits and the same $\sigma_x$.

However, the gain will change, as we have a different value for M.
$Gain = 10log_{10}(M/2) = 10log_{10}(256/2) = 21.07$

For M = 256 we will have a nois floor of 95.06 DBFS

\subsection{Matlab execution}
Executing the task\_4\_2.m script, we can see a noise peak of 0 DBFS at $f_0$ and a level of noise floor
of approximately 95 DBFS
\begin{figure}[H]
    \centering
    \includegraphics[width=1\textwidth]{img/task4_2.png}
    \label{fig:task4_2}
\end{figure}
\vspace{1cm}

\textbf{Question: Set again M = 1024, and repeat the analysis for decreasing resolutions of 10, 8 and 6 bits.}
\subsection{Theoretical values}

As the frequency $f_0$ is the same, we would also have a signal peak on that point.
Since is full-scale too, thew value of the peak would also be 0 DBFS.

The Gain will be the same as on task4\_1 because we have thje same M.

The SQNR will change, because we have different values for N:
\begin{itemize}
    \item N = 10: $SQNR = 6.02N +4.77 -20log_{10}(FS/\sigma_x) = 6.02 * 10 +4.77 -20log_{10}(\sqrt{2}) = 61.95 DBFS$
    \item N = 8: $SQNR = 6.02N +4.77 -20log_{10}(FS/\sigma_x) = 6.02 * 8 +4.77 -20log_{10}(\sqrt{2}) = 49.91 DBFS$
    \item N = 6: $SQNR = 6.02N +4.77 -20log_{10}(FS/\sigma_x) = 6.02 * 6 +4.77 -20log_{10}(\sqrt{2}) = 37.87 DBFS$
\end{itemize}

So the noise floor for each value of N will be:
\begin{itemize}
    \item N = 10: $-(61.95 + 27.09) = -89.04 DBFS $
    \item N = 8: $-(49.91 + 27.09) = -77 DBFS$
    \item N = 6: $-(37.87 + 27.09) = -64.96 DBFS$
\end{itemize}

\subsection{Matlab execution}

Executing the the task\_4\_3.m script, we can see three figures with a peak of 0 DBFS on $f_0$.
We also see a noise floor value of :
\begin{itemize}
    \item N = 10: $-89.05 DBFS $
    \item N = 8: $-77.01 DBFS$
    \item N = 6: $-64.97 DBFS$
\end{itemize}

\begin{figure}[H]
    \begin{subfigure}[t]{.5\textwidth}
        \centering
        \includegraphics[width=\linewidth]{img/task4_3_n10.png}
        \caption{N=10 bits}
    \end{subfigure}
    \begin{subfigure}[t]{.5\textwidth}
        \centering
        \includegraphics[width=\linewidth]{img/task4_3_n8.png}
        \caption{N=8 bits}
    \end{subfigure}
    \begin{subfigure}[t]{.5\textwidth}
        \centering
        \includegraphics[width=\linewidth]{img/task4_3_n6.png}
        \caption{N=6 bits}
    \end{subfigure}
\end{figure}
\vspace{1cm}

\textbf{Question: Consider again M = 1024 and N = 12 bits. Repeat the analysis reducing the amplitude of
    the sinusoid to 1/3 of the full scale value, and compare your observations with the theoretical
    prediction.
}
\subsection{Theoretical values}
For a siusoid of amplitude $A =\alpha * FS , \alpha < 1$, we have $SQNR = 6.02N + 1,76 + 20log{10}(A)$.

For $A=1/3$ we have a SQNR of 64.81 DBFS. The gain will be the same, so we will have a noise floor of 91.9 DBFS


\vspace{1cm}

\textbf{Question: Let M = 1024, N = 12 bits and a full-scale sinusoid. Slightly change the frequency of the
    sinusoid to 37.12 MHz and repeat the analysis. How do your observations change? Does it make
    any difference if you use a larger number of samples, say 100 * M ? What happens if you increase
    the resolution to 16 bits?
    How do you explain all these?
}

The $f_0$ we had, was a $k$ of the FFT, so all the energy was on a single $k$: $ k =\frac{f_0*M}{fs} = \frac{37.1094*10^6*1024}{100*10^6}= 380$
When using the new frequency, the energy is between $k = 380$ and $k = 381$, so we see this power leak.

\begin{figure}[H]
    \centering
    \includegraphics[width=1\textwidth]{img/task4_5.png}
    \label{fig:task4_5}
\end{figure}

Increasing the number of samples or the resolution will not have effect, because we need to have the signal on a $k$.
To do this, we can change either the $f_0$, the $fs$ or M.

\vspace{0.5cm}
\section{Task 5}
\textbf{Question: Plot $g_\gamma(x)$ vs. $x$ in the range $x\in [-{\rm FS},{\rm FS}]$ for $\gamma = 0$, $1$ and $2$. For input signals whose values are always much smaller than $\rm FS$ (in absolute value), what will be the effect of the nonlinearity?
}
\vspace{0.5cm}

xd

\vspace{1cm}
\textbf{Question: Modify the code in {\tt quanti.m} and write a Matlab function {\tt dquanti.m} implementing this nonuniform quantizer. The format should be similar to that of {\tt quanti.m}, but including an additional input parameter {\tt gama}:}
\begin{center}
{\tt xq = dquanti( x, FS, Nbits, gama ); }
\end{center} 
\vspace{0.5cm}

xd

\vspace{1cm}
\textbf{Question: Generate  samples (at 100 MHz) of a full-scale sinusoid with $f_0 = 6.8359$ MHz.
Quantize them to $N=11$ bits using $\gamma = 0.003$ in {\tt dquanti}. 
Determine the SFDR in dBFS using an FFT size $M=2048$, and then with $M=512$. 
Does the SFDR depend on the FFT size? Does the noise floor depend on the FFT size? How do you explain this?
}
\vspace{0.5cm}

xd

\vspace{1cm}
\textbf{Question: Using $M=2048$, repeat the previous step for $\gamma = 0.01$ and $0.1$. Are the spectral spurs located where you would expect?
}
\vspace{0.5cm}

xd

\vspace{1cm}
\textbf{Question: Set now the amplitude to $\frac{\rm FS}{3}$. Using $M=2048$, measure the SFDR and express it in both dBFS and dBc for $\gamma=0.005$, $0.05$ and $0.1$. Will these values change if you repeat the analysis with $M=512$?
}
\vspace{0.5cm}

xd

\vspace{1cm}
\textbf{Question: Consider now samples (at 100 MHz and with 11-bit resolution) of a sinusoid with frequency $3.3202$ MHz and amplitude $\frac{\rm FS}{2}$. Obtain the THD for this nonuniform ADC with $\gamma = 0.3$ under the IEEE 1241-2000 specification, expressed in both dB and percentage.
}
\vspace{0.5cm}

xd



\vspace{0.5cm}
\section{Task 6}
We decide to invest in an additional pressure sensor, which we install it at the opposite side of the bottom of the tank. In that way, the measurement errors at the two sensors can be assumed independent of each other. However, the original sensor is out of stock, so we are forced to purchase a different model whose errors need not have the same variance as the first sensor.

%%%%%%%%%%%%%%%%%%%%%%%%%%%%%%%%%%%%%%%%%%%%%%%%%%%%%%%%%%%%%%%%%%%%%%%%%%%%%%%%%%%%%%%%%%%%%%%%%%%%%
%%%%%%%%%%%%%%%%%%%%%%%%%%%%%%%%%%%%%%%%%%%%%%%%%%%%%%%%%%%%%%%%%%%%%%%%%%%%%%%%%%%%%%%%%%%%%%%%%%%%%
\question{Question: Note that the addition of a new sensor does not alter the ''state evolution'' description of the system, but it obviously changes its ``measurement'' description. Give the corresponding details.}
\vspace{0.5cm}

\begin{itemize}
    \item \textbf{State vector (unchanged):}
    
    The state remains two-dimensional, containing both level and flow:
    $$\mathbf{s}_n = \begin{bmatrix} l_n \\ q_n \end{bmatrix}$$
    
    \item \textbf{State transition matrix (unchanged):}
    
    The dynamics remain identical to Task 5:
    $$\mathbf{A} = \begin{bmatrix} 1 & \frac{T}{A_{\text{tank}}} \\ 0 & 1 \end{bmatrix}$$
    
    where $T = 5$ s and $A_{\text{tank}} = \pi (11)^2 \approx 380.13$ cm$^2$.
    
    \item \textbf{Measurement matrix (modified):}
    
    In Task 5 we had one sensor measuring pressure:
    $$x_n = c_h \cdot l_n + w_n \quad \Rightarrow \quad \mathbf{H} = \begin{bmatrix} c_h & 0 \end{bmatrix}$$
    
    Now with \textbf{two sensors}, both measuring pressure (proportional to level), the measurement becomes a vector:
    $$\mathbf{x}_n = \begin{bmatrix} x_n^{(1)} \\ x_n^{(2)} \end{bmatrix} = \begin{bmatrix} c_h & 0 \\ c_h & 0 \end{bmatrix} \begin{bmatrix} l_n \\ q_n \end{bmatrix} + \begin{bmatrix} w_n^{(1)} \\ w_n^{(2)} \end{bmatrix}$$
    
    Therefore:
    $$\boxed{\mathbf{H} = \begin{bmatrix} c_h & 0 \\ c_h & 0 \end{bmatrix}}$$
    
    Both sensors measure the same physical quantity (level via pressure), but with different noise levels.
    
    \item \textbf{Measurement noise covariance (modified):}
    
    The two sensors have independent errors with different variances:
    $$\mathbf{R} = \begin{bmatrix} \sigma_{v,1}^2 & 0 \\ 0 & \sigma_{v,2}^2 \end{bmatrix} = \begin{bmatrix} 20^2 & 0 \\ 0 & 80^2 \end{bmatrix} \text{ mbar}^2$$
    
    The off-diagonal terms are zero because the sensor errors are independent.
\end{itemize}

\textbf{Summary:} The state evolution remains unchanged. The only modification is expanding the measurement equation from a scalar to a 2D vector, with each row of $\mathbf{H}$ corresponding to one sensor.

%%%%%%%%%%%%%%%%%%%%%%%%%%%%%%%%%%%%%%%%%%%%%%%%%%%%%%%%%%%%%%%%%%%%%%%%%%%%%%%%%%%%%%%%%%%%%%%%%%%%%
%%%%%%%%%%%%%%%%%%%%%%%%%%%%%%%%%%%%%%%%%%%%%%%%%%%%%%%%%%%%%%%%%%%%%%%%%%%%%%%%%%%%%%%%%%%%%%%%%%%%%
\question{Question: Write a Matlab script {\tt kalman\_flow2.m} to simulate this dynamical system and the corresponding Kalman filter to estimate the fluid (benzene) level and the fluid flow. Assume that the two sensors work correctly (i.e., $p=0$ in \eqref{eq:xno}), measuring pressure in mbar, taking one measurement every 5 seconds, and that the cross-section of the tank is circular with diameter 22 cm. Fluid level is to be expressed in cm, as before, whereas fluid flow is to be expressed in cm$^3/$s.}
\question{
          The initial true fluid level is 340 cm; our guess is 250 cm, with a standard deviation of 11 cm. The true fluid flow is constant with time and equal to 33 cm$^3$/s; our guess is 0 cm$^3/$s, with standard deviation of 10 cm$^3/$s. Measurement errors are zero-mean uncorrelated Gaussian, with standard deviation of 20 mbar at the first sensor and of 80 mbar at the second.}

The script \texttt{kalman\_flow.m~\ref{app:kalman_flow2}} implements the Kalman filter for the described two-sensor dynamic system.

%%%%%%%%%%%%%%%%%%%%%%%%%%%%%%%%%%%%%%%%%%%%%%%%%%%%%%%%%%%%%%%%%%%%%%%%%%%%%%%%%%%%%%%%%%%%%%%%%%%%%
%%%%%%%%%%%%%%%%%%%%%%%%%%%%%%%%%%%%%%%%%%%%%%%%%%%%%%%%%%%%%%%%%%%%%%%%%%%%%%%%%%%%%%%%%%%%%%%%%%%%%
\question{Question: Plot the time evolution of the fluid level and fluid flow, the measurements, and the estimates, for up to $60$ minutes, and for two different executions. Comment on your results.}
\vspace{0.5cm}

We executed the simulation for 60 minutes with two independent executions. The results are shown below.

\subsection*{System Parameters}
\begin{itemize}
    \item \textbf{Tank:} Diameter 22 cm, sampling period $T=5$ s.
    \item \textbf{True initial state:} Level 340 cm, flow 33 cm$^3$/s (constant).
    \item \textbf{Initial guesses:} Level 250 cm ($\sigma_l=11$ cm), flow 0 cm$^3$/s ($\sigma_q=10$ cm$^3$/s).
    \item \textbf{Sensor noise:} Sensor 1: $\sigma_{v,1}=20$ mbar, Sensor 2: $\sigma_{v,2}=80$ mbar.
\end{itemize}

\subsection*{Simulation Results}

\begin{figure}[H]
    \centering
    \includegraphics[width=0.85\textwidth]{img/task6_level.png}
    \caption{Time evolution of fluid level with two sensors. Gray asterisks: Sensor 1 measurements (low noise). Green asterisks: Sensor 2 measurements (high noise). Blue/red lines: Kalman filter estimates. Black dashed: true level.}
    \label{fig:task6_level}
\end{figure}

\begin{figure}[H]
    \centering
    \includegraphics[width=0.85\textwidth]{img/task6_flow.png}
    \caption{Time evolution of fluid flow with two sensors. The filter successfully estimates the hidden flow variable, converging from 0 cm$^3$/s to the true value of 33 cm$^3$/s.}
    \label{fig:task6_flow}
\end{figure}

\subsection*{Discussion}

\textbf{Fluid Level (Figure~\ref{fig:task6_level}):}
\begin{enumerate}
    \item \textbf{Sensor Quality Difference:} Sensor 1 (gray points) produces measurements with low scatter, clustering near the true trajectory. Sensor 2 (green points) has much higher noise, with measurements spread widely (up to $\pm 90$ cm equivalent deviation).
    
    \item \textbf{Filter Robustness:} Despite the noisy second sensor, the Kalman filter estimates (blue/red lines) track the true level accurately. The filter automatically gives more weight to Sensor 1 (which has higher precision) while still extracting useful information from Sensor 2.
    
    \item \textbf{Level Evolution:} The true level increases linearly at approximately $\frac{33}{380.13} \approx 0.087$ cm/s (5.2 cm/min) due to the constant positive flow.
\end{enumerate}

\textbf{Fluid Flow (Figure~\ref{fig:task6_flow}):}
\begin{enumerate}
    \item \textbf{Convergence:} Starting from an initial guess of 0 cm$^3$/s, the filter estimates converge to the true value of 33 cm$^3$/s within approximately 10-15 minutes.
    
    \item \textbf{Indirect Observation:} The flow is not directly measured. The filter infers it by observing how fast the level is changing over time, combining information from both sensors.
    
    \item \textbf{Stability:} After convergence, the estimates remain stable near 33 cm$^3$/s, demonstrating successful tracking of the hidden state.
\end{enumerate}

%%%%%%%%%%%%%%%%%%%%%%%%%%%%%%%%%%%%%%%%%%%%%%%%%%%%%%%%%%%%%%%%%%%%%%%%%%%%%%%%%%%%%%%%%%%%%%%%%%%%%
%%%%%%%%%%%%%%%%%%%%%%%%%%%%%%%%%%%%%%%%%%%%%%%%%%%%%%%%%%%%%%%%%%%%%%%%%%%%%%%%%%%%%%%%%%%%%%%%%%%%%
\question{Question: Plot the time evolution of the standard deviation of the estimation error for both the fluid level (in cm) and the fluid flow (in cm$^3/$s), also for two different realizations. Superimpose the corresponding curves when only the first sensor is available, and explain what you see.}
\vspace{0.5cm}

Figure~\ref{fig:task6_comparison_std} compares the evolution of estimation uncertainty between the two-sensor configuration (Task 6) and the single-sensor case (Task 5).

\begin{figure}[!htbp]
    \centering
    \includegraphics[width=0.9\textwidth]{img/task6_comparison_std.png}
    \caption{Estimation uncertainty: level (left) and flow (right). 
    Blue (Task 6, 2 sensors) vs red (Task 5, 1 sensor). Log scale.}
    \label{fig:task6_comparison_std}
\end{figure}

\subsection*{Comments}

The two-sensor configuration reduces both level and flow uncertainty compared to the single-sensor case. Level uncertainty decreases faster with two sensors. Flow uncertainty shows a smaller but consistent improvement, 
since flow estimation depends indirectly on observing level changes. The Kalman filter automatically weights each sensor by its reliability, so the poor-quality sensor still helps without degrading performance.


\appendix
\section{Appendix: MATLAB scripts and data}
\label{app:matlab}

\subsection{Task 1}
\subsubsection{task1.m}

\begin{lstlisting}[language=Matlab]
x = linspace(-7,7,1000);
xq = quanti(x,7,2);
xq_4 = quanti(x,7,4);

figure;
plot(x,x,'b',x,xq,'r',x,xq_4,'y');
grid on

quantized_error = x - xq;
quantized_error_4b = x - xq_4;

figure;
plot(x,quantized_error, 'r', x, quantized_error_4b, 'y');
title('Quantization Error as a Function of Input Amplitude');
xlabel('Input Amplitude');
ylabel('Quantization Error');
grid on;
\end{lstlisting}

\subsection{Task 2}
\subsubsection{task2.m}

\begin{lstlisting}[language=Matlab]
FS = 5;
fs = 100e6;
f0 = 18.17e6;
M = 15 * 2^10;
t = (0:M-1)/fs;

A_list = [0.5, 0.75, 1.0, 1.03] * FS;
Nbits_list = [12, 10, 8, 6, 4];

results = [];

for A = A_list
    x = A * cos(2*pi*f0*t);
    sigma_x = A / sqrt(2);
    var_x_emp = var(x,1);
    for Nbits = Nbits_list
        LSB = FS / 2^(Nbits-1);
        xq = quanti(x, FS, Nbits);

        e = x - xq;
        var_e_emp = var(e,1);
        var_e_theor = LSB^2 / 12;

        SQNR_emp = 10*log10(var_x_emp / var_e_emp);
        SQNR_theor_formula = 6.02*Nbits + 4.77 - 20 * log10(FS / sigma_x);

        % Save
        results = [results; A, Nbits, var_e_emp, var_e_theor, SQNR_emp, SQNR_theor_formula];
    end
end
\end{lstlisting}


\subsection{Task 3}
\subsubsection{task3\_2.m}

\begin{lstlisting}[language=Matlab]

rng(0)

x0 = 1;
A = -x0; B = 0; C = +x0;

pd = makedist('Triangular','A',A,'B',B,'C',C);
N = 100000;
samples = random(pd, N, 1);

emp_mean = mean(samples);
emp_var = var(samples);
emp_desv_std = std(samples);

theo_var = (A^2 + B^2 + C^2 - A*B - A*C - B*C)/18;

rms = 20*log10(sqrt(emp_var)/x0);

fprintf('Theorical mean: 0; emp mean: %g\n',emp_mean);
fprintf('Theorical var: %.2f; emp var: %.2f\n',theo_var,emp_var);
fprintf('Sigma value: %.2f\n',sqrt(theo_var));
fprintf('rms value in dBFS: %g\n',rms)

xgrid = linspace(A,C,400)';
figure
histogram(samples,100,'Normalization','pdf', 'DisplayName', 'Generated samples')
hold on; grid on;

plot(xgrid, pdf(pd,xgrid), 'r-', 'LineWidth', 2, 'DisplayName', 'Theoretical PDF');

title('Symmetric Triangular Distribution - makedist');
xlabel('Value');
ylabel('PDF');
legend('Location','best');
hold off
\end{lstlisting}

\subsubsection{task3\_tri.m}
\begin{lstlisting}[language=Matlab]
rng(0);

x0 = 1;
sigma_x = x0 / sqrt(6);
N = 100000;

a = x0 / 2;

x1 = (2*rand(N,1) - 1) * a;
x2 = (2*rand(N,1) - 1) * a;
y = x1 + x2;

mean_y = mean(y);
var_y = var(y,1);
rms_dBFS = 20*log10(sqrt(var_y)/x0);

fprintf('--- Validation ---\n');
fprintf('Expected mean = 0\n');
fprintf('Sample mean   = %.5f\n\n', mean_y);

fprintf('Expected RMS [dBFS] = -7.78\n');
fprintf('Sample RMS   [dBFS] = %g\n', rms_dBFS);

figure;
histogram(y, 100, 'Normalization', 'pdf', 'DisplayName', 'Generated samples');
hold on; grid on;

x_pdf = linspace(-x0, x0, 400);
f_pdf = (x0 - abs(x_pdf)) / (x0^2);
plot(x_pdf, f_pdf, 'r-', 'LineWidth', 2, 'DisplayName', 'Theoretical PDF');

title('Symmetric Triangular Distribution - Sum 2 dist');
xlabel('Value');
ylabel('PDF');
legend('Location','best');
hold off;

\end{lstlisting}

\subsubsection{task3\_normal\_dist\_set.m}
\begin{lstlisting}[language=Matlab]
rng(0);

x0 = 1;
sigma_x = 10^(-9.54/20);
N = 100000;

x = sigma_x * randn(1, N);

emp_mean = mean(x);
emp_var  = var(x,1);
rms_dBFS = 20*log10(sqrt(emp_var)/x0);

fprintf('--- Validation ---\n');
fprintf('Expected mean = 0\n');
fprintf('Sample mean   = %g\n\n', emp_mean);

fprintf('Expected RMS [dBFS] = -9.54\n');
fprintf('Sample RMS   [dBFS] = %g\n', rms_dBFS);

figure;
histogram(x, 60, 'Normalization', 'pdf');
hold on; grid on;

xx = linspace(-4*sigma_x, 4*sigma_x, 400);
plot(xx, (1/(sigma_x*sqrt(2*pi))) * exp(-0.5*(xx/sigma_x).^2), 'r-', 'LineWidth',1.5);

title('Gaussian Distribution');
xlabel('Value');
ylabel('PDF');
legend('Empirical PDF','Theoretical');
grid on; hold off;
\end{lstlisting}

\subsection{Task 4}
\subsubsection{task4\_1.m}

\begin{lstlisting}[language=Matlab]
f0 = 37.1094e6;
M = 1024;
fs = 100e6;
Nbits = 12;
blocks = 15;
FS = 1;
                                                                                                                                                                                                                                                                                                                                                                                                                                                                                                                                                                                                                                                    
Nsamples = blocks * M;

%% Generar FS sinusoidal
n = 0:Nsamples-1;
xt = FS * cos(2*n*pi*f0/fs);
xq = quanti(xt, FS, Nbits);

xqblocks = reshape(xq, M, 15);

X = fft(xqblocks, M);

P_avg = mean(abs(X).^2, 2);

norm_const = (M / 2)^2;
P_dbfs = 10 * log10(P_avg / norm_const);

f_axis = (0:M-1) * fs / M; 
plot(f_axis(1:M/2 + 1) / 1e6, P_dbfs(1:M/2 + 1));
grid on;
xlabel('Frequency (MHz)');
ylabel('Power (dBFS)');
title('Espectro de Potencia Promedio (N=12 bits, M=1024)');
ylim([-140, 10]);
\end{lstlisting}


\subsubsection{task4\_2.m}
\begin{lstlisting}[language=Matlab]
f0 = 37.1094e6;
M = 256;
fs = 100e6;
Nbits = 12;
blocks = 15;
FS = 1;

Nsamples = blocks * M;

%% Generar FS sinusoidal
n = 0:Nsamples-1;
xt = FS * cos(2*n*pi*f0/fs);
xq = quanti(xt, FS, Nbits);

xqblocks = reshape(xq, M, 15);

X = fft(xqblocks, M);

P_avg = mean(abs(X).^2, 2);

norm_const = (M / 2)^2;
P_dbfs = 10 * log10(P_avg / norm_const);

f_axis = (0:M-1) * fs / M;
plot(f_axis(1:M/2 + 1) / 1e6, P_dbfs(1:M/2 + 1));
grid on;
xlabel('Frequency (MHz)');
ylabel('Power (dBFS)');
title('Espectro de Potencia Promedio (N=12 bits, M=256)');
ylim([-140, 10]);
\end{lstlisting}


\subsubsection{task4\_3.m}
\begin{lstlisting}[language=Matlab]
f0 = 37.1094e6;
M = 1024;
fs = 100e6;
blocks = 15;
FS = 1;
Nsamples = blocks * M;
f_axis = (0:M-1) * fs / M;
norm_const = (M / 2)^2; 

n = 0:Nsamples-1;
xt = FS * cos(2*n*pi*f0/fs);

Nbits_list = [10, 8, 6];

for i = 1:length(Nbits_list)
    Nbits = Nbits_list(i);
    
    xq = quanti(xt, FS, Nbits);
    
    xqblocks = reshape(xq, M, blocks);
    X = fft(xqblocks, M);
    
    P_avg = mean(abs(X).^2, 2);
    P_dbfs = 10 * log10(P_avg / norm_const);
    
    sqnr_teorico = 6.02 * Nbits + 1.76;
    piso_ruido_teorico = -sqnr_teorico - 10*log10(M/2);
    
    figure;
    
    plot(f_axis(1:M/2 + 1) / 1e6, P_dbfs(1:M/2 + 1), 'b');
    hold on;
    yline(piso_ruido_teorico, 'r--', 'LineWidth', 1.5);
    hold off;
    
    grid on;
    xlabel('Frequency (MHz)');
    ylabel('Power (dBFS)');
    title(sprintf('Espectro (N = %d bits, M = 1024)', Nbits));
    ylim([-140, 10]);
    legend('Espectro medido', ...
           sprintf('Noise Floor (%.2f dBFS)', piso_ruido_teorico));
end
\end{lstlisting}


\subsubsection{task4\_4.m}
\begin{lstlisting}[language=Matlab]
f0 = 37.1094e6;
M = 1024;
fs = 100e6;
Nbits = 12;
blocks = 15;
FS = 1;

Nsamples = blocks * M;

n = 0:Nsamples-1;
xt = (1/3)*FS * cos(2*n*pi*f0/fs);
xq = quanti(xt, FS, Nbits);

xqblocks = reshape(xq, M, 15);

X = fft(xqblocks, M);

P_avg = mean(abs(X).^2, 2);

norm_const = (M / 2)^2;
P_dbfs = 10 * log10(P_avg / norm_const);

f_axis = (0:M-1) * fs / M; 
plot(f_axis(1:M/2 + 1) / 1e6, P_dbfs(1:M/2 + 1));
grid on;
xlabel('Frequency (MHz)');
ylabel('Power (dBFS)');
title('Espectro de Potencia Promedio (N=12 bits, M=1024)');
ylim([-140, 10]);
\end{lstlisting}

\subsubsection{task4\_5.m}
\begin{lstlisting}[language=Matlab]

f0 = 37.12e6;
M = 1024;
fs = 100e6;
Nbits = 16;
%blocks = 15;
blocks = 100;
FS = 1;

Nsamples = blocks * M;

n = 0:Nsamples-1;
xt = FS * cos(2*n*pi*f0/fs);
xq = quanti(xt, FS, Nbits);

xqblocks = reshape(xq, M, blocks);

X = fft(xqblocks, M);

P_avg = mean(abs(X).^2, 2);

norm_const = (M / 2)^2;
P_dbfs = 10 * log10(P_avg / norm_const);

f_axis = (0:M-1) * fs / M;
plot(f_axis(1:M/2 + 1) / 1e6, P_dbfs(1:M/2 + 1));
grid on;
xlabel('Frequency (MHz)');
ylabel('Power (dBFS)');
title('Espectro de Potencia Promedio (N=16 bits, M=1024)');
ylim([-140, 10]);
\end{lstlisting}

\subsection{Task 5}
\subsubsection{task5\_1.m}
\begin{lstlisting}[language=Matlab]
FS = 1;
x = linspace(-FS, FS, 1000);

g_0 = x;

gama_1 = 1;
g_1 = sign(x) .* (FS / log(1 + gama_1)) .* log(1 + gama_1 .* abs(x) / FS);
g_1(x == 0) = 0;

gama_2 = 2;
g_2 = sign(x) .* (FS / log(1 + gama_2)) .* log(1 + gama_2 .* abs(x) / FS);
g_2(x == 0) = 0;

figure;
plot(x, g_0, 'b', 'LineWidth', 2);
hold on;
plot(x, g_1, 'r', 'LineWidth', 2);
plot(x, g_2, 'g', 'LineWidth', 2);
grid on;
xlabel('x');
ylabel('g(x)');
title('Distortion Function g_\gamma(x)');
legend('\gamma = 0 (Ideal)', '\gamma = 1', '\gamma = 2');
\end{lstlisting}

\subsubsection{task5\_3.m}
\begin{lstlisting}[language=Matlab]

\end{lstlisting}


\subsubsection{task5\_4.m}
\begin{lstlisting}[language=Matlab]
f0 = 6.8359e6;
fs = 100e6;
FS = 1;
gamma_list = [0.01,0.1];
Nbits = 11;
M = 2048;
blocks = 15;

for gamma = gamma_list

    Nsamples = blocks * M;
    n = (0:Nsamples-1).';
    xt = FS * cos(2*pi*f0/fs * n);

    xq = dquanti(xt, FS, Nbits, gamma);
    xqblocks = reshape(xq, M, blocks);

    X = fft(xqblocks, M);

    P_avg = mean(abs(X).^2, 2);

    k0 = round(f0 * M / fs);
    n0 = (0:M-1).';
    xref = FS * cos(2*pi*(k0/M) * n0);
    Pref = max(abs(fft(xref, M)).^2);

    half = 1:(M/2);
    freqs = (half-1) * (fs / M);
    P_half = P_avg(half);

    P_dbfs = 10*log10( P_half / Pref );

    figure('Name',sprintf('M=%d, y=%.2f',M,gamma));
    plot(freqs/1e6, P_dbfs, 'LineWidth', 1.2);
    title(sprintf('PSD averaged, M=%d, N=%d, \\gamma=%.4g', M, Nbits, gamma));
    legend('PSD (avg)');
    grid on;
end

\end{lstlisting}

\subsubsection{task5\_5.m}
\begin{lstlisting}[language=Matlab]
f0 = 6.8359e6;
fs = 100e6;
FS = 1;
gamma_list = [0.005,0.05,0.1];
Nbits = 11;
M_list = [2048, 512];
blocks = 15;

for M = M_list
    for gamma = gamma_list
        Nsamples = blocks * M;
        n = (0:Nsamples-1).';
        xt = (FS/3) * cos(2*pi*f0/fs * n);
    
        xq = dquanti(xt, FS, Nbits, gamma);
        xqblocks = reshape(xq, M, blocks);
    
        X = fft(xqblocks, M);
    
        P_avg = mean(abs(X).^2, 2);
    
        k0 = round(f0 * M / fs);
        n0 = (0:M-1).';
        xref = FS * cos(2*pi*(k0/M) * n0);
        Pref = max(abs(fft(xref, M)).^2);
    
        half = 1:(M/2);
        freqs = (half-1) * (fs / M);
        P_half = P_avg(half);
    
        P_dbfs = 10*log10( P_half / Pref );
    
        figure('Name',sprintf('M=%d, y=%.2f',M,gamma));
        plot(freqs/1e6, P_dbfs, 'LineWidth', 1.2);
        title(sprintf('PSD averaged, M=%d, N=%d, \\gamma=%.4g', M, Nbits, gamma));
        legend('PSD (avg)');
        grid on;
    end
end
\end{lstlisting}

\subsubsection{task5\_6.m}
\begin{lstlisting}[language=Matlab]
f0 = 3.3202e6;
fs = 100e6;
FS = 1;
gamma = 0.3;
Nbits = 11;
M = 2048;
blocks = 15;

Nsamples = blocks * M;
n = (0:Nsamples-1).';
xt = (FS/2) * cos(2*pi*f0/fs * n);

xq = dquanti(xt, FS, Nbits, gamma);
xqblocks = reshape(xq, M, blocks);

X = fft(xqblocks, M);

P_avg = mean(abs(X).^2, 2);

k0 = round(f0 * M / fs);
n0 = (0:M-1).';
xref = FS * cos(2*pi*(k0/M) * n0);
Pref = max(abs(fft(xref, M)).^2);

half = 1:(M/2);
freqs = (half-1) * (fs / M);
P_half = P_avg(half);

P_dbfs = 10*log10( P_half / Pref );

figure('Name',sprintf('M=%d, y=%.2f',M,gamma));
plot(freqs/1e6, P_dbfs, 'LineWidth', 1.2);
title(sprintf('PSD averaged, M=%d, N=%d, \\gamma=%.4g', M, Nbits, gamma));
legend('PSD (avg)');
grid on;

\end{lstlisting}

\subsection{Task 6}
\subsubsection{task6\_3.m}
\begin{lstlisting}[language=Matlab]
fs = 100e6;
Nbits = 12;
FS = 1;
M = 1024;
blocks = 100;
Nsamples = M * blocks;

k0 = 410;
fc = k0 * fs / M;


sigma_list_ps = [10, 0.1];

f_axis = (0:M/2) * fs / M;

Pq_dBFS = -(6.02 * Nbits + 1.76);
Pq_linear = 10^(Pq_dBFS / 10);
FFT_gain_dB = 10 * log10(M / 2);

for sigma_ps = sigma_list_ps
    sigma_tau = sigma_ps * 1e-12;
    
    SNR_jitter_dB = 20 * log10(1 / (2 * pi * fc * sigma_tau));
    Pj_dBFS = -SNR_jitter_dB;
    Pj_linear = 10^(Pj_dBFS / 10);
    
    P_total_linear = Pq_linear + Pj_linear;
    P_total_dBFS = 10 * log10(P_total_linear);
    
    Expected_Floor_dBFS = P_total_dBFS - FFT_gain_dB;
    
    fprintf('--- Caso sigma = %.1f ps ---\n', sigma_ps);
    fprintf('  P_cuantizacion (Pq): %.2f dBFS\n', Pq_dBFS);
    fprintf('  P_jitter (Pj):       %.2f dBFS\n', Pj_dBFS);
    fprintf('  P_ruido_total:       %.2f dBFS\n', P_total_dBFS);
    fprintf('  Piso FFT Esperado:   %.2f dBFS\n', Expected_Floor_dBFS);
    
    if sigma_ps == 10
        floor_10ps = Expected_Floor_dBFS;
    else
        floor_0_1ps = Expected_Floor_dBFS;
    end
end


for sigma_ps = sigma_list_ps
    sigma_tau = sigma_ps * 1e-12;
    
    n = (0:Nsamples-1)';
    t_ideal = n / fs;
    
    a = sigma_tau * sqrt(3);
    tau_n = -a + (2 * a) * rand(Nsamples, 1);
    
    t_jittered = t_ideal + tau_n;
    
    xt = FS * cos(2 * pi * fc * t_jittered);
    
    xq = quanti(xt, FS, Nbits);
    
    xq_blocks = reshape(xq, M, blocks);
    
    X_fft = fft(xq_blocks, M);
    P_avg = mean(abs(X_fft).^2, 2);
    
    norm_const = (M / 2)^2;
    P_dbfs = 10 * log10(P_avg / norm_const);
    
    figure;
    plot(f_axis / 1e6, P_dbfs(1:M/2 + 1));
    hold on;
    
    if sigma_ps == 10
        yline(floor_10ps, 'r--', 'LineWidth', 2, ...
            'Label', sprintf('Piso Teorico (%.1f dBFS)', floor_10ps));
    else
        yline(floor_0_1ps, 'r--', 'LineWidth', 2, ...
            'Label', sprintf('Piso Teorico (%.1f dBFS)', floor_0_1ps));
    end
    
    grid on;
    title(sprintf('Efecto del Jitter (\\sigma_{\\tau} = %.1f ps), M=1024', sigma_ps));
    xlabel('Frecuencia (MHz)');
    ylabel('Potencia (dBFS)');
    ylim([-140, 10]);
    legend('Espectro Simulado', 'Piso Teorico Esperado');
end
\end{lstlisting}

\end{document}