\documentclass[11pt,a4paper]{article}
\usepackage[utf8]{inputenc}
\usepackage[spanish]{babel}
\usepackage{graphicx}
\usepackage{amsmath,amssymb,amsfonts}
\usepackage{mathtools}
\usepackage{geometry}
\usepackage{xcolor}
\usepackage{hyperref}
\usepackage{booktabs}
\usepackage{enumitem}
\usepackage{fancyhdr}
\usepackage{titlesec}
\usepackage{microtype}
\usepackage{float}

% Establecer márgenes
\geometry{a4paper, margin=1in, top=1.2in, headheight=15pt}

% Configurar hipervínculos
\hypersetup{
    colorlinks=true,
    linkcolor=blue,
    filecolor=blue,
    citecolor=blue,
    urlcolor=blue,
    pdftitle={Práctica 1},
    pdfauthor={Renato Bedriñana Cárdenas, Hugo Blanco Demelo},
    pdfsubject={Práctica 1 - Sampling and Quantization},
}

% Configuración de encabezado y pie de página
\pagestyle{fancy}
\fancyhf{}
\fancyhead[L]{\footnotesize Práctica 1}
\fancyhead[R]{\footnotesize Sampling and Quantization}
\fancyfoot[C]{\thepage}
\renewcommand{\headrulewidth}{0.4pt}
\renewcommand{\footrulewidth}{0.4pt}

% Formato de títulos
\titleformat{\section}
  {\normalfont\large\bfseries\color{blue!70!black}}
  {\thesection}{1em}{}
\titleformat{\subsection}
  {\normalfont\normalsize\bfseries\color{blue!60!black}}
  {\thesubsection}{1em}{}

% Espacio después de secciones
\titlespacing*{\section}{0pt}{3.5ex plus 1ex minus .2ex}{2.3ex plus .2ex}
\titlespacing*{\subsection}{0pt}{3.25ex plus 1ex minus .2ex}{1.5ex plus .2ex}

% Definición de comandos para matemáticas
\newcommand{\dB}{\text{dB}}
\newcommand{\dBFS}{\text{dBFS}}

% Información del documento
\title{\vspace{-1.5cm}\Large\textbf{Práctica 1: Sampling and Quantization}}
\author{\normalsize Renato Bedriñana Cárdenas \and \normalsize Hugo Blanco Demelo}
\date{\normalsize\today}

\begin{document}

\maketitle
\section{Task 1}
\textbf{Question: Give your interpretation of the resulting graphs. Do the quantization levels correspond with the values you had expected?}

Both graphics start at the -FS point, increase the number of bits, decrease the error and add more values we can get.


\textbf{Question: For both cases, represent the quantization error as a function of input amplitude in the range $[-7, +7]$ and comment on your results. Is this error always within the $[-\Delta /2, +\Delta /2]$ interval?}

The $[-\Delta /2, +\Delta /2]$ in each case is as follows:
\begin{itemize}
    \item For N = 2 the $\Delta$ value we get is $\Delta = 3.5$, so the interval should be $[-1.75, 1.75]$.
    \item For N = 4 the $\Delta$ value we get is $\Delta = 0.875$, so the interval should be $[-0.4375, 0.4375]$.
\end{itemize}
So yes, the error is always inside the range.

% \begin{figure}[h!]
%     \centering
%     \includegraphics[width=0.8\textwidth]{error1.png}
%     \caption{Error de cuantificación en función de la amplitud de entrada para N=2.}
%     \label{fig:error_cuantificacion}
% \end{figure}

\section{Task 2}
\textbf{Question: Assume a full-scale sinusoidal input and plot the histogram of the quantization error. Do you observe what you expected, or not?}

$\Delta= \frac{2*FS}{2^N} = 0.0098$, so the $[-\frac{\Delta}{2}, +\frac{\Delta}{2}]$ interval should be $[-0.0049, +0.0049]$.
In the histogram we can see that in that interval the error is uniformly distributed, but there is an error tail in the positive extreme.
It means that there is \textbf{clipping} in the positive.

\textbf{Question: Explain the operation of the Matlab command var. Estimate the variance of the quantization error using var,
    and compare it to its theoretical value. Estimate the value (in dB) of the Signal to-Quantization Noise Ratio (SQNR)
    and compare it to its theoretical value (1).
}
ADD

\section{Task 3}
\textbf{Question: Suppose that you have an N-bit A/D converter with tunable FS, and you know that your input samples follow
    a symmetric triangular pdf in some interval $[-x_0,x_0]$. Intuitively, how would you set the FS value of your converter?
    What would the resulting rms value $\sigma_x$ in dBFS be?
}

The value of FS should be $x_0$. And the value of $\sigma_x = \frac{x_0}{\sqrt{2}}$ and in dBFS would be $20\log_{10}(\sqrt{2})$ dBFS.

\textbf{Question: Explain how to generate in Matlab samples of a random variable following a symmetric triangular pdf with
    zero mean and rms value $\sigma_0$. Check the histogram and use the commands mean and var to validate your approach
}

Rand function allows to give a distribution to generate the values following that distribution. So we need to create that distribution and then give it to rand().
\end{document}