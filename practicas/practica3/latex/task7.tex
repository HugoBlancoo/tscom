To conclude the assignment, we study the case in which the fluid flow does not remain constant with time; instead, it exhibits random fluctuations, according to
\begin{eqnarray}\label{eq:rndflow}
  \mbox{[fluid flow at time $nT$]} &=& \mbox{[fluid flow at time $(n-1)T$]} \, + \, \mbox{[random increment]},
\end{eqnarray}
where the random increment is modeled as a zero-mean Gaussian random variable with standard deviation of 0.35 cm$^3/$s.

%%%%%%%%%%%%%%%%%%%%%%%%%%%%%%%%%%%%%%%%%%%%%%%%%%%%%%%%%%%%%%%%%%%%%%%%%%%%%%%%%%%%%%%%%%%%%%%%%%%%%
%%%%%%%%%%%%%%%%%%%%%%%%%%%%%%%%%%%%%%%%%%%%%%%%%%%%%%%%%%%%%%%%%%%%%%%%%%%%%%%%%%%%%%%%%%%%%%%%%%%%%
\textbf{Question: Modify your script from Task 6 to incorporate \eqref{eq:rndflow}. Execute your Kalman filter simulation to cover up to $60$ minutes, plot the corresponding curves, and comment on your observations.}
\vspace{0.5cm}

\subsection*{Modification to the Model}

The key change from Task 6 is the addition of \textbf{process noise} to the flow equation. The flow now evolves as:
\begin{equation*}
  q_n = q_{n-1} + w_n, \quad w_n \sim \mathcal{N}(0, 0.35^2) \text{ cm}^3\text{/s}.
\end{equation*}

This is implemented by updating the process noise covariance matrix:
\begin{equation*}
  \mathbf{Q} = \begin{bmatrix} 0 & 0 \\ 0 & 0.35^2 \end{bmatrix}.
\end{equation*}

In Task 6, we had $\mathbf{Q} = \mathbf{0}$ (perfect model). Now, the flow is stochastic.

\subsection*{Simulation Results}

\begin{figure}[H]
  \centering
  \includegraphics[width=0.85\textwidth]{img/task7_level.png}
  \caption{Fluid level estimation with random flow. The true level (black dashed) follows an irregular trajectory due to flow fluctuations. The Kalman filter estimates (blue/red) successfully track the unpredictable changes.}
  \label{fig:task7_level}
\end{figure}

\begin{figure}[H]
  \centering
  \includegraphics[width=0.85\textwidth]{img/task7_flow.png}
  \caption{Fluid flow estimation with random fluctuations. The true flow (black dashed) varies randomly over time. The filter estimates (blue/red) track the general trend while smoothing rapid fluctuations.}
  \label{fig:task7_flow}
\end{figure}

\subsection*{Discussion}

\textbf{Fluid Level (Figure~\ref{fig:task7_level}):}
\begin{enumerate}
  \item The true level no longer increases linearly as in Task 6. Instead, it follows an irregular trajectory reflecting the random flow variations.
  \item The Kalman filter estimates track the true level successfully despite the unpredictable changes, remaining close to the true trajectory.
  \item The filter adapts to the random behavior, demonstrating robustness to process uncertainty.
\end{enumerate}

\textbf{Fluid Flow (Figure~\ref{fig:task7_flow}):}
\begin{enumerate}
  \item The true flow fluctuates randomly between approximately 15 and 35 cm$^3$/s, rather than remaining constant at 33 cm$^3$/s as in Task 6.
  \item The filter estimates track these variations with some smoothing, capturing the general trend without following every rapid fluctuation.
  \item The estimates show more variability than in Task 6, reflecting the increased uncertainty introduced by the random process.
\end{enumerate}

\textbf{Key Difference from Task 6:}

In Task 6, the system was \textbf{deterministic} (constant flow). Once the filter learned the flow value, it could predict future states accurately. In Task 7, the system is \textbf{stochastic}—the flow changes unpredictably at each step. The filter must continuously balance trusting its predictions versus incorporating new measurements.

%%%%%%%%%%%%%%%%%%%%%%%%%%%%%%%%%%%%%%%%%%%%%%%%%%%%%%%%%%%%%%%%%%%%%%%%%%%%%%%%%%%%%%%%%%%%%%%%%%%%%
%%%%%%%%%%%%%%%%%%%%%%%%%%%%%%%%%%%%%%%%%%%%%%%%%%%%%%%%%%%%%%%%%%%%%%%%%%%%%%%%%%%%%%%%%%%%%%%%%%%%%
\textbf{Question: What are the asymptotic values of the standard deviation of the estimation error of: (i)  the fluid level (in cm); and (ii) the fluid flow (in cm$^3/$s)?}
\vspace{0.5cm}

\begin{figure}[h]
  \centering
  \includegraphics[width=0.85\textwidth]{img/task7_std.png}
  \caption{Evolution of estimation uncertainty. Solid lines: level uncertainty (cm). Dashed lines: flow uncertainty (cm$^3$/s). Both uncertainties stabilize after approximately 20 minutes.}
\end{figure}

From the uncertainty plot (Figure X), we observe that both uncertainties decrease initially and then stabilize at constant values:

\begin{itemize}
  \item \textbf{Level estimation error:} stabilizes at approximately \textbf{3.0--3.5 cm}
  \item \textbf{Flow estimation error:} stabilizes at approximately \textbf{3.3--3.4 cm$^3$/s}
\end{itemize}

\subsubsection*{Why Do They Stabilize?}

In Task 6 (constant flow), the uncertainties kept decreasing because the system was deterministic—more measurements always meant more knowledge.

In Task 7 (random flow), the uncertainties reach a floor because:
\begin{itemize}
  \item The random flow introduces new uncertainty at every time step (process noise).
  \item The measurements provide information that reduces uncertainty.
  \item Eventually, these two effects balance out: the rate at which uncertainty is added (by randomness) equals the rate at which it is removed (by measurements).
\end{itemize}

This steady-state represents the best the filter can do given the inherent randomness of the system.

\subsubsection*{Comparison with Task 6}

\begin{itemize}
  \item \textbf{Task 6 (constant flow):} Level uncertainty $\approx$ 2-3 cm, flow uncertainty $\approx$ 0.2-0.3 cm$^3$/s after 60 minutes (still decreasing).
  \item \textbf{Task 7 (random flow):} Level uncertainty $\approx$ 3-3.5 cm, flow uncertainty $\approx$ 3.3-3.4 cm$^3$/s after 20 minutes (stabilized).
\end{itemize}

The key observation is that the flow uncertainty in Task 7 ($\sim$3.3 cm$^3$/s) is much higher than in Task 6 ($\sim$0.3 cm$^3$/s), and it stabilizes near the process noise level (0.35 cm$^3$/s). This makes sense: when the flow changes randomly by $\pm$0.35 cm$^3$/s at each step, the filter cannot estimate it more precisely than that fundamental randomness.

\textbf{Conclusion:} With process noise, the Kalman filter reaches a steady state where estimation uncertainty no longer decreases. This steady-state value depends on the balance between measurement quality and process randomness.
