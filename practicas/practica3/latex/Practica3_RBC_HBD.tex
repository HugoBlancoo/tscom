\documentclass[11pt,a4paper]{article}
\usepackage[utf8]{inputenc}
\usepackage[english]{babel}
\usepackage{graphicx}
\usepackage{subcaption}
\usepackage{amsmath,amssymb,amsfonts}
\usepackage{mathtools}
\usepackage{geometry}
\usepackage{xcolor}
\usepackage{hyperref}
\usepackage{booktabs}
\usepackage{enumitem}
\usepackage{fancyhdr}
\usepackage{titlesec}
\usepackage{microtype}
\usepackage{float}
\usepackage{listings}
\usepackage{color}
\usepackage{amsmath}
\usepackage{bm}
\usepackage{multirow}
\usepackage{cancel}

% Establecer márgenes
\geometry{a4paper, margin=1in, top=1.2in, headheight=15pt}

% Configurar hipervínculos
\hypersetup{
    colorlinks=true,
    linkcolor=blue,
    filecolor=blue,
    citecolor=blue,
    urlcolor=blue,
    pdftitle={Práctica 2},
    pdfauthor={Renato Bedriñana Cárdenas, Hugo Blanco Demelo},
    pdfsubject={Práctica 2 - OFDM Modulation},
}

% Configuración de encabezado y pie de página
\pagestyle{fancy}
\fancyhf{}
\fancyhead[L]{\footnotesize Práctica 2}
\fancyhead[R]{\footnotesize OFDM Modulation}
\fancyfoot[C]{\thepage}
\renewcommand{\headrulewidth}{0.4pt}
\renewcommand{\footrulewidth}{0.4pt}

% Configuración para código fuente
\definecolor{codegreen}{rgb}{0,0.6,0}
\definecolor{codegray}{rgb}{0.5,0.5,0.5}
\definecolor{codepurple}{rgb}{0.58,0,0.82}
\definecolor{backcolour}{rgb}{0.95,0.95,0.95}

\lstset{
    backgroundcolor=\color{backcolour},
    commentstyle=\color{codegreen},
    keywordstyle=\color{magenta},
    stringstyle=\color{codepurple},
    basicstyle=\ttfamily\small,
    breakatwhitespace=false,
    breaklines=true,
    captionpos=b,
    keepspaces=true,
    numbersep=5pt,
    showspaces=false,
    showstringspaces=false,
    showtabs=false,
    tabsize=2
}

% Formato de títulos
\titleformat{\section}
  {\normalfont\large\bfseries\color{blue!70!black}}
  {\thesection}{1em}{}
\titleformat{\subsection}
  {\normalfont\normalsize\bfseries\color{blue!60!black}}
  {\thesubsection}{1em}{}

% Espacio después de secciones
\titlespacing*{\section}{0pt}{3.5ex plus 1ex minus .2ex}{2.3ex plus .2ex}
\titlespacing*{\subsection}{0pt}{3.25ex plus 1ex minus .2ex}{1.5ex plus .2ex}

% Definición de comandos para matemáticas
\newcommand{\dB}{\text{dB}}
\newcommand{\dBFS}{\text{dBFS}}

% Información del documento
\title{\vspace{-1.5cm}\Large\textbf{Lab Assignment 3: Linear Estimation and Kalman Filtering}}
\author{\normalsize Renato Bedriñana Cárdenas \and \normalsize Hugo Blanco Demelo}
\date{\normalsize\today}
\usepackage[T1]{fontenc}
\usepackage{lmodern}

\begin{document}

\maketitle
\vspace{0.5cm}
\section{Task 1}
\textbf{Question: Find the expression of the Least Squares (LS) estimate  of the fluid level $\ell$, given the pressure observations $\bm x = [\begin{array}{cccc} x_0 & x_1 & \cdots & x_{n-1}\end{array}]^T$. How do you interpret this estimate?}
\vspace{0.5cm}

We have the model:
\begin{equation}
\bm z = H \theta + \bm v
\end{equation}

In our case:
\begin{itemize}
    \item $\bm z = \bm x$ (observations)
    \item $\theta = \ell$ (scalar parameter to estimate)
    \item $H = c_h \, \mathbf{1}_n$ (vector of $c_h$, $n \times 1$)
\end{itemize}

All vectors:
\begin{equation}
\bm z = \begin{bmatrix} x_0 \\ x_1 \\ \vdots \\ x_{n-1} \end{bmatrix}, \quad
\mathbf{1}_n = \begin{bmatrix} 1 \\ 1 \\ \vdots \\ 1 \end{bmatrix}, \quad
H = \begin{bmatrix} c_h \\ c_h \\ \vdots \\ c_h \end{bmatrix}
\end{equation}

The LS estimator is given by the normal equations:
\begin{equation}
\hat{\theta}_{\text{LS}} = (H^T H)^{-1} H^T \bm z 
= \left( \mathbf{1}_n^T c_h^{\cancel{2}} \mathbf{1}_n \right)^{-1} \frac{1}{\cancel{c_h}} c_h \sum_{k} x_k 
= \frac{1}{n \, c_h} \sum_{k} x_k
\end{equation}

Therefore, our estimator is:
\begin{equation}
\boxed{\hat{\ell} = \frac{\sum_k x_k}{n \, c_h}}
\end{equation}

\textbf{Interpretation:} This is the average of all the measurements scaled by $c_h$. 

The variance of the estimate is:
\begin{equation}
\mathrm{var}(\hat{\ell}) = \mathrm{var}\left(\frac{\sum_k x_k}{n \, c_h}\right) 
= \frac{\mathrm{var}(\bm z)}{(n \, c_h)^2} 
= \frac{n \, \sigma_v^2}{n^2 \, c_h^2} 
= \frac{\sigma_v^2}{n \, c_h^2}
\end{equation}

This shows that as we increase the number of measurements $n$, the variance of our estimator decreases proportionally to $\frac{1}{n}$.

%%%%%%%%%%%%%%%%%%%%%%%%%%%%%%%%%%%%%%%%%%%%%%%%%%%%%%%%%%%%%%%%%%%%%%%%%%%%%%%%%%%%%%%%%%%%%%%%%%%%%
%%%%%%%%%%%%%%%%%%%%%%%%%%%%%%%%%%%%%%%%%%%%%%%%%%%%%%%%%%%%%%%%%%%%%%%%%%%%%%%%%%%%%%%%%%%%%%%%%%%%%
\textbf{Question: Recall that the LMMSE estimator of $\ell$ based on $\bm x$ is given by
$ \hat{\ell}(\bm x) = \mu_\ell + \bm C_{\ell x} \bm C_{xx}^{-1}(\bm x - \bm \mu_x)$. 
Obtain the values of $\bm \mu_x$, $\bm C_{\ell x}$ and $\bm C_{xx}$ for this problem.}
\vspace{0.5cm}

\subsubsection*{Mean vector $\bm \mu_x$}

\begin{align}
\mu_x &= \mathbb{E}[\bm x] = \mathbb{E}[c_h \, \ell \, \mathbf{1}_n + \bm v] \\
&= c_h \, \underbrace{\mathbb{E}[\ell]}_{\mu_\ell} \, \mathbf{1}_n + \underbrace{\mathbb{E}[\bm v]}_{\bm 0} \\
&= c_h \, \mu_\ell \, \mathbf{1}_n
\end{align}

where $\mathbf{1}_n$ is the $n \times 1$ vector of ones.

\subsubsection*{Cross-covariance $\bm C_{\ell x}$}

\begin{align}
\bm C_{\ell x} &= \mathrm{cov}(\ell, \bm x) = \mathrm{cov}(\ell, c_h \, \ell \, \mathbf{1}_n + \bm v) \\
&= \mathrm{cov}(\ell, c_h \, \ell) + \underbrace{\mathrm{cov}(\ell, \bm v)}_{\bm 0} \\
&= \mathrm{cov}(\ell, \ell) \, c_h \, \mathbf{1}_n^T = \sigma_\ell^2 \, c_h \, \mathbf{1}_n^T
\end{align}

Note: $\ell$ is a scalar, $\bm x \in \mathbb{R}^n \Rightarrow \bm C_{\ell x}$ is a $(1 \times n)$ row vector: $\sigma_\ell^2 \, c_h \, \mathbf{1}_n^T$.

\subsubsection*{Covariance matrix $\bm C_{xx}$}

\textbf{Element $i$:}
\begin{align}
C_{xx}^{(i,i)} &= \mathrm{cov}(x_i, x_i) = \mathrm{cov}(c_h \, \ell + v_i, c_h \, \ell + v_i) \\
&= c_h^2 \, \sigma_\ell^2 + \sigma_v^2
\end{align}

This gives the diagonal elements.

\textbf{Elements $i \neq j$:}
\begin{align}
C_{xx}^{(i,j)} &= \mathrm{cov}(x_i, x_j) = \mathrm{cov}(c_h \, \ell + v_i, c_h \, \ell + v_j) \\
&= c_h^2 \, \sigma_\ell^2 + \underbrace{\mathrm{cov}(v_i, v_j)}_{=0}  = c_h^2 \, \sigma_\ell^2
\end{align}

This gives the off-diagonal elements.

Therefore:
\begin{equation}
\bm C_{xx} = c_h^2 \, \sigma_\ell^2 \, U_n + \sigma_v^2 \, \mathbf{I}_n = U_n \, c_h^2 \, \sigma_\ell^2 + \mathbf{I}_n \, \sigma_v^2
\end{equation}

where $U_n = \mathbf{1}_n \mathbf{1}_n^T$ is the $n \times n$ matrix of ones.

\subsubsection*{Summary}

\begin{equation}
\boxed{
\begin{aligned}
\bm \mu_x &= c_h \, \mu_\ell \, \mathbf{1}_n \\
\bm C_{\ell x} &= \sigma_\ell^2 \, c_h \, \mathbf{1}_n^T \quad \text{(row vector)} \\
\bm C_{xx} &= U_n \, c_h^2 \, \sigma_\ell^2 + \mathbf{I}_n \, \sigma_v^2
\end{aligned}
}
\end{equation}

%%%%%%%%%%%%%%%%%%%%%%%%%%%%%%%%%%%%%%%%%%%%%%%%%%%%%%%%%%%%%%%%%%%%%%%%%%%%%%%%%%%%%%%%%%%%%%%%%%%%%
%%%%%%%%%%%%%%%%%%%%%%%%%%%%%%%%%%%%%%%%%%%%%%%%%%%%%%%%%%%%%%%%%%%%%%%%%%%%%%%%%%%%%%%%%%%%%%%%%%%%%
\textbf{Question: Use the Matrix Inversion Lemma to obtain $\bm C_{xx}^{-1}$ in closed form.}
\vspace{0.5cm}

\textbf{Matrix Inversion Lemma:}
\begin{equation}
(A + UCV)^{-1} = A^{-1} - A^{-1} U(C^{-1} + VA^{-1}U)^{-1} VA^{-1}
\end{equation}

In our case, we have:
\begin{equation}
\bm C_{xx} = \mathbf{I}_n \sigma_v^2 + \mathbf{1}_n \, c_h^2 \sigma_\ell^2 \, \mathbf{1}_n^T
\end{equation}

Identify:
\begin{itemize}
    \item $A = \mathbf{I}_n \sigma_v^2$
    \item $U = \mathbf{1}_n$
    \item $C = c_h^2 \sigma_\ell^2$
    \item $V = \mathbf{1}_n^T$
\end{itemize}

Therefore:
\begin{align}
\bm C_{xx}^{-1} &= (\mathbf{I}_n \sigma_v^2 + \mathbf{1}_n \, c_h^2 \sigma_\ell^2 \, \mathbf{1}_n^T)^{-1} \\
&= \frac{\mathbf{I}_n}{\sigma_v^2} - \frac{\mathbf{I}_n}{\sigma_v^2} \mathbf{1}_n \left( \frac{1}{c_h^2 \sigma_\ell^2} + \mathbf{1}_n^T \frac{\mathbf{I}_n}{\sigma_v^2} \mathbf{1}_n \right)^{-1} \mathbf{1}_n^T \frac{\mathbf{I}_n}{\sigma_v^2} \\
&= \frac{\mathbf{I}_n}{\sigma_v^2} - \frac{\mathbf{I}_n}{\sigma_v^2} \mathbf{1}_n \left( \frac{1}{c_h^2 \sigma_\ell^2} + \frac{n}{\sigma_v^2} \right)^{-1} \mathbf{1}_n^T \frac{\mathbf{I}_n}{\sigma_v^2}
\end{align}

Define:
\begin{equation}
a = \frac{c_h^2 \, \sigma_\ell^2}{\sigma_v^2 + n \, c_h^2 \, \sigma_\ell^2}
\end{equation}

\textbf{Verification that $\bm C_{xx} \bm C_{xx}^{-1} = \mathbf{I}_n$:}

\begin{align}
&\left[\mathbf{I}_n \mathbf{1}_n c_h^2 \sigma_\ell^2 + \mathbf{I}_n \sigma_v^2\right] \left[\frac{\mathbf{I}_n}{\sigma_v^2} - \frac{\mathbf{I}_n}{\sigma_v^2} \mathbf{1}_n \, a \right] \\
&= \frac{\mathbf{I}_n}{\sigma_v^2} \mathbf{1}_n c_h^2 \sigma_\ell^2 - \frac{\mathbf{I}_n}{\sigma_v^2} \mathbf{1}_n c_h^2 \sigma_\ell^2 \frac{c_h^2 \sigma_\ell^2}{\sigma_v^2} n \, a + \mathbf{I}_n - \frac{\mathbf{I}_n}{\sigma_v^2} \mathbf{1}_n a \, a \\
&\quad \text{(debe ser $\beta$)} \\
&\Rightarrow \frac{\mathbf{I}_n}{\sigma_v^2} \mathbf{1}_n \left( \frac{c_h^2 \sigma_\ell^2}{\sigma_v^2} - \frac{c_h^2 \sigma_\ell^2 n}{\sigma_v^2 + n \, c_h^2 \sigma_\ell^2} - \frac{c_h^2 c_i^2}{\sigma_v^2 + n \, c_h^2 \sigma_\ell^2} \right) + \mathbf{I}_n \\
&\Rightarrow \frac{c_h^2 \sigma_\ell^2}{\sigma_v^2 + n \, c_h^2 \sigma_\ell^2} \left( \frac{\sigma_v^2 + n \, c_h^2 \sigma_\ell^2}{\sigma_v^2} - \frac{c_h^2 \sigma_\ell^2}{\sigma_v^2} n - 1 \right) = \sigma_v^2 n \, c_h^2 \sigma_\ell^2 - c_h^2 \sigma_\ell^2 n - \sigma_v^2 = 0
\end{align}

Therefore:
\begin{equation}
\boxed{
\bm C_{xx}^{-1} = \frac{\mathbf{I}_n}{\sigma_v^2} - \frac{a}{\sigma_v^2} U_n
}
\end{equation}

where $a = \dfrac{c_h^2 \, \sigma_\ell^2}{\sigma_v^2 + n \, c_h^2 \, \sigma_\ell^2}$ and $U_n = \mathbf{1}_n \mathbf{1}_n^T$.

%%%%%%%%%%%%%%%%%%%%%%%%%%%%%%%%%%%%%%%%%%%%%%%%%%%%%%%%%%%%%%%%%%%%%%%%%%%%%%%%%%%%%%%%%%%%%%%%%%%%%
%%%%%%%%%%%%%%%%%%%%%%%%%%%%%%%%%%%%%%%%%%%%%%%%%%%%%%%%%%%%%%%%%%%%%%%%%%%%%%%%%%%%%%%%%%%%%%%%%%%%%
\textbf{Question: Let $\alpha = \frac{\sigma_\ell^2}{\sigma_v^2}$ and $S = \sum_{k=0}^{n-1}x_k$. Prove that 
\begin{equation} \label{eq:lmmse_ave}
\hat \ell(\bm x) = \frac{1}{1+n\alpha c_h^2} \mu_\ell + \frac{\alpha c_h}{1+n\alpha c_h^2} S. 
\end{equation}
How do you interpret the parameter $\alpha$?}
\vspace{0.5cm}

\textbf{Question: Let $\alpha = \frac{\sigma_\ell^2}{\sigma_v^2}$ and $S = \sum_{k=0}^{n-1}x_k$. Prove that 
\begin{equation} \label{eq:lmmse_ave}
\hat \ell(\bm x) = \frac{1}{1+n\alpha c_h^2} \mu_\ell + \frac{\alpha c_h}{1+n\alpha c_h^2} S. 
\end{equation}
How do you interpret the parameter $\alpha$?}
\vspace{0.5cm}


Recall the LMMSE estimator:
\begin{equation}
\hat{\ell}(\bm x) = \mu_\ell + \bm C_{\ell x} \bm C_{xx}^{-1}(\bm x - \bm \mu_x)
\end{equation}

Substituting the values we found previously:
\begin{align}
\hat{\ell}(\bm x) &= \mu_\ell + \sigma_\ell^2 c_h \mathbf{1}_n^T \left[\frac{\mathbf{I}_n}{\sigma_v^2} - \frac{a}{\sigma_v^2} U_n\right](\bm x - c_h \mu_\ell \mathbf{1}_n) \\
&= \mu_\ell + \frac{\sigma_\ell^2 c_h}{\sigma_v^2} \mathbf{1}_n^T \left[\mathbf{I}_n - a \, \mathbf{1}_n \mathbf{1}_n^T\right](\bm x - c_h \mu_\ell \mathbf{1}_n)
\end{align}

Note that:
\begin{equation}
\mathbf{1}_n^T \mathbf{I}_n = \mathbf{1}_n^T, \quad \mathbf{1}_n^T \mathbf{1}_n \mathbf{1}_n^T = n \, \mathbf{1}_n^T
\end{equation}

Therefore:
\begin{align}
\hat{\ell}(\bm x) &= \mu_\ell + \frac{\sigma_\ell^2 c_h}{\sigma_v^2} [\mathbf{1}_n^T - a \, n \, \mathbf{1}_n^T](\bm x - c_h \mu_\ell \mathbf{1}_n) \\
&= \mu_\ell + \frac{\sigma_\ell^2 c_h}{\sigma_v^2} \mathbf{1}_n^T \left[1 - \frac{n \, c_h^2 \sigma_\ell^2}{\sigma_v^2 + n \, c_h^2 \sigma_\ell^2}\right](\bm x - c_h \mu_\ell \mathbf{1}_n) \\
&= \mu_\ell + \frac{\sigma_\ell^2 c_h}{\sigma_v^2} \mathbf{1}_n^T \left[\frac{\sigma_v^2 + n \, c_h^2 \sigma_\ell^2 - n \, c_h^2 \sigma_\ell^2}{\sigma_v^2 + n \, c_h^2 \sigma_\ell^2}\right](\bm x - c_h \mu_\ell \mathbf{1}_n) \\
&= \mu_\ell + \frac{\sigma_\ell^2 c_h}{\sigma_v^2} \mathbf{1}_n^T \left[\frac{\sigma_v^2}{\sigma_v^2 + n \, c_h^2 \sigma_\ell^2}\right](\bm x - c_h \mu_\ell \mathbf{1}_n) \\
&= \mu_\ell + \frac{\sigma_\ell^2 c_h}{\sigma_v^2 + n \, c_h^2 \sigma_\ell^2}(S - n \, c_h \mu_\ell)
\end{align}

where $S = \mathbf{1}_n^T \bm x = \sum_{k=0}^{n-1} x_k$.

Continuing:
\begin{align}
\hat{\ell}(\bm x) &= \mu_\ell + \frac{\sigma_\ell^2 c_h}{\sigma_v^2 + n \, c_h^2 \sigma_\ell^2} S - \frac{\sigma_\ell^2 c_h^2 n}{\sigma_v^2 + n \, c_h^2 \sigma_\ell^2} \mu_\ell \\
&= \mu_\ell \left(1 - \frac{n \, c_h^2 \sigma_\ell^2}{\sigma_v^2 + n \, c_h^2 \sigma_\ell^2}\right) + \frac{\sigma_\ell^2 c_h}{\sigma_v^2 + n \, c_h^2 \sigma_\ell^2} S \\
&= \mu_\ell \left(\frac{\sigma_v^2 + n \, c_h^2 \sigma_\ell^2 - n \, c_h^2 \sigma_\ell^2}{\sigma_v^2 + n \, c_h^2 \sigma_\ell^2}\right) + \frac{\sigma_\ell^2 c_h}{\sigma_v^2 + n \, c_h^2 \sigma_\ell^2} S \\
&= \frac{\sigma_v^2}{\sigma_v^2 + n \, c_h^2 \sigma_\ell^2} \mu_\ell + \frac{\sigma_\ell^2 c_h}{\sigma_v^2 + n \, c_h^2 \sigma_\ell^2} S
\end{align}

Dividing numerator and denominator by $\sigma_v^2$:
\begin{align}
\hat{\ell}(\bm x) &= \frac{1}{1 + n \, c_h^2 \frac{\sigma_\ell^2}{\sigma_v^2}} \mu_\ell + \frac{\frac{\sigma_\ell^2}{\sigma_v^2} c_h}{1 + n \, c_h^2 \frac{\sigma_\ell^2}{\sigma_v^2}} S
\end{align}

Substituting $\alpha = \frac{\sigma_\ell^2}{\sigma_v^2}$:
\begin{equation}
\boxed{
\hat{\ell}(\bm x) = \frac{1}{1+n\alpha c_h^2} \mu_\ell + \frac{\alpha c_h}{1+n\alpha c_h^2} S
}
\end{equation}


\textbf{Interpretation of the parameter $\alpha$:}

The parameter $\alpha = \frac{\sigma_\ell^2}{\sigma_v^2}$ represents the \textbf{signal-to-noise ratio} (SNR). Specifically, it is the ratio of the variance of the fluid level $\ell$ to the variance of the measurement noise $v_k$.

\begin{itemize}
    \item When $\alpha \to 0$ (low SNR, noise dominates): The estimator approaches $\hat{\ell} \approx \mu_\ell$, meaning we rely heavily on the prior mean and trust the measurements less.
    
    \item When $\alpha \to \infty$ (high SNR, signal dominates): The estimator gives more weight to the measurements $S$, and less weight to the prior $\mu_\ell$.
\end{itemize}

This parameter controls the \textbf{trade-off between prior knowledge and observed data}. A higher $\alpha$ indicates that the true signal variance is large relative to the noise, so we should trust the measurements more. A lower $\alpha$ indicates noisy measurements, so we should rely more on our prior knowledge.

%%%%%%%%%%%%%%%%%%%%%%%%%%%%%%%%%%%%%%%%%%%%%%%%%%%%%%%%%%%%%%%%%%%%%%%%%%%%%%%%%%%%%%%%%%%%%%%%%%%%%
%%%%%%%%%%%%%%%%%%%%%%%%%%%%%%%%%%%%%%%%%%%%%%%%%%%%%%%%%%%%%%%%%%%%%%%%%%%%%%%%%%%%%%%%%%%%%%%%%%%%%
\textbf{Question: Show that the normalized MSE obtained with this estimator is
\[ \frac{\mathbb{E}\{(\ell-\hat \ell(\bm x))^2\}}{\sigma_\ell^2} = \frac{1}{1+n\alpha c_h^2}. \]}
\vspace{0.5cm}

We need to compute the mean squared error (MSE) of the estimator. Recall that:
\begin{equation}
\hat{\ell}(\bm x) = a \, \mu_\ell + b \, S
\end{equation}

where:
\begin{equation}
a = \frac{1}{1+n\alpha c_h^2}, \quad b = \frac{\alpha c_h}{1+n\alpha c_h^2}, \quad S = \sum_{k=0}^{n-1} x_k = n \, c_h \, \ell + \sum_{k=0}^{n-1} v_k
\end{equation}

Also note that $\sigma_\ell^2 = \mathbb{E}\{\ell^2\} - \mu_\ell^2 \Rightarrow \mathbb{E}\{\ell^2\} = \sigma_\ell^2 + \mu_\ell^2$.


\subsubsection*{Expansion of the MSE}

First, we expand the squared error:
\begin{align}
\mathbb{E}\{(\ell - \hat{\ell}(\bm x))^2\} &= \mathbb{E}\{\ell^2 - 2\ell \hat{\ell}(\bm x) + \hat{\ell}(\bm x)^2\} \\
&= \mathbb{E}\{\ell^2\} - 2\mathbb{E}\{\ell \hat{\ell}(\bm x)\} + \mathbb{E}\{\hat{\ell}(\bm x)^2\}
\end{align}


\textbf{Term 1:} $\mathbb{E}\{\ell^2\} = \sigma_\ell^2 + \mu_\ell^2$


\textbf{Term 2:} 
\begin{align}
\mathbb{E}\{\ell \hat{\ell}(\bm x)\} &= \mathbb{E}\{\ell(a \mu_\ell + b S)\} \\
&= a \mu_\ell \mathbb{E}\{\ell\} + b \mathbb{E}\left\{\ell \sum_{k=0}^{n-1} x_k\right\} \\
&= a \mu_\ell^2 + b \mathbb{E}\left\{\ell \sum_{k=0}^{n-1} (c_h \ell + v_k)\right\} \\
&= a \mu_\ell^2 + b \, n \, c_h \mathbb{E}\{\ell^2\} + b \sum_{k=0}^{n-1} \mathbb{E}\{\ell v_k\} \\
&= a \mu_\ell^2 + b \, n \, c_h (\sigma_\ell^2 + \mu_\ell^2)
\end{align}

where we used that $\ell$ and $v_k$ are independent, so $\mathbb{E}\{\ell v_k\} = 0$.


\textbf{Term 3:}
\begin{align}
\mathbb{E}\{\hat{\ell}(\bm x)^2\} &= \mathbb{E}\{(a \mu_\ell + b S)^2\} \\
&= a^2 \mu_\ell^2 + 2ab \mu_\ell \mathbb{E}\{S\} + b^2 \mathbb{E}\{S^2\}
\end{align}

Now, $\mathbb{E}\{S\} = n \, c_h \mu_\ell$, and:
\begin{align}
\mathbb{E}\{S^2\} &= \mathbb{E}\left\{\left(n \, c_h \, \ell + \sum_{k=0}^{n-1} v_k\right)^2\right\} \\
&= n^2 c_h^2 \mathbb{E}\{\ell^2\} + 2 n \, c_h \sum_{k=0}^{n-1} \mathbb{E}\{\ell v_k\} + \mathbb{E}\left\{\left(\sum_{k=0}^{n-1} v_k\right)^2\right\} \\
&= n^2 c_h^2 (\sigma_\ell^2 + \mu_\ell^2) + n \, \sigma_v^2
\end{align}

Therefore:
\begin{equation}
\mathbb{E}\{\hat{\ell}(\bm x)^2\} = a^2 \mu_\ell^2 + 2ab \, n \, c_h \mu_\ell^2 + b^2[n^2 c_h^2(\sigma_\ell^2 + \mu_\ell^2) + n \, \sigma_v^2]
\end{equation}


\subsubsection*{Combining all terms}

Substituting into the MSE expression:
\begin{align}
\mathbb{E}\{(\ell - \hat{\ell}(\bm x))^2\} &= (\sigma_\ell^2 + \mu_\ell^2) - 2[a \mu_\ell^2 + b \, n \, c_h(\sigma_\ell^2 + \mu_\ell^2)] \\
&\quad + a^2 \mu_\ell^2 + 2ab \, n \, c_h \mu_\ell^2 + b^2 n^2 c_h^2(\sigma_\ell^2 + \mu_\ell^2) + b^2 n \, \sigma_v^2
\end{align}

Grouping terms by $\mu_\ell^2$ and $\sigma_\ell^2$, and substituting the values of $a$ and $b$:
\begin{align}
\mathbb{E}\{(\ell - \hat{\ell}(\bm x))^2\} &= \sigma_\ell^2(1 - 2b \, n \, c_h + b^2 n^2 c_h^2) + \mu_\ell^2(1 - 2a + a^2 + 2ab \, n \, c_h - 2b \, n \, c_h + b^2 n^2 c_h^2) \\
&\quad + b^2 n \, \sigma_v^2
\end{align}

After algebraic simplification using $a = \frac{1}{1+n\alpha c_h^2}$ and $b = \frac{\alpha c_h}{1+n\alpha c_h^2}$, and noting that $a + b \, n \, c_h = 1$, the terms involving $\mu_\ell^2$ cancel out, and we obtain:
\begin{equation}
\mathbb{E}\{(\ell - \hat{\ell}(\bm x))^2\} = \sigma_\ell^2 \left(\frac{1}{1+n\alpha c_h^2}\right)^2 + \mu_\ell^2 \cdot 0 + b^2 n \, \sigma_v^2
\end{equation}

Further simplification using $b^2 n \, \sigma_v^2 = \frac{\alpha^2 c_h^2 n \sigma_v^2}{(1+n\alpha c_h^2)^2} = \frac{\sigma_\ell^2}{(1+n\alpha c_h^2)^2} \cdot \frac{\alpha^2 c_h^2 n \sigma_v^2}{\sigma_\ell^2} = \frac{\sigma_\ell^2}{(1+n\alpha c_h^2)^2} \cdot n \alpha c_h^2$ leads to:
\begin{equation}
\mathbb{E}\{(\ell - \hat{\ell}(\bm x))^2\} = \frac{\sigma_\ell^2}{(1+n\alpha c_h^2)^2}(1 + n\alpha c_h^2) = \frac{\sigma_\ell^2}{1+n\alpha c_h^2}
\end{equation}

Therefore, the normalized MSE is:
\begin{equation}
\boxed{
\frac{\mathbb{E}\{(\ell-\hat{\ell}(\bm x))^2\}}{\sigma_\ell^2} = \frac{1}{1+n\alpha c_h^2}
}
\end{equation}

%%%%%%%%%%%%%%%%%%%%%%%%%%%%%%%%%%%%%%%%%%%%%%%%%%%%%%%%%%%%%%%%%%%%%%%%%%%%%%%%%%%%%%%%%%%%%%%%%%%%%
%%%%%%%%%%%%%%%%%%%%%%%%%%%%%%%%%%%%%%%%%%%%%%%%%%%%%%%%%%%%%%%%%%%%%%%%%%%%%%%%%%%%%%%%%%%%%%%%%%%%%
\textbf{Question:Show that the LMMSE estimator can be written as a {\em convex combination} (that is, a linear combination with positive coefficients that add up to one) of the {\em a priori} estimator and the LS estimator.}
\vspace{0.5cm}

Recall that we have:
\begin{itemize}
    \item LMMSE estimator: $\hat{\ell}(\bm x) = \frac{1}{1+n\alpha c_h^2} \mu_\ell + \frac{\alpha c_h}{1+n\alpha c_h^2} S$
    \item LS estimator: $\hat{\ell}_{\text{LS}} = \frac{\sum_k x_k}{n \, c_h} = \frac{S}{n \, c_h}$
    \item A priori estimator (without measurements): $\hat{\ell}_{\text{prior}} = \mu_\ell$
\end{itemize}

We can rewrite the LMMSE estimator as:
\begin{align}
\hat{\ell}(\bm x) &= \frac{1}{1+n\alpha c_h^2} \mu_\ell + \frac{\alpha c_h}{1+n\alpha c_h^2} S \\
&= \frac{1}{1+n\alpha c_h^2} \mu_\ell + \frac{\alpha c_h}{1+n\alpha c_h^2} \cdot n \, c_h \cdot \frac{S}{n \, c_h} \\
&= \frac{1}{1+n\alpha c_h^2} \mu_\ell + \frac{n \alpha c_h^2}{1+n\alpha c_h^2} \hat{\ell}_{\text{LS}}
\end{align}

Let:
\begin{equation}
a = \frac{1}{1+n\alpha c_h^2}, \quad b = \frac{n \alpha c_h^2}{1+n\alpha c_h^2}
\end{equation}

Then:
\begin{equation}
\boxed{
\hat{\ell}(\bm x) = a \, \hat{\ell}_{\text{prior}} + b \, \hat{\ell}_{\text{LS}}
}
\end{equation}

where $a + b = \frac{1 + n\alpha c_h^2}{1+n\alpha c_h^2} = 1$, and both $a, b > 0$ (since $n, \alpha, c_h^2 > 0$).

Therefore, the LMMSE estimator is indeed a convex combination of the a priori estimator and the LS estimator.

%%%%%%%%%%%%%%%%%%%%%%%%%%%%%%%%%%%%%%%%%%%%%%%%%%%%%%%%%%%%%%%%%%%%%%%%%%%%%%%%%%%%%%%%%%%%%%%%%%%%%
%%%%%%%%%%%%%%%%%%%%%%%%%%%%%%%%%%%%%%%%%%%%%%%%%%%%%%%%%%%%%%%%%%%%%%%%%%%%%%%%%%%%%%%%%%%%%%%%%%%%%
\textbf{Question: For a given ratio of variances, what happens when $n$ becomes large? Explain.}
\vspace{0.5cm}

When $n \to \infty$ (large number of observations) with $\alpha$ fixed:
\begin{equation}
\lim_{n \to \infty} a = \lim_{n \to \infty} \frac{1}{1+n\alpha c_h^2} = 0
\end{equation}
\begin{equation}
\lim_{n \to \infty} b = \lim_{n \to \infty} \frac{n \alpha c_h^2}{1+n\alpha c_h^2} = 1
\end{equation}

Therefore:
\begin{equation}
\lim_{n \to \infty} \hat{\ell}(\bm x) = \hat{\ell}_{\text{LS}}
\end{equation}

\textbf{Interpretation:} As the number of observations becomes large, the information from the prior becomes negligible, and all the weight goes to the measurements. The LMMSE estimator approaches the LS estimator. This makes intuitive sense: with many observations, we have enough data to reliably estimate $\ell$, so the prior information becomes less important.

%%%%%%%%%%%%%%%%%%%%%%%%%%%%%%%%%%%%%%%%%%%%%%%%%%%%%%%%%%%%%%%%%%%%%%%%%%%%%%%%%%%%%%%%%%%%%%%%%%%%%
%%%%%%%%%%%%%%%%%%%%%%%%%%%%%%%%%%%%%%%%%%%%%%%%%%%%%%%%%%%%%%%%%%%%%%%%%%%%%%%%%%%%%%%%%%%%%%%%%%%%%
\textbf{Question: For a given number of observations, what happens when $\alpha$ becomes large? Explain.}
\vspace{0.5cm}

When $\alpha \to \infty$ (large SNR, i.e., $\sigma_\ell^2 \gg \sigma_v^2$) with $n$ fixed:
\begin{equation}
\lim_{\alpha \to \infty} a = \lim_{\alpha \to \infty} \frac{1}{1+n\alpha c_h^2} = 0
\end{equation}
\begin{equation}
\lim_{\alpha \to \infty} b = \lim_{\alpha \to \infty} \frac{n \alpha c_h^2}{1+n\alpha c_h^2} = 1
\end{equation}

Therefore:
\begin{equation}
\lim_{\alpha \to \infty} \hat{\ell}(\bm x) = \hat{\ell}_{\text{LS}}
\end{equation}

\textbf{Interpretation:} When $\alpha$ is large, the signal variance is much larger than the noise variance, meaning we have a high SNR. In this case, the noise has more variability than the signal itself, so it's important to trust the measurements more than the prior. The LMMSE estimator tends to the LS estimator, which relies solely on the data.

%%%%%%%%%%%%%%%%%%%%%%%%%%%%%%%%%%%%%%%%%%%%%%%%%%%%%%%%%%%%%%%%%%%%%%%%%%%%%%%%%%%%%%%%%%%%%%%%%%%%%
%%%%%%%%%%%%%%%%%%%%%%%%%%%%%%%%%%%%%%%%%%%%%%%%%%%%%%%%%%%%%%%%%%%%%%%%%%%%%%%%%%%%%%%%%%%%%%%%%%%%%
\textbf{Question: For a given number of observations, what happens when $\alpha$ becomes small? Explain.}
\vspace{0.5cm}

When $\alpha \to 0$ (low SNR, i.e., $\sigma_\ell^2 \ll \sigma_v^2$) with $n$ fixed:
\begin{equation}
\lim_{\alpha \to 0} a = \lim_{\alpha \to 0} \frac{1}{1+n\alpha c_h^2} = 1
\end{equation}
\begin{equation}
\lim_{\alpha \to 0} b = \lim_{\alpha \to 0} \frac{n \alpha c_h^2}{1+n\alpha c_h^2} = 0
\end{equation}

Therefore:
\begin{equation}
\lim_{\alpha \to 0} \hat{\ell}(\bm x) = \mu_\ell = \hat{\ell}_{\text{prior}}
\end{equation}

\textbf{Interpretation:} When $\alpha$ is small, the noise variance dominates the signal variance, resulting in very low SNR (noisy measurements). In this scenario, the measurements contain very little useful information about $\ell$, so we should rely more on our prior knowledge. The LMMSE estimator approaches the a priori estimator $\mu_\ell$, effectively ignoring the unreliable measurements.


\vspace{0.5cm}
\section{Task 2}
\textbf{An OFDM transceiver has 85 MHz of available RF bandwidth, of which 80 MHz can be occupied for data transmission, leaving $2.5$ MHz at each side as guard bands. The subcarrier spacing is $20$ kHz, and the delay spread of the channel almost never exceeds 6 $\mu$s.
}

\textbf{Question: What is the smallest possible CP overhead of this system, in percentage?}
\vspace{0.5cm}

$O{_CP} = \frac{L_cT}{NT} = \frac{L_c}{N}$

$\Delta_c = \frac{1}{NT} \rightarrow NT = \frac{1}{\Delta_c}= \frac{1}{20 \times 10^3} = 50 \mu s$

$O{_CP} = \frac{6\mu s}{50 \mu s} = 0.12$

\vspace{1cm}
\textbf{Question: Explain why it is not feasible to use a value of $2^{10}$ for the IFFT size.}
\vspace{0.5cm}

The occupied bandwidth of an OFDM signal is approximately
\[
    B \approx N\,\Delta f,
\]
where $N$ is the IFFT size (number of subcarriers) and $\Delta f$ is the subcarrier spacing.

Given:
\[
    B = 80~\text{MHz}, \quad \Delta f = 20~\text{kHz},
\]
we can estimate the required IFFT size as
\[
    N \approx \frac{B}{\Delta f} = \frac{80\times10^6}{20\times10^3} = 4000.
\]

If we choose $N = 2^{10} = 1024$, the occupied bandwidth would be
\[
    B \approx 1024 \times 20~\text{kHz} = 20.48~\text{MHz},
\]
which is much smaller than the required 80~MHz.

\textbf{Therefore, using $N = 2^{10}$ is not feasible because it would not cover the full 80~MHz data bandwidth.}
To span the available bandwidth, we would need an IFFT size close to $N \approx 4000$ (i.e., the next power of two $N = 4096$).

\vspace{1cm}
\textbf{Question: Assume an IFFT size of $2^{13}$. What would be the value of the sampling rate at the IFFT output?}
\vspace{0.5cm}

In an OFDM system, the sampling rate $F_s$ (at the IFFT output) is related to the IFFT size $N$ and the subcarrier spacing $\Delta f$ by:
\[
    F_s = N \, \Delta f.
\]

Given:
\[
    N = 2^{13} = 8192, \quad \Delta f = 20~\text{kHz},
\]
then
\[
    F_s = 8192 \times 20 \times 10^3 = 163.84 \times 10^6~\text{Hz} = \textbf{163.84~MHz}.
\]

Therefore, the sampling rate at the IFFT output is \textbf{163.84~MHz}.


\vspace{1cm}
\textbf{Question: What is the data rate (in Mbits/s) of this system, when all active subcarriers are used to carry data and a QPSK constellation is used?}
\vspace{0.5cm}

COMPLETAR

\vspace{1cm}
\textbf{Question:  Now assume that we decide to implement an IFFT with size $4 096$, and redo the previous two points.}
\vspace{0.5cm}
In an OFDM system, the sampling rate $F_s$ (at the IFFT output) is related to the IFFT size $N$ and the subcarrier spacing $\Delta f$ by:
\[
    F_s = N \, \Delta f.
\]

Given:
\[
    N = 4096, \quad \Delta f = 20~\text{kHz},
\]
then
\[
    F_s = 4096 \times 20 \times 10^3 = 81,92 \times 10^6~\text{Hz} = \textbf{81,92~MHz}.
\]

Therefore, the sampling rate at the IFFT output is \textbf{81,92~MHz}.

COMPLETAR

\vspace{0.5cm}
\section{Task 3}
\textbf{Question: Find a ``state evolution'' and ``measurement'' description for (1) in the form of (4).
    Note that the state of the system (the fluid level) is static, i.e., $s_n = s$ is constant with $n$.}
\vspace{0.5cm}

Given that the fluid level is static ($s_n = s_{n-1} = l$) and a scalar, the state evolution equation simplifies.
We identify the state transition matrix $A_n$ and the process noise $u_n$:
\begin{equation}
    s_n = 1 \cdot s_{n-1} + 0
\end{equation}
Thus, identifying terms with the general form $s_n = A_n s_{n-1} + G_n f_n + u_n$:
\begin{itemize}
    \item State variable: $s_n = l$ (scalar).
    \item State transition matrix: $A_n = 1$.
    \item Control input: $G_n = 0$ (no external forcing).
    \item Process noise: $u_n = 0$, which implies the process noise covariance is $Q_n = 0$.
\end{itemize}
For the measurement equation, we relate the pressure observation $x_n$ to the fluid level using the hydrostatic constant $c_h$:
\begin{equation}
    x_n = c_h \cdot s_n + v_n
\end{equation}
Comparing this to $x_n = H_n s_n + w_n$:
\begin{itemize}
    \item Observation matrix: $H_n = c_h$.
    \item Measurement noise: $w_n = v_n$.
    \item Measurement noise covariance: $R_n = \sigma_v^2$ (given variance of $v_k$).
\end{itemize}
%% Does this expression look familiar?
\vspace{0.5cm}
\section{Task 4}
\textbf{Question: Assume a full-scale sinusoidal input with $f_0 = 37.1094 MHz$, and let the FFT size be M = 1024.
    Generate $15 \cdot M$ samples of $x(t)$ (at fs = 100 MHz) and quantize them to N = 12 bits. Break
    the vector xq of quantized samples into 15 size-M blocks using, e.g., the command reshape:}

\vspace{0.5cm}

\begin{lstlisting}[language=Matlab]
    xqblocks = reshape(xq, M, 15);
\end{lstlisting}

\textbf{so that each column of the $M \times 15$ matrix xqblocks will contain the corresponding block of size
    M . Now, since the fft command computes the FFT columnwise, in order to apply an M -point
    FFT to each block, we simply make
}
\begin{lstlisting}[language=Matlab]
    X = fft(xqblocks, M);
\end{lstlisting}

\textbf{Average the squared magnitude of the DFT coefficients over the 15 blocks and plot the results
    between 0 and fs/2, in dBFS.
    Observe the location and peak value of the principal frequency component, as well as the value
    of the noise floor. Do your observations agree (quantitatively) with what you would expect?
}
\vspace{0.5cm}

\subsection{Theoretical values}
First, we need to calculate the expected theoretical values for the signal peak and for the noise floor value.
\subsubsection{Signal Peak}
We have $f_0 = 37.1094 MHz$ and $f_s = 100 MHz$. As $f_0 < f_s/2$ we don't have aliasing.
Terefore, we expect a signal peak at $f_0$, with a value of 0 DBFS, as it is a full-scale signal.

\subsubsection{Noise Floor}

To calculate the theoretical SQNR we have the formula
$SQNR = 6.02N +4.77 -20log_{10}(FS/\sigma_x)$
As we have a full scale sinusoid we have $\sigma_x = A/\sqrt{2} = FS/\sqrt{2}$
So, for $N=12$ and $\sigma_x = FS/\sqrt{2}$ we have $SQNR = 73.99 DBFS$

We have to calculate the processing gain, with the formula $10log_{10}(M/2)$.
For $M=1024$, we have a gain of 27.09 DBFS.

The noise floor will be $-(73.99+27.09)=101.08 DBFS$

\subsection{Matlab execution}

Executing the task\_4\_1.m matlab script we can see the next figure.

\begin{figure}[H]
    \centering
    \includegraphics[width=1\textwidth]{img/task4_1.png}
    \label{fig:task4_1}
\end{figure}

On the figure we can see a signal peak at 37.1094 MHz, whith a value of 0 DBFS.

This agrees quantitatively with the theory, which predicts a peak at the input frequency.

$f_0 = 37.1094 MHz$ and a level of 0 dBFS due to the normalization used for a full-scale signal.

We can also see that the noise floor is around the 100 DBFS, which agrees with the theoretical value.
\vspace{1cm}

\textbf{Question: Repeat the previous steps for an FFT size M = 256.
}
\subsection{Theoretical values}
As the frequency $f_0$ is the same, we would also have a signal peak on that point.
Since is full-scale too, thew value of the peak would also be 0 DBFS.

The SQNR will be the same, because we have the same number of bits and the same $\sigma_x$.

However, the gain will change, as we have a different value for M.
$Gain = 10log_{10}(M/2) = 10log_{10}(256/2) = 21.07$

For M = 256 we will have a nois floor of 95.06 DBFS

\subsection{Matlab execution}
Executing the task\_4\_2.m script, we can see a noise peak of 0 DBFS at $f_0$ and a level of noise floor
of approximately 95 DBFS
\begin{figure}[H]
    \centering
    \includegraphics[width=1\textwidth]{img/task4_2.png}
    \label{fig:task4_2}
\end{figure}
\vspace{1cm}

\textbf{Question: Set again M = 1024, and repeat the analysis for decreasing resolutions of 10, 8 and 6 bits.}
\subsection{Theoretical values}

As the frequency $f_0$ is the same, we would also have a signal peak on that point.
Since is full-scale too, thew value of the peak would also be 0 DBFS.

The Gain will be the same as on task4\_1 because we have thje same M.

The SQNR will change, because we have different values for N:
\begin{itemize}
    \item N = 10: $SQNR = 6.02N +4.77 -20log_{10}(FS/\sigma_x) = 6.02 * 10 +4.77 -20log_{10}(\sqrt{2}) = 61.95 DBFS$
    \item N = 8: $SQNR = 6.02N +4.77 -20log_{10}(FS/\sigma_x) = 6.02 * 8 +4.77 -20log_{10}(\sqrt{2}) = 49.91 DBFS$
    \item N = 6: $SQNR = 6.02N +4.77 -20log_{10}(FS/\sigma_x) = 6.02 * 6 +4.77 -20log_{10}(\sqrt{2}) = 37.87 DBFS$
\end{itemize}

So the noise floor for each value of N will be:
\begin{itemize}
    \item N = 10: $-(61.95 + 27.09) = -89.04 DBFS $
    \item N = 8: $-(49.91 + 27.09) = -77 DBFS$
    \item N = 6: $-(37.87 + 27.09) = -64.96 DBFS$
\end{itemize}

\subsection{Matlab execution}

Executing the the task\_4\_3.m script, we can see three figures with a peak of 0 DBFS on $f_0$.
We also see a noise floor value of :
\begin{itemize}
    \item N = 10: $-89.05 DBFS $
    \item N = 8: $-77.01 DBFS$
    \item N = 6: $-64.97 DBFS$
\end{itemize}

\begin{figure}[H]
    \begin{subfigure}[t]{.5\textwidth}
        \centering
        \includegraphics[width=\linewidth]{img/task4_3_n10.png}
        \caption{N=10 bits}
    \end{subfigure}
    \begin{subfigure}[t]{.5\textwidth}
        \centering
        \includegraphics[width=\linewidth]{img/task4_3_n8.png}
        \caption{N=8 bits}
    \end{subfigure}
    \begin{subfigure}[t]{.5\textwidth}
        \centering
        \includegraphics[width=\linewidth]{img/task4_3_n6.png}
        \caption{N=6 bits}
    \end{subfigure}
\end{figure}
\vspace{1cm}

\textbf{Question: Consider again M = 1024 and N = 12 bits. Repeat the analysis reducing the amplitude of
    the sinusoid to 1/3 of the full scale value, and compare your observations with the theoretical
    prediction.
}
\subsection{Theoretical values}
For a siusoid of amplitude $A =\alpha * FS , \alpha < 1$, we have $SQNR = 6.02N + 1,76 + 20log{10}(A)$.

For $A=1/3$ we have a SQNR of 64.81 DBFS. The gain will be the same, so we will have a noise floor of 91.9 DBFS


\vspace{1cm}

\textbf{Question: Let M = 1024, N = 12 bits and a full-scale sinusoid. Slightly change the frequency of the
    sinusoid to 37.12 MHz and repeat the analysis. How do your observations change? Does it make
    any difference if you use a larger number of samples, say 100 * M ? What happens if you increase
    the resolution to 16 bits?
    How do you explain all these?
}

The $f_0$ we had, was a $k$ of the FFT, so all the energy was on a single $k$: $ k =\frac{f_0*M}{fs} = \frac{37.1094*10^6*1024}{100*10^6}= 380$
When using the new frequency, the energy is between $k = 380$ and $k = 381$, so we see this power leak.

\begin{figure}[H]
    \centering
    \includegraphics[width=1\textwidth]{img/task4_5.png}
    \label{fig:task4_5}
\end{figure}

Increasing the number of samples or the resolution will not have effect, because we need to have the signal on a $k$.
To do this, we can change either the $f_0$, the $fs$ or M.

\vspace{0.5cm}
\section{Task 5}
\textbf{Question: Plot $g_\gamma(x)$ vs. $x$ in the range $x\in [-{\rm FS},{\rm FS}]$ for $\gamma = 0$, $1$ and $2$. For input signals whose values are always much smaller than $\rm FS$ (in absolute value), what will be the effect of the nonlinearity?
}
\vspace{0.5cm}

xd

\vspace{1cm}
\textbf{Question: Modify the code in {\tt quanti.m} and write a Matlab function {\tt dquanti.m} implementing this nonuniform quantizer. The format should be similar to that of {\tt quanti.m}, but including an additional input parameter {\tt gama}:}
\begin{center}
{\tt xq = dquanti( x, FS, Nbits, gama ); }
\end{center} 
\vspace{0.5cm}

xd

\vspace{1cm}
\textbf{Question: Generate  samples (at 100 MHz) of a full-scale sinusoid with $f_0 = 6.8359$ MHz.
Quantize them to $N=11$ bits using $\gamma = 0.003$ in {\tt dquanti}. 
Determine the SFDR in dBFS using an FFT size $M=2048$, and then with $M=512$. 
Does the SFDR depend on the FFT size? Does the noise floor depend on the FFT size? How do you explain this?
}
\vspace{0.5cm}

xd

\vspace{1cm}
\textbf{Question: Using $M=2048$, repeat the previous step for $\gamma = 0.01$ and $0.1$. Are the spectral spurs located where you would expect?
}
\vspace{0.5cm}

xd

\vspace{1cm}
\textbf{Question: Set now the amplitude to $\frac{\rm FS}{3}$. Using $M=2048$, measure the SFDR and express it in both dBFS and dBc for $\gamma=0.005$, $0.05$ and $0.1$. Will these values change if you repeat the analysis with $M=512$?
}
\vspace{0.5cm}

xd

\vspace{1cm}
\textbf{Question: Consider now samples (at 100 MHz and with 11-bit resolution) of a sinusoid with frequency $3.3202$ MHz and amplitude $\frac{\rm FS}{2}$. Obtain the THD for this nonuniform ADC with $\gamma = 0.3$ under the IEEE 1241-2000 specification, expressed in both dB and percentage.
}
\vspace{0.5cm}

xd



\vspace{0.5cm}
\section{Task 6}
We decide to invest in an additional pressure sensor, which we install it at the opposite side of the bottom of the tank. In that way, the measurement errors at the two sensors can be assumed independent of each other. However, the original sensor is out of stock, so we are forced to purchase a different model whose errors need not have the same variance as the first sensor.

%%%%%%%%%%%%%%%%%%%%%%%%%%%%%%%%%%%%%%%%%%%%%%%%%%%%%%%%%%%%%%%%%%%%%%%%%%%%%%%%%%%%%%%%%%%%%%%%%%%%%
%%%%%%%%%%%%%%%%%%%%%%%%%%%%%%%%%%%%%%%%%%%%%%%%%%%%%%%%%%%%%%%%%%%%%%%%%%%%%%%%%%%%%%%%%%%%%%%%%%%%%
\question{Question: Note that the addition of a new sensor does not alter the ''state evolution'' description of the system, but it obviously changes its ``measurement'' description. Give the corresponding details.}
\vspace{0.5cm}

\begin{itemize}
    \item \textbf{State vector (unchanged):}
    
    The state remains two-dimensional, containing both level and flow:
    $$\mathbf{s}_n = \begin{bmatrix} l_n \\ q_n \end{bmatrix}$$
    
    \item \textbf{State transition matrix (unchanged):}
    
    The dynamics remain identical to Task 5:
    $$\mathbf{A} = \begin{bmatrix} 1 & \frac{T}{A_{\text{tank}}} \\ 0 & 1 \end{bmatrix}$$
    
    where $T = 5$ s and $A_{\text{tank}} = \pi (11)^2 \approx 380.13$ cm$^2$.
    
    \item \textbf{Measurement matrix (modified):}
    
    In Task 5 we had one sensor measuring pressure:
    $$x_n = c_h \cdot l_n + w_n \quad \Rightarrow \quad \mathbf{H} = \begin{bmatrix} c_h & 0 \end{bmatrix}$$
    
    Now with \textbf{two sensors}, both measuring pressure (proportional to level), the measurement becomes a vector:
    $$\mathbf{x}_n = \begin{bmatrix} x_n^{(1)} \\ x_n^{(2)} \end{bmatrix} = \begin{bmatrix} c_h & 0 \\ c_h & 0 \end{bmatrix} \begin{bmatrix} l_n \\ q_n \end{bmatrix} + \begin{bmatrix} w_n^{(1)} \\ w_n^{(2)} \end{bmatrix}$$
    
    Therefore:
    $$\boxed{\mathbf{H} = \begin{bmatrix} c_h & 0 \\ c_h & 0 \end{bmatrix}}$$
    
    Both sensors measure the same physical quantity (level via pressure), but with different noise levels.
    
    \item \textbf{Measurement noise covariance (modified):}
    
    The two sensors have independent errors with different variances:
    $$\mathbf{R} = \begin{bmatrix} \sigma_{v,1}^2 & 0 \\ 0 & \sigma_{v,2}^2 \end{bmatrix} = \begin{bmatrix} 20^2 & 0 \\ 0 & 80^2 \end{bmatrix} \text{ mbar}^2$$
    
    The off-diagonal terms are zero because the sensor errors are independent.
\end{itemize}

\textbf{Summary:} The state evolution remains unchanged. The only modification is expanding the measurement equation from a scalar to a 2D vector, with each row of $\mathbf{H}$ corresponding to one sensor.

%%%%%%%%%%%%%%%%%%%%%%%%%%%%%%%%%%%%%%%%%%%%%%%%%%%%%%%%%%%%%%%%%%%%%%%%%%%%%%%%%%%%%%%%%%%%%%%%%%%%%
%%%%%%%%%%%%%%%%%%%%%%%%%%%%%%%%%%%%%%%%%%%%%%%%%%%%%%%%%%%%%%%%%%%%%%%%%%%%%%%%%%%%%%%%%%%%%%%%%%%%%
\question{Question: Write a Matlab script {\tt kalman\_flow2.m} to simulate this dynamical system and the corresponding Kalman filter to estimate the fluid (benzene) level and the fluid flow. Assume that the two sensors work correctly (i.e., $p=0$ in \eqref{eq:xno}), measuring pressure in mbar, taking one measurement every 5 seconds, and that the cross-section of the tank is circular with diameter 22 cm. Fluid level is to be expressed in cm, as before, whereas fluid flow is to be expressed in cm$^3/$s.}
\question{
          The initial true fluid level is 340 cm; our guess is 250 cm, with a standard deviation of 11 cm. The true fluid flow is constant with time and equal to 33 cm$^3$/s; our guess is 0 cm$^3/$s, with standard deviation of 10 cm$^3/$s. Measurement errors are zero-mean uncorrelated Gaussian, with standard deviation of 20 mbar at the first sensor and of 80 mbar at the second.}

The script \texttt{kalman\_flow.m~\ref{app:kalman_flow2}} implements the Kalman filter for the described two-sensor dynamic system.

%%%%%%%%%%%%%%%%%%%%%%%%%%%%%%%%%%%%%%%%%%%%%%%%%%%%%%%%%%%%%%%%%%%%%%%%%%%%%%%%%%%%%%%%%%%%%%%%%%%%%
%%%%%%%%%%%%%%%%%%%%%%%%%%%%%%%%%%%%%%%%%%%%%%%%%%%%%%%%%%%%%%%%%%%%%%%%%%%%%%%%%%%%%%%%%%%%%%%%%%%%%
\question{Question: Plot the time evolution of the fluid level and fluid flow, the measurements, and the estimates, for up to $60$ minutes, and for two different executions. Comment on your results.}
\vspace{0.5cm}

We executed the simulation for 60 minutes with two independent executions. The results are shown below.

\subsection*{System Parameters}
\begin{itemize}
    \item \textbf{Tank:} Diameter 22 cm, sampling period $T=5$ s.
    \item \textbf{True initial state:} Level 340 cm, flow 33 cm$^3$/s (constant).
    \item \textbf{Initial guesses:} Level 250 cm ($\sigma_l=11$ cm), flow 0 cm$^3$/s ($\sigma_q=10$ cm$^3$/s).
    \item \textbf{Sensor noise:} Sensor 1: $\sigma_{v,1}=20$ mbar, Sensor 2: $\sigma_{v,2}=80$ mbar.
\end{itemize}

\subsection*{Simulation Results}

\begin{figure}[H]
    \centering
    \includegraphics[width=0.85\textwidth]{img/task6_level.png}
    \caption{Time evolution of fluid level with two sensors. Gray asterisks: Sensor 1 measurements (low noise). Green asterisks: Sensor 2 measurements (high noise). Blue/red lines: Kalman filter estimates. Black dashed: true level.}
    \label{fig:task6_level}
\end{figure}

\begin{figure}[H]
    \centering
    \includegraphics[width=0.85\textwidth]{img/task6_flow.png}
    \caption{Time evolution of fluid flow with two sensors. The filter successfully estimates the hidden flow variable, converging from 0 cm$^3$/s to the true value of 33 cm$^3$/s.}
    \label{fig:task6_flow}
\end{figure}

\subsection*{Discussion}

\textbf{Fluid Level (Figure~\ref{fig:task6_level}):}
\begin{enumerate}
    \item \textbf{Sensor Quality Difference:} Sensor 1 (gray points) produces measurements with low scatter, clustering near the true trajectory. Sensor 2 (green points) has much higher noise, with measurements spread widely (up to $\pm 90$ cm equivalent deviation).
    
    \item \textbf{Filter Robustness:} Despite the noisy second sensor, the Kalman filter estimates (blue/red lines) track the true level accurately. The filter automatically gives more weight to Sensor 1 (which has higher precision) while still extracting useful information from Sensor 2.
    
    \item \textbf{Level Evolution:} The true level increases linearly at approximately $\frac{33}{380.13} \approx 0.087$ cm/s (5.2 cm/min) due to the constant positive flow.
\end{enumerate}

\textbf{Fluid Flow (Figure~\ref{fig:task6_flow}):}
\begin{enumerate}
    \item \textbf{Convergence:} Starting from an initial guess of 0 cm$^3$/s, the filter estimates converge to the true value of 33 cm$^3$/s within approximately 10-15 minutes.
    
    \item \textbf{Indirect Observation:} The flow is not directly measured. The filter infers it by observing how fast the level is changing over time, combining information from both sensors.
    
    \item \textbf{Stability:} After convergence, the estimates remain stable near 33 cm$^3$/s, demonstrating successful tracking of the hidden state.
\end{enumerate}

%%%%%%%%%%%%%%%%%%%%%%%%%%%%%%%%%%%%%%%%%%%%%%%%%%%%%%%%%%%%%%%%%%%%%%%%%%%%%%%%%%%%%%%%%%%%%%%%%%%%%
%%%%%%%%%%%%%%%%%%%%%%%%%%%%%%%%%%%%%%%%%%%%%%%%%%%%%%%%%%%%%%%%%%%%%%%%%%%%%%%%%%%%%%%%%%%%%%%%%%%%%
\question{Question: Plot the time evolution of the standard deviation of the estimation error for both the fluid level (in cm) and the fluid flow (in cm$^3/$s), also for two different realizations. Superimpose the corresponding curves when only the first sensor is available, and explain what you see.}
\vspace{0.5cm}

Figure~\ref{fig:task6_comparison_std} compares the evolution of estimation uncertainty between the two-sensor configuration (Task 6) and the single-sensor case (Task 5).

\begin{figure}[!htbp]
    \centering
    \includegraphics[width=0.9\textwidth]{img/task6_comparison_std.png}
    \caption{Estimation uncertainty: level (left) and flow (right). 
    Blue (Task 6, 2 sensors) vs red (Task 5, 1 sensor). Log scale.}
    \label{fig:task6_comparison_std}
\end{figure}

\subsection*{Comments}

The two-sensor configuration reduces both level and flow uncertainty compared to the single-sensor case. Level uncertainty decreases faster with two sensors. Flow uncertainty shows a smaller but consistent improvement, 
since flow estimation depends indirectly on observing level changes. The Kalman filter automatically weights each sensor by its reliability, so the poor-quality sensor still helps without degrading performance.


\vspace{0.5cm}
\section{Task 7}
To conclude the assignment, we study the case in which the fluid flow does not remain constant with time; instead, it exhibits random fluctuations, according to
\begin{eqnarray}\label{eq:rndflow}
 \mbox{[fluid flow at time $nT$]} &=& \mbox{[fluid flow at time $(n-1)T$]} \, + \, \mbox{[random increment]},
\end{eqnarray}
where the random increment is modeled as a zero-mean Gaussian random variable with standard deviation of 0.35 cm$^3/$s.

%%%%%%%%%%%%%%%%%%%%%%%%%%%%%%%%%%%%%%%%%%%%%%%%%%%%%%%%%%%%%%%%%%%%%%%%%%%%%%%%%%%%%%%%%%%%%%%%%%%%%
%%%%%%%%%%%%%%%%%%%%%%%%%%%%%%%%%%%%%%%%%%%%%%%%%%%%%%%%%%%%%%%%%%%%%%%%%%%%%%%%%%%%%%%%%%%%%%%%%%%%%
\textbf{Question: Modify your script from Task 6 to incorporate \eqref{eq:rndflow}. Execute your Kalman filter simulation to cover up to $60$ minutes, plot the corresponding curves, and comment on your observations.}
\vspace{0.5cm}

COMPLETAR

%%%%%%%%%%%%%%%%%%%%%%%%%%%%%%%%%%%%%%%%%%%%%%%%%%%%%%%%%%%%%%%%%%%%%%%%%%%%%%%%%%%%%%%%%%%%%%%%%%%%%
%%%%%%%%%%%%%%%%%%%%%%%%%%%%%%%%%%%%%%%%%%%%%%%%%%%%%%%%%%%%%%%%%%%%%%%%%%%%%%%%%%%%%%%%%%%%%%%%%%%%%
\textbf{Question: What are the asymptotic values of the standard deviation of the estimation error of: (i)  the fluid level (in cm); and (ii) the fluid flow (in cm$^3/$s)?}
\vspace{0.5cm}

COMPLETAR


% ==============================================================================
% APÉNDICE B: Desarrollo Manual (Task 1 y 2)
% ==============================================================================
\section{Hand-Written Solutions (Tasks 1 \& 2)}
\label{app:handwritten}

This appendix contains the hand-written derivations and solutions for Tasks 1 and 2.

\subsection{Task 1: Least Squares Estimation}

\subsubsection{Question 1: LS Estimator Expression}

\begin{figure}[H]
    \centering
    \includegraphics[width=0.9\textwidth]{img/task1_q1.png}
    \caption{Derivation of the LS estimator using convex optimization.}
    \label{fig:task1_q1}
\end{figure}

\subsubsection{Question 2: Interpretation}

\begin{figure}[H]
    \centering
    \includegraphics[width=0.9\textwidth]{img/task1_q2.png}
    \caption{Interpretation of the LS estimator: weighted average of measurements and prior knowledge.}
    \label{fig:task1_q2}
\end{figure}

\subsubsection{Question 3}

\begin{figure}[H]
    \centering
    \includegraphics[width=0.9\textwidth]{img/task1_q3.png}
    \caption{Additional derivations for Task 1, Question 3.}
    \label{fig:task1_q3}
\end{figure}

\subsubsection{Question 4}

\begin{figure}[H]
    \centering
    \includegraphics[width=0.9\textwidth]{img/task1_q4.png}
    \caption{Task 1, Question 4: Further analysis.}
    \label{fig:task1_q4}
\end{figure}

\subsubsection{Question 5: Normalized MSE}

\begin{figure}[H]
    \centering
    \includegraphics[width=0.9\textwidth]{img/task1_q5.png}
    \caption{Derivation of normalized Mean Squared Error.}
    \label{fig:task1_q5}
\end{figure}

\subsubsection{Question 6: Recursive Form}

\begin{figure}[H]
    \centering
    \includegraphics[width=0.9\textwidth]{img/task1_q6.png}
    \caption{Recursive formulation of the LS estimator.}
    \label{fig:task1_q6}
\end{figure}

\subsubsection{Questions 7-9}

\begin{figure}[H]
    \centering
    \includegraphics[width=0.9\textwidth]{img/task1_q7-9.png}
    \caption{Task 1, Questions 7, 8, and 9: Evaluation and asymptotic behavior.}
    \label{fig:task1_q7-9}
\end{figure}


\subsection{Task 2: LMMSE and Recursive Estimation}

\subsubsection{Question 1: LMMSE Parameters}

\begin{figure}[H]
    \centering
    \includegraphics[width=0.9\textwidth]{img/task2_q1.png}
    \caption{Calculation of LMMSE parameters: $\bm{\mu}_x$, $\bm{C}_{\ell x}$, and $\bm{C}_{xx}$.}
    \label{fig:task2_q1}
\end{figure}

\subsubsection{Question 2: Matrix Inversion Lemma}

\begin{figure}[H]
    \centering
    \includegraphics[width=0.9\textwidth]{img/task2_q2.png}
    \caption{Application of Matrix Inversion Lemma to obtain $\bm{C}_{xx}^{-1}$ and recursive LMMSE expression.}
    \label{fig:task2_q2}
\end{figure}


\appendix
\section{Appendix: MATLAB scripts and data}
\label{app:matlab}

\subsection{Task 1}
\subsubsection{task1.m}

\begin{lstlisting}[language=Matlab]
x = linspace(-7,7,1000);
xq = quanti(x,7,2);
xq_4 = quanti(x,7,4);

figure;
plot(x,x,'b',x,xq,'r',x,xq_4,'y');
grid on

quantized_error = x - xq;
quantized_error_4b = x - xq_4;

figure;
plot(x,quantized_error, 'r', x, quantized_error_4b, 'y');
title('Quantization Error as a Function of Input Amplitude');
xlabel('Input Amplitude');
ylabel('Quantization Error');
grid on;
\end{lstlisting}

\subsection{Task 2}
\subsubsection{task2.m}

\begin{lstlisting}[language=Matlab]
FS = 5;
fs = 100e6;
f0 = 18.17e6;
M = 15 * 2^10;
t = (0:M-1)/fs;

A_list = [0.5, 0.75, 1.0, 1.03] * FS;
Nbits_list = [12, 10, 8, 6, 4];

results = [];

for A = A_list
    x = A * cos(2*pi*f0*t);
    sigma_x = A / sqrt(2);
    var_x_emp = var(x,1);
    for Nbits = Nbits_list
        LSB = FS / 2^(Nbits-1);
        xq = quanti(x, FS, Nbits);

        e = x - xq;
        var_e_emp = var(e,1);
        var_e_theor = LSB^2 / 12;

        SQNR_emp = 10*log10(var_x_emp / var_e_emp);
        SQNR_theor_formula = 6.02*Nbits + 4.77 - 20 * log10(FS / sigma_x);

        % Save
        results = [results; A, Nbits, var_e_emp, var_e_theor, SQNR_emp, SQNR_theor_formula];
    end
end
\end{lstlisting}


\subsection{Task 3}
\subsubsection{task3\_2.m}

\begin{lstlisting}[language=Matlab]

rng(0)

x0 = 1;
A = -x0; B = 0; C = +x0;

pd = makedist('Triangular','A',A,'B',B,'C',C);
N = 100000;
samples = random(pd, N, 1);

emp_mean = mean(samples);
emp_var = var(samples);
emp_desv_std = std(samples);

theo_var = (A^2 + B^2 + C^2 - A*B - A*C - B*C)/18;

rms = 20*log10(sqrt(emp_var)/x0);

fprintf('Theorical mean: 0; emp mean: %g\n',emp_mean);
fprintf('Theorical var: %.2f; emp var: %.2f\n',theo_var,emp_var);
fprintf('Sigma value: %.2f\n',sqrt(theo_var));
fprintf('rms value in dBFS: %g\n',rms)

xgrid = linspace(A,C,400)';
figure
histogram(samples,100,'Normalization','pdf', 'DisplayName', 'Generated samples')
hold on; grid on;

plot(xgrid, pdf(pd,xgrid), 'r-', 'LineWidth', 2, 'DisplayName', 'Theoretical PDF');

title('Symmetric Triangular Distribution - makedist');
xlabel('Value');
ylabel('PDF');
legend('Location','best');
hold off
\end{lstlisting}

\subsubsection{task3\_tri.m}
\begin{lstlisting}[language=Matlab]
rng(0);

x0 = 1;
sigma_x = x0 / sqrt(6);
N = 100000;

a = x0 / 2;

x1 = (2*rand(N,1) - 1) * a;
x2 = (2*rand(N,1) - 1) * a;
y = x1 + x2;

mean_y = mean(y);
var_y = var(y,1);
rms_dBFS = 20*log10(sqrt(var_y)/x0);

fprintf('--- Validation ---\n');
fprintf('Expected mean = 0\n');
fprintf('Sample mean   = %.5f\n\n', mean_y);

fprintf('Expected RMS [dBFS] = -7.78\n');
fprintf('Sample RMS   [dBFS] = %g\n', rms_dBFS);

figure;
histogram(y, 100, 'Normalization', 'pdf', 'DisplayName', 'Generated samples');
hold on; grid on;

x_pdf = linspace(-x0, x0, 400);
f_pdf = (x0 - abs(x_pdf)) / (x0^2);
plot(x_pdf, f_pdf, 'r-', 'LineWidth', 2, 'DisplayName', 'Theoretical PDF');

title('Symmetric Triangular Distribution - Sum 2 dist');
xlabel('Value');
ylabel('PDF');
legend('Location','best');
hold off;

\end{lstlisting}

\subsubsection{task3\_normal\_dist\_set.m}
\begin{lstlisting}[language=Matlab]
rng(0);

x0 = 1;
sigma_x = 10^(-9.54/20);
N = 100000;

x = sigma_x * randn(1, N);

emp_mean = mean(x);
emp_var  = var(x,1);
rms_dBFS = 20*log10(sqrt(emp_var)/x0);

fprintf('--- Validation ---\n');
fprintf('Expected mean = 0\n');
fprintf('Sample mean   = %g\n\n', emp_mean);

fprintf('Expected RMS [dBFS] = -9.54\n');
fprintf('Sample RMS   [dBFS] = %g\n', rms_dBFS);

figure;
histogram(x, 60, 'Normalization', 'pdf');
hold on; grid on;

xx = linspace(-4*sigma_x, 4*sigma_x, 400);
plot(xx, (1/(sigma_x*sqrt(2*pi))) * exp(-0.5*(xx/sigma_x).^2), 'r-', 'LineWidth',1.5);

title('Gaussian Distribution');
xlabel('Value');
ylabel('PDF');
legend('Empirical PDF','Theoretical');
grid on; hold off;
\end{lstlisting}

\subsection{Task 4}
\subsubsection{task4\_1.m}

\begin{lstlisting}[language=Matlab]
f0 = 37.1094e6;
M = 1024;
fs = 100e6;
Nbits = 12;
blocks = 15;
FS = 1;
                                                                                                                                                                                                                                                                                                                                                                                                                                                                                                                                                                                                                                                    
Nsamples = blocks * M;

%% Generar FS sinusoidal
n = 0:Nsamples-1;
xt = FS * cos(2*n*pi*f0/fs);
xq = quanti(xt, FS, Nbits);

xqblocks = reshape(xq, M, 15);

X = fft(xqblocks, M);

P_avg = mean(abs(X).^2, 2);

norm_const = (M / 2)^2;
P_dbfs = 10 * log10(P_avg / norm_const);

f_axis = (0:M-1) * fs / M; 
plot(f_axis(1:M/2 + 1) / 1e6, P_dbfs(1:M/2 + 1));
grid on;
xlabel('Frequency (MHz)');
ylabel('Power (dBFS)');
title('Espectro de Potencia Promedio (N=12 bits, M=1024)');
ylim([-140, 10]);
\end{lstlisting}


\subsubsection{task4\_2.m}
\begin{lstlisting}[language=Matlab]
f0 = 37.1094e6;
M = 256;
fs = 100e6;
Nbits = 12;
blocks = 15;
FS = 1;

Nsamples = blocks * M;

%% Generar FS sinusoidal
n = 0:Nsamples-1;
xt = FS * cos(2*n*pi*f0/fs);
xq = quanti(xt, FS, Nbits);

xqblocks = reshape(xq, M, 15);

X = fft(xqblocks, M);

P_avg = mean(abs(X).^2, 2);

norm_const = (M / 2)^2;
P_dbfs = 10 * log10(P_avg / norm_const);

f_axis = (0:M-1) * fs / M;
plot(f_axis(1:M/2 + 1) / 1e6, P_dbfs(1:M/2 + 1));
grid on;
xlabel('Frequency (MHz)');
ylabel('Power (dBFS)');
title('Espectro de Potencia Promedio (N=12 bits, M=256)');
ylim([-140, 10]);
\end{lstlisting}


\subsubsection{task4\_3.m}
\begin{lstlisting}[language=Matlab]
f0 = 37.1094e6;
M = 1024;
fs = 100e6;
blocks = 15;
FS = 1;
Nsamples = blocks * M;
f_axis = (0:M-1) * fs / M;
norm_const = (M / 2)^2; 

n = 0:Nsamples-1;
xt = FS * cos(2*n*pi*f0/fs);

Nbits_list = [10, 8, 6];

for i = 1:length(Nbits_list)
    Nbits = Nbits_list(i);
    
    xq = quanti(xt, FS, Nbits);
    
    xqblocks = reshape(xq, M, blocks);
    X = fft(xqblocks, M);
    
    P_avg = mean(abs(X).^2, 2);
    P_dbfs = 10 * log10(P_avg / norm_const);
    
    sqnr_teorico = 6.02 * Nbits + 1.76;
    piso_ruido_teorico = -sqnr_teorico - 10*log10(M/2);
    
    figure;
    
    plot(f_axis(1:M/2 + 1) / 1e6, P_dbfs(1:M/2 + 1), 'b');
    hold on;
    yline(piso_ruido_teorico, 'r--', 'LineWidth', 1.5);
    hold off;
    
    grid on;
    xlabel('Frequency (MHz)');
    ylabel('Power (dBFS)');
    title(sprintf('Espectro (N = %d bits, M = 1024)', Nbits));
    ylim([-140, 10]);
    legend('Espectro medido', ...
           sprintf('Noise Floor (%.2f dBFS)', piso_ruido_teorico));
end
\end{lstlisting}


\subsubsection{task4\_4.m}
\begin{lstlisting}[language=Matlab]
f0 = 37.1094e6;
M = 1024;
fs = 100e6;
Nbits = 12;
blocks = 15;
FS = 1;

Nsamples = blocks * M;

n = 0:Nsamples-1;
xt = (1/3)*FS * cos(2*n*pi*f0/fs);
xq = quanti(xt, FS, Nbits);

xqblocks = reshape(xq, M, 15);

X = fft(xqblocks, M);

P_avg = mean(abs(X).^2, 2);

norm_const = (M / 2)^2;
P_dbfs = 10 * log10(P_avg / norm_const);

f_axis = (0:M-1) * fs / M; 
plot(f_axis(1:M/2 + 1) / 1e6, P_dbfs(1:M/2 + 1));
grid on;
xlabel('Frequency (MHz)');
ylabel('Power (dBFS)');
title('Espectro de Potencia Promedio (N=12 bits, M=1024)');
ylim([-140, 10]);
\end{lstlisting}

\subsubsection{task4\_5.m}
\begin{lstlisting}[language=Matlab]

f0 = 37.12e6;
M = 1024;
fs = 100e6;
Nbits = 16;
%blocks = 15;
blocks = 100;
FS = 1;

Nsamples = blocks * M;

n = 0:Nsamples-1;
xt = FS * cos(2*n*pi*f0/fs);
xq = quanti(xt, FS, Nbits);

xqblocks = reshape(xq, M, blocks);

X = fft(xqblocks, M);

P_avg = mean(abs(X).^2, 2);

norm_const = (M / 2)^2;
P_dbfs = 10 * log10(P_avg / norm_const);

f_axis = (0:M-1) * fs / M;
plot(f_axis(1:M/2 + 1) / 1e6, P_dbfs(1:M/2 + 1));
grid on;
xlabel('Frequency (MHz)');
ylabel('Power (dBFS)');
title('Espectro de Potencia Promedio (N=16 bits, M=1024)');
ylim([-140, 10]);
\end{lstlisting}

\subsection{Task 5}
\subsubsection{task5\_1.m}
\begin{lstlisting}[language=Matlab]
FS = 1;
x = linspace(-FS, FS, 1000);

g_0 = x;

gama_1 = 1;
g_1 = sign(x) .* (FS / log(1 + gama_1)) .* log(1 + gama_1 .* abs(x) / FS);
g_1(x == 0) = 0;

gama_2 = 2;
g_2 = sign(x) .* (FS / log(1 + gama_2)) .* log(1 + gama_2 .* abs(x) / FS);
g_2(x == 0) = 0;

figure;
plot(x, g_0, 'b', 'LineWidth', 2);
hold on;
plot(x, g_1, 'r', 'LineWidth', 2);
plot(x, g_2, 'g', 'LineWidth', 2);
grid on;
xlabel('x');
ylabel('g(x)');
title('Distortion Function g_\gamma(x)');
legend('\gamma = 0 (Ideal)', '\gamma = 1', '\gamma = 2');
\end{lstlisting}

\subsubsection{task5\_3.m}
\begin{lstlisting}[language=Matlab]

\end{lstlisting}


\subsubsection{task5\_4.m}
\begin{lstlisting}[language=Matlab]
f0 = 6.8359e6;
fs = 100e6;
FS = 1;
gamma_list = [0.01,0.1];
Nbits = 11;
M = 2048;
blocks = 15;

for gamma = gamma_list

    Nsamples = blocks * M;
    n = (0:Nsamples-1).';
    xt = FS * cos(2*pi*f0/fs * n);

    xq = dquanti(xt, FS, Nbits, gamma);
    xqblocks = reshape(xq, M, blocks);

    X = fft(xqblocks, M);

    P_avg = mean(abs(X).^2, 2);

    k0 = round(f0 * M / fs);
    n0 = (0:M-1).';
    xref = FS * cos(2*pi*(k0/M) * n0);
    Pref = max(abs(fft(xref, M)).^2);

    half = 1:(M/2);
    freqs = (half-1) * (fs / M);
    P_half = P_avg(half);

    P_dbfs = 10*log10( P_half / Pref );

    figure('Name',sprintf('M=%d, y=%.2f',M,gamma));
    plot(freqs/1e6, P_dbfs, 'LineWidth', 1.2);
    title(sprintf('PSD averaged, M=%d, N=%d, \\gamma=%.4g', M, Nbits, gamma));
    legend('PSD (avg)');
    grid on;
end

\end{lstlisting}

\subsubsection{task5\_5.m}
\begin{lstlisting}[language=Matlab]
f0 = 6.8359e6;
fs = 100e6;
FS = 1;
gamma_list = [0.005,0.05,0.1];
Nbits = 11;
M_list = [2048, 512];
blocks = 15;

for M = M_list
    for gamma = gamma_list
        Nsamples = blocks * M;
        n = (0:Nsamples-1).';
        xt = (FS/3) * cos(2*pi*f0/fs * n);
    
        xq = dquanti(xt, FS, Nbits, gamma);
        xqblocks = reshape(xq, M, blocks);
    
        X = fft(xqblocks, M);
    
        P_avg = mean(abs(X).^2, 2);
    
        k0 = round(f0 * M / fs);
        n0 = (0:M-1).';
        xref = FS * cos(2*pi*(k0/M) * n0);
        Pref = max(abs(fft(xref, M)).^2);
    
        half = 1:(M/2);
        freqs = (half-1) * (fs / M);
        P_half = P_avg(half);
    
        P_dbfs = 10*log10( P_half / Pref );
    
        figure('Name',sprintf('M=%d, y=%.2f',M,gamma));
        plot(freqs/1e6, P_dbfs, 'LineWidth', 1.2);
        title(sprintf('PSD averaged, M=%d, N=%d, \\gamma=%.4g', M, Nbits, gamma));
        legend('PSD (avg)');
        grid on;
    end
end
\end{lstlisting}

\subsubsection{task5\_6.m}
\begin{lstlisting}[language=Matlab]
f0 = 3.3202e6;
fs = 100e6;
FS = 1;
gamma = 0.3;
Nbits = 11;
M = 2048;
blocks = 15;

Nsamples = blocks * M;
n = (0:Nsamples-1).';
xt = (FS/2) * cos(2*pi*f0/fs * n);

xq = dquanti(xt, FS, Nbits, gamma);
xqblocks = reshape(xq, M, blocks);

X = fft(xqblocks, M);

P_avg = mean(abs(X).^2, 2);

k0 = round(f0 * M / fs);
n0 = (0:M-1).';
xref = FS * cos(2*pi*(k0/M) * n0);
Pref = max(abs(fft(xref, M)).^2);

half = 1:(M/2);
freqs = (half-1) * (fs / M);
P_half = P_avg(half);

P_dbfs = 10*log10( P_half / Pref );

figure('Name',sprintf('M=%d, y=%.2f',M,gamma));
plot(freqs/1e6, P_dbfs, 'LineWidth', 1.2);
title(sprintf('PSD averaged, M=%d, N=%d, \\gamma=%.4g', M, Nbits, gamma));
legend('PSD (avg)');
grid on;

\end{lstlisting}

\subsection{Task 6}
\subsubsection{task6\_3.m}
\begin{lstlisting}[language=Matlab]
fs = 100e6;
Nbits = 12;
FS = 1;
M = 1024;
blocks = 100;
Nsamples = M * blocks;

k0 = 410;
fc = k0 * fs / M;


sigma_list_ps = [10, 0.1];

f_axis = (0:M/2) * fs / M;

Pq_dBFS = -(6.02 * Nbits + 1.76);
Pq_linear = 10^(Pq_dBFS / 10);
FFT_gain_dB = 10 * log10(M / 2);

for sigma_ps = sigma_list_ps
    sigma_tau = sigma_ps * 1e-12;
    
    SNR_jitter_dB = 20 * log10(1 / (2 * pi * fc * sigma_tau));
    Pj_dBFS = -SNR_jitter_dB;
    Pj_linear = 10^(Pj_dBFS / 10);
    
    P_total_linear = Pq_linear + Pj_linear;
    P_total_dBFS = 10 * log10(P_total_linear);
    
    Expected_Floor_dBFS = P_total_dBFS - FFT_gain_dB;
    
    fprintf('--- Caso sigma = %.1f ps ---\n', sigma_ps);
    fprintf('  P_cuantizacion (Pq): %.2f dBFS\n', Pq_dBFS);
    fprintf('  P_jitter (Pj):       %.2f dBFS\n', Pj_dBFS);
    fprintf('  P_ruido_total:       %.2f dBFS\n', P_total_dBFS);
    fprintf('  Piso FFT Esperado:   %.2f dBFS\n', Expected_Floor_dBFS);
    
    if sigma_ps == 10
        floor_10ps = Expected_Floor_dBFS;
    else
        floor_0_1ps = Expected_Floor_dBFS;
    end
end


for sigma_ps = sigma_list_ps
    sigma_tau = sigma_ps * 1e-12;
    
    n = (0:Nsamples-1)';
    t_ideal = n / fs;
    
    a = sigma_tau * sqrt(3);
    tau_n = -a + (2 * a) * rand(Nsamples, 1);
    
    t_jittered = t_ideal + tau_n;
    
    xt = FS * cos(2 * pi * fc * t_jittered);
    
    xq = quanti(xt, FS, Nbits);
    
    xq_blocks = reshape(xq, M, blocks);
    
    X_fft = fft(xq_blocks, M);
    P_avg = mean(abs(X_fft).^2, 2);
    
    norm_const = (M / 2)^2;
    P_dbfs = 10 * log10(P_avg / norm_const);
    
    figure;
    plot(f_axis / 1e6, P_dbfs(1:M/2 + 1));
    hold on;
    
    if sigma_ps == 10
        yline(floor_10ps, 'r--', 'LineWidth', 2, ...
            'Label', sprintf('Piso Teorico (%.1f dBFS)', floor_10ps));
    else
        yline(floor_0_1ps, 'r--', 'LineWidth', 2, ...
            'Label', sprintf('Piso Teorico (%.1f dBFS)', floor_0_1ps));
    end
    
    grid on;
    title(sprintf('Efecto del Jitter (\\sigma_{\\tau} = %.1f ps), M=1024', sigma_ps));
    xlabel('Frecuencia (MHz)');
    ylabel('Potencia (dBFS)');
    ylim([-140, 10]);
    legend('Espectro Simulado', 'Piso Teorico Esperado');
end
\end{lstlisting}

\end{document}