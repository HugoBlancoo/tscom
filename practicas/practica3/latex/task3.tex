We are going to recast the estimation problem from the previous section in the framework of the Kalman filter, which is based in the following two equations ({\em state evolution} and {\em measurement}):
\begin{equation}\label{eq:statespace}
  \bm s_n = \bm A_n \bm s_{n-1} + \bm G_n \bm f_n + \bm u_n, \qquad \bm x_n = \bm H_n \bm s_n + \bm w_n.
\end{equation}
(The definitions of the different quantities should be familiar, and can be found in the class notes).
In our problem we have a single variable to estimate (the fluid level), so the state vector $\bm s_n$ is actually a scalar $s_n$. At time $n$ we obtain a single new pressure measurement, so the vector $\bm x_n$ also reduces to a scalar, $x_n$.

%%%%%%%%%%%%%%%%%%%%%%%%%%%%%%%%%%%%%%%%%%%%%%%%%%%%%%%%%%%%%%%%%%%%%%%%%%%%%%%%%%%%%%%%%%%%%%%%%%%%%
%%%%%%%%%%%%%%%%%%%%%%%%%%%%%%%%%%%%%%%%%%%%%%%%%%%%%%%%%%%%%%%%%%%%%%%%%%%%%%%%%%%%%%%%%%%%%%%%%%%%%
\vspace{0.5cm}
\question{Question: Find a ``state evolution'' and ``measurement'' description for \eqref{eq:simple_obs} in the form of \eqref{eq:statespace}.
  Note that the state of the system (the fluid level) is static, i.e., $s_n = s$ is constant with $n$.}
\vspace{0.5cm}

Given that the fluid level is static ($s_n = s_{n-1} = l$) and a scalar, the state evolution equation simplifies.
We identify the state transition matrix $A_n$ and the process noise $u_n$:
\begin{equation*}
  s_n = 1 \cdot s_{n-1} + 0
\end{equation*}
Thus, identifying terms with the general form $s_n = A_n s_{n-1} + G_n f_n + u_n$:
\begin{itemize}
  \item State variable: $s_n = l$ (scalar).
  \item State transition matrix: $A_n = 1$.
  \item Control input: $G_n = 0$ (no external forcing).
  \item Process noise: $u_n = 0$, which implies the process noise covariance is $Q_n = 0$.
\end{itemize}
For the measurement equation, we relate the pressure observation $x_n$ to the fluid level using the hydrostatic constant $c_h$:
\begin{equation*}
  x_n = c_h \cdot s_n + v_n
\end{equation*}
Comparing this to $x_n = H_n s_n + w_n$:
\begin{itemize}
  \item Observation matrix: $H_n = c_h$.
  \item Measurement noise: $w_n = v_n$.
  \item Measurement noise covariance: $R_n = \sigma_v^2$ (given variance of $v_k$).
\end{itemize}

%%%%%%%%%%%%%%%%%%%%%%%%%%%%%%%%%%%%%%%%%%%%%%%%%%%%%%%%%%%%%%%%%%%%%%%%%%%%%%%%%%%%%%%%%%%%%%%%%%%%%
%%%%%%%%%%%%%%%%%%%%%%%%%%%%%%%%%%%%%%%%%%%%%%%%%%%%%%%%%%%%%%%%%%%%%%%%%%%%%%%%%%%%%%%%%%%%%%%%%%%%%
\vspace{0.5cm}
\question{Question:Derive the Kalman filter equations for this problem, in order to obtain the LMMSE estimate of the state given the observations $x_0$,\ldots,$x_n$ ( i.e., $\hat s_{n|n}$).}
\vspace{0.5cm}

\subsubsection*{Derivation of the Kalman Filter Equations}

Since the system state is a scalar constant ($s_n = s_{n-1}$), we identified that $A_n = 1$, $G_n = 0$, $Q_n = 0$. The measurement follows $x_n = c_h s_n + v_n$, implying $H_n = c_h$ and $R_n = \sigma_v^2$. The Kalman filter recursive steps for $n \ge 0$ are derived as follows:

\begin{enumerate}
  \item \textbf{Prediction ($\hat{s}_{n|n-1}$):}
        Using the state evolution equation with $f_n=0$:
        \begin{equation*}
          \hat{s}_{n|n-1} = A_n \hat{s}_{n-1|n-1} = 1 \cdot \hat{s}_{n-1|n-1} = \hat{s}_{n-1|n-1}
        \end{equation*}

  \item \textbf{Prediction Error Covariance ($\Sigma_{n|n-1}$):}
        Using the general prediction covariance formula with $Q_n=0$:
        \begin{equation*}
          \Sigma_{n|n-1} = A_n \Sigma_{n-1|n-1} A_n^T + Q_n = 1 \cdot \Sigma_{n-1|n-1} \cdot 1 + 0 = \Sigma_{n-1|n-1}
        \end{equation*}

  \item \textbf{Kalman Gain ($K_n$):}
        Since $H_n$ and $R_n$ are scalars ($c_h$ and $\sigma_v^2$), the matrix inversion becomes a scalar division:
        \begin{equation*}
          K_n = \Sigma_{n|n-1} H_n^T (H_n \Sigma_{n|n-1} H_n^T + R_n)^{-1}
        \end{equation*}
        Substituting the known values:
        \begin{equation*}
          K_n = \frac{\Sigma_{n|n-1} \cdot c_h}{c_h^2 \Sigma_{n|n-1} + \sigma_v^2}
        \end{equation*}

  \item \textbf{Correction ($\hat{s}_{n|n}$):}
        Updating the a priori estimate with the new measurement $x_n$:
        \begin{equation*}
          \hat{s}_{n|n} = \hat{s}_{n|n-1} + K_n (x_n - H_n \hat{s}_{n|n-1})
        \end{equation*}
        Substituting the prediction from step 1:
        \begin{equation*}
          \hat{s}_{n|n} = \hat{s}_{n-1|n-1} + K_n (x_n - c_h \hat{s}_{n-1|n-1})
        \end{equation*}

  \item \textbf{Estimation Error Covariance ($\Sigma_{n|n}$):}
        Updating the error covariance. Note that the identity matrix $\mathbf{I}$ becomes the scalar $1$:
        \begin{equation*}
          \Sigma_{n|n} = (I - K_n H_n) \Sigma_{n|n-1}
        \end{equation*}
        Substituting scalars and the result from step 2:
        \begin{equation*}
          \Sigma_{n|n} = (1 - K_n c_h) \Sigma_{n-1|n-1}
        \end{equation*}
\end{enumerate}

%%%%%%%%%%%%%%%%%%%%%%%%%%%%%%%%%%%%%%%%%%%%%%%%%%%%%%%%%%%%%%%%%%%%%%%%%%%%%%%%%%%%%%%%%%%%%%%%%%%%%
%%%%%%%%%%%%%%%%%%%%%%%%%%%%%%%%%%%%%%%%%%%%%%%%%%%%%%%%%%%%%%%%%%%%%%%%%%%%%%%%%%%%%%%%%%%%%%%%%%%%%
\vspace{0.5cm}
\question{Question: How should we choose the initial estimate $\hat s_{-1|-1}$ and estimation error covariance $\Sigma_{-1|-1}$?}
\vspace{0.5cm}

Before collecting any measurements, we encode our a priori knowledge about the fluid level:

\begin{enumerate}
  \item \textbf{$\hat{s}_{-1|-1}$:}
        The best guess a priori for the fluid level is its prior mean:
        \begin{equation*}
          \hat{s}_{-1|-1} = \mu_\ell
        \end{equation*}
        
  \item \textbf{$\Sigma_{-1|-1}$:}
        Our uncertainty about this guess is quantified by the prior variance:
        \begin{equation*}
          \Sigma_{-1|-1} = \sigma_\ell^2
        \end{equation*}
\end{enumerate}

\textbf{Connection to Task 2:} These initial conditions ensure that $\hat{\ell}_0$ (obtained from the Kalman filter with $n=0$) matches the LMMSE formula with one observation, and similarly for $\Sigma_{0|0}$. This validates our choices.

%%%%%%%%%%%%%%%%%%%%%%%%%%%%%%%%%%%%%%%%%%%%%%%%%%%%%%%%%%%%%%%%%%%%%%%%%%%%%%%%%%%%%%%%%%%%%%%%%%%%%
%%%%%%%%%%%%%%%%%%%%%%%%%%%%%%%%%%%%%%%%%%%%%%%%%%%%%%%%%%%%%%%%%%%%%%%%%%%%%%%%%%%%%%%%%%%%%%%%%%%%%
\vspace{0.5cm}
\question{Question: Write a Matlab script\footnote{You may use the file {\tt kalman\_dc\_template.m} as starting point.} to simulate this dynamical system and the Kalman filter. Assume that the sensor measures pressure in mbar, fluid level is in cm, and that the fluid is benzene (density $874$ kg/m$^3$ at 25$^\circ$C).
  Our initial guess of the fluid level is $\mu_\ell=250$ cm, and we assign a standard deviation $\sigma_\ell = 11$ cm to reflect our uncertainty about this guess. Measurement errors are uncorrelated, zero-mean Gaussian with standard deviation $\sigma_v = 25$ mbar, and the true fluid level is $340$ cm. Plot the time evolution of the state, the measurements, and the estimate, up to $n=200$, and for two different executions. Also, plot the time evolution of the standard deviation of the fluid level estimation error, in cm. Comment on your results.
}

\subsubsection*{Simulation Results}

The Kalman filter was simulated for the constant fluid level estimation problem using the derived scalar equations. The system parameters were set to $\mu_l = 250$ cm, $\sigma_l = 11$ cm, and the true level $l_{true} = 340$ cm. The measurement noise standard deviation was $\sigma_v = 25$ mbar.

The simulation results for $N=200$ iterations and two different noise realizations are shown below:

\begin{figure}[h!]
  \centering
  % Sustituye 'figura1.png' por el nombre real de tu archivo de imagen
  \includegraphics[width=0.8\textwidth]{img/task1_4_1.png}
  \caption{Time evolution of the true state (black dashed), noisy measurements (dots), and Kalman filter estimates (colored lines) for two executions.}
  \label{fig:kalman_state}
\end{figure}

\begin{figure}[h!]
  \centering
  % Sustituye 'figura2.png' por el nombre real de tu archivo de imagen
  \includegraphics[width=0.8\textwidth]{img/task1_4_2.png}
  \caption{Time evolution of the standard deviation of the estimation error ($\sqrt{\Sigma_{n|n}}$) for two executions.}
  \label{fig:kalman_sigma}
\end{figure}

\subsubsection*{Comments on Results}

Based on the generated plots, we can draw the following conclusions:

\begin{itemize}
  \item \textbf{Convergence of the Estimate:} As seen in Figure \ref{fig:kalman_state}, the estimate starts at the initial guess ($\hat{s}_{-1|-1} = 250$ cm) and rapidly converges towards the true value ($340$ cm). The filter effectively corrects the large initial error (90 cm bias) within the first few iterations.

  \item \textbf{Noise Reduction:} The raw measurements (grey dots) exhibit high variance due to the sensor noise ($\sigma_v \approx 25$ mbar). The Kalman filter estimate behaves as a low-pass filter, smoothing out this noise. As time progresses ($n$ increases), the estimate becomes increasingly stable and closer to the true constant value.

  \item \textbf{Error Covariance Behavior:} Figure \ref{fig:kalman_sigma} shows that the standard deviation of the estimation error ($\sqrt{\Sigma_{n|n}}$) decreases monotonically from the initial uncertainty ($\sigma_l = 11$ cm) towards zero. This asymptotic behavior towards zero is expected because the process noise is zero ($Q=0$); the filter assumes the state is perfectly constant, so it accumulates infinite information over time, theoretically achieving perfect precision as $n \to \infty$.

  \item \textbf{Independence from Realizations:} The error standard deviation curves for both executions are identical and superimposed. This confirms a fundamental property of the Kalman Filter: the covariance evolution ($\Sigma_{n|n}$) depends only on the system matrices ($A, H, Q, R$) and is independent of the specific random values of the measurements ($x_n$). Therefore, the performance metric is deterministic and can be computed offline.
\end{itemize}

\vspace{0.5cm}

%%%%%%%%%%%%%%%%%%%%%%%%%%%%%%%%%%%%%%%%%%%%%%%%%%%%%%%%%%%%%%%%%%%%%%%%%%%%%%%%%%%%%%%%%%%%%%%%%%%%%
%%%%%%%%%%%%%%%%%%%%%%%%%%%%%%%%%%%%%%%%%%%%%%%%%%%%%%%%%%%%%%%%%%%%%%%%%%%%%%%%%%%%%%%%%%%%%%%%%%%%%
\vspace{0.5cm}
\question{Question: Using mathematical induction, prove that $\Sigma_{n|n} = \frac{\sigma_\ell^2}{1+(n+1)\alpha c_h^2}$ for all $n\geq 0$. Does this expression look familiar? Evaluate it for $n=200$ with the parameters of the previous point, and check the result of your simulation.  What is the asymptotic value of $\Sigma_{n|n}$?}

\vspace{0.5cm}
\begin{itemize}
  \item For n = 0, $\Sigma_{0|-1} = \Sigma_{-1|-1} = \sigma_\ell^2$, $Q=0$ and $A=1$:
        $$K_0 = \frac{\Sigma_{0|-1} c_h}{c_h^2 \Sigma_{0|-1} + \sigma_v^2} = \frac{\sigma_\ell^2 c_h}{c_h^2 \sigma_\ell^2 + \sigma_v^2}$$
        $$\Sigma_{0|0} = (1 - K_0 c_h) \Sigma_{0|-1}$$
        Substuying $K_0$:
        $$\Sigma_{0|0} = \left( 1 - \frac{\sigma_\ell^2 c_h^2}{c_h^2 \sigma_\ell^2 + \sigma_v^2} \right) \sigma_\ell^2 = \left( \frac{c_h^2 \sigma_\ell^2 + \sigma_v^2 - \sigma_\ell^2 c_h^2}{c_h^2 \sigma_\ell^2 + \sigma_v^2} \right) \sigma_\ell^2$$
        $$\Sigma_{0|0} = \frac{\sigma_v^2 \sigma_\ell^2}{c_h^2 \sigma_\ell^2 + \sigma_v^2} = \frac{\sigma_\ell^2}{c_h^2 \frac{\sigma_\ell^2}{\sigma_v^2} + 1} = \frac{\sigma_\ell^2}{1 + \alpha c_h^2}$$
        So, cheking the formula for n=0:
        $$\frac{\sigma_\ell^2}{1+(0+1)\alpha c_h^2} = \frac{\sigma_\ell^2}{1+\alpha c_h^2}$$
  \item For n = k:
        We know that $\Sigma_{k|k-1} = \Sigma_{k-1|k-1}$, so
        $$\frac{1}{\Sigma_{k|k}} = \frac{1}{\Sigma_{k|k-1}} + \frac{H^2}{R} = \frac{1}{\Sigma_{k-1|k-1}} + \frac{c_h^2}{\sigma_v^2}$$
        $$\frac{1}{\Sigma_{k|k}} = \frac{1 + k\alpha c_h^2}{\sigma_\ell^2} + \frac{c_h^2}{\sigma_v^2}$$
        $\alpha = \frac{\sigma_\ell^2}{\sigma_v^2}$, so $\frac{1}{\sigma_v^2} = \frac{\alpha}{\sigma_\ell^2}$
        $$\frac{1}{\Sigma_{k|k}} = \frac{1 + k\alpha c_h^2}{\sigma_\ell^2} + \frac{c_h^2 \alpha}{\sigma_\ell^2}$$
        $$\frac{1}{\Sigma_{k|k}} = \frac{1 + k\alpha c_h^2 + \alpha c_h^2}{\sigma_\ell^2} = \frac{1 + (k+1)\alpha c_h^2}{\sigma_\ell^2}$$
        $$\Sigma_{k|k} = \frac{\sigma_\ell^2}{1 + (k+1)\alpha c_h^2}$$
\end{itemize}

\subsubsection{Evaluate it for $n = 200$:}
\begin{itemize}
  \item $\sigma_\ell = 11 cm \implies \sigma_\ell^2 = 121 $
  \item $\sigma_v = 25 mbar \implies \sigma_v^2 = 625$
  \item $\alpha = \frac{121}{625} = 0.1936$
  \item Density $\rho = 0.874g/cm^3$
  \item $c_h = 0.98 \times \rho = 0.98 \times 0.874 = 0.85652 mbar/cm$
  \item $c_h^2 \approx 0.7336$
\end{itemize}

$$\Sigma_{200|200} = \frac{121}{1 + (200+1) \cdot 0.1936 \cdot 0.7336}$$
$$\Sigma_{200|200} = \frac{121}{1 + 201 \cdot 0.1420}$$
$$\Sigma_{200|200} = \frac{121}{1 + 28.547} = \frac{121}{29.547} \approx 4.095 \text{ cm}^2$$

$$\sqrt{\Sigma_{200|200}} = \sqrt{4.095} \approx \mathbf{2.02 \text{ cm}}$$
In the plot of the Figure \ref{fig:kalman_sigma} we can see how the curve drop to approximately 2 at $n = 200$

%%%%%%%%%%%%%%%%%%%%%%%%%%%%%%%%%%%%%%%%%%%%%%%%%%%%%%%%%%%%%%%%%%%%%%%%%%%%%%%%%%%%%%%%%%%%%%%%%%%%%
%%%%%%%%%%%%%%%%%%%%%%%%%%%%%%%%%%%%%%%%%%%%%%%%%%%%%%%%%%%%%%%%%%%%%%%%%%%%%%%%%%%%%%%%%%%%%%%%%%%%%
\vspace{0.5cm}
\question{Question: Show that
  \begin{equation*}\label{eq:est_2}
    \hat s_{n|n} = \frac{1+n\alpha c_h^2}{1+n\alpha c_h^2 + \alpha c_h^2} \hat s_{n-1|n-1} + \frac{\alpha c_h}{1+ n\alpha c_h^2 + \alpha c_h^2}x_n.
  \end{equation*}
  Compare this with your answer to the first point of Task 2 and give your conclusions.}

\vspace{0.5cm}

\subsubsection*{Derivation}

From the Kalman filter correction step, we have:
\begin{equation*}
  \hat{s}_{n|n} = \hat{s}_{n-1|n-1} + K_n(x_n - c_h \hat{s}_{n-1|n-1})
\end{equation*}

Rearranging:
\begin{equation*}
  \hat{s}_{n|n} = (1 - K_n c_h)\hat{s}_{n-1|n-1} + K_n x_n
\end{equation*}

From the previous question, we proved that:
\begin{equation*}
  \Sigma_{n|n} = \frac{\sigma_\ell^2}{1+(n+1)\alpha c_h^2}
\end{equation*}

We also know that $\Sigma_{n|n-1} = \Sigma_{n-1|n-1} = \frac{\sigma_\ell^2}{1+n\alpha c_h^2}$ (since $Q_n=0$).

The Kalman gain is:
\begin{equation*}
  K_n = \frac{\Sigma_{n|n-1} c_h}{c_h^2 \Sigma_{n|n-1} + \sigma_v^2}
\end{equation*}

Substituting $\Sigma_{n|n-1}$:
\begin{align*}
  K_n &= \frac{\frac{\sigma_\ell^2}{1+n\alpha c_h^2} c_h}{c_h^2 \frac{\sigma_\ell^2}{1+n\alpha c_h^2} + \sigma_v^2} \\
  &= \frac{\sigma_\ell^2 c_h}{c_h^2 \sigma_\ell^2 + \sigma_v^2(1+n\alpha c_h^2)} \\
  &= \frac{\sigma_\ell^2 c_h}{\sigma_v^2(1+n\alpha c_h^2 + \alpha c_h^2)} \quad \text{(using } \alpha = \frac{\sigma_\ell^2}{\sigma_v^2}\text{)} \\
  &= \frac{\alpha c_h}{1+(n+1)\alpha c_h^2}
\end{align*}

Now, compute $1 - K_n c_h$:
\begin{align*}
  1 - K_n c_h &= 1 - \frac{\alpha c_h^2}{1+(n+1)\alpha c_h^2} \\
  &= \frac{1+(n+1)\alpha c_h^2 - \alpha c_h^2}{1+(n+1)\alpha c_h^2} \\
  &= \frac{1+n\alpha c_h^2}{1+(n+1)\alpha c_h^2}
\end{align*}

Substituting back into the correction equation:
\begin{equation*}
  \boxed{\hat{s}_{n|n} = \frac{1+n\alpha c_h^2}{1+(n+1)\alpha c_h^2} \hat{s}_{n-1|n-1} + \frac{\alpha c_h}{1+(n+1)\alpha c_h^2} x_n}
\end{equation*}

Note that $1+(n+1)\alpha c_h^2 = 1+n\alpha c_h^2 + \alpha c_h^2$, so this is exactly the desired form.

\subsubsection*{Comparison with Task 2}

In Task 2, Question 1, we derived the recursive LMMSE estimator:
\begin{equation*}
  \hat{\ell}_n = \frac{1+n\alpha c_h^2}{1+(n+1)\alpha c_h^2} \hat{\ell}_{n-1} + \frac{\alpha c_h}{1+(n+1)\alpha c_h^2} x_n
\end{equation*}

\textbf{Conclusion:} The Kalman filter estimate $\hat{s}_{n|n}$ is \textbf{identical} to the recursive LMMSE estimator $\hat{\ell}_n$ from Task 2. This confirms a fundamental result:

\begin{center}
  \fbox{\parbox{0.9\textwidth}{%
      \textbf{The Kalman filter is the optimal recursive implementation of the LMMSE estimator for linear Gaussian systems.}
    }}
\end{center}

Both approaches yield the same estimate, but the Kalman filter framework provides:
\begin{itemize}
  \item A systematic way to incorporate process dynamics (state evolution)
  \item Error covariance tracking ($\Sigma_{n|n}$)
  \item Generalization to multivariate and time-varying systems
  \item A recursive structure with constant computational complexity $O(1)$
\end{itemize}

%%%%%%%%%%%%%%%%%%%%%%%%%%%%%%%%%%%%%%%%%%%%%%%%%%%%%%%%%%%%%%%%%%%%%%%%%%%%%%%%%%%%%%%%%%%%%%%%%%%%%
%%%%%%%%%%%%%%%%%%%%%%%%%%%%%%%%%%%%%%%%%%%%%%%%%%%%%%%%%%%%%%%%%%%%%%%%%%%%%%%%%%%%%%%%%%%%%%%%%%%%%
\vspace{0.5cm}
\question{Question: Repeat your simulations after changing $\sigma_\ell$ to 50 cm. Repeat again for $\sigma_\ell = 5$ cm. Comment on your results.}
\subsection*{Analysis of Sensitivity to Initial Uncertainty}

We repeated the simulations modifying the standard deviation of the initial guess $\sigma_\ell$. The results are shown in Figure \ref{fig:sensitivity} and analyzed below.

% --- Gráfica de Sensibilidad (Figura 3 de Matlab) ---
\begin{figure}[H]
  \centering
  \includegraphics[width=0.85\textwidth]{img/task3_7.png}
  \caption{Effect of initial uncertainty $\sigma_\ell$ on the convergence speed. High uncertainty ($\sigma_\ell=50$) leads to fast convergence but high initial sensitivity. High confidence ($\sigma_\ell=5$) leads to slow convergence.}
  \label{fig:sensitivity}
\end{figure}

\subsubsection*{Case A: High Uncertainty ($\sigma_\ell = 50$ cm)}
\begin{itemize}
  \item \textbf{Observation:} The estimation curve jumps very rapidly from the initial guess ($250$) towards the true value ($340$) in just the first few samples ($n=1$ to $n=5$).
  \item \textbf{Explanation:} By setting a high initial variance ($\Sigma_{-1|-1} = 2500$), the Kalman Gain ($K_0$) becomes very high. The filter assigns much more weight to the incoming measurements than to the previous estimate.
  \item \textbf{Pros/Cons:} It corrects the initial bias ($90$ cm error) extremely fast, but the estimate is more sensitive to measurement noise in the very first steps.
\end{itemize}

\subsubsection*{Case B: Low Uncertainty / High Confidence ($\sigma_\ell = 5$ cm)}
\begin{itemize}
  \item \textbf{Observation:} The estimation curve rises very slowly. It takes many iterations to get close to the true level of $340$ cm.
  \item \textbf{Explanation:} We are telling the filter that we are ``sure'' the level is $250 \pm 5$. The filter trusts this initial information heavily and treats the measurements as noise. The Kalman Gain is initially very small.
  \item \textbf{Pros/Cons:} The filter is very smooth, but fails to correct the large initial bias quickly (overconfidence).
\end{itemize}