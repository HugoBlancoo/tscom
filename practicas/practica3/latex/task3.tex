\textbf{Question: Find a ``state evolution'' and ``measurement'' description for (1) in the form of (4).
    Note that the state of the system (the fluid level) is static, i.e., $s_n = s$ is constant with $n$.}
\vspace{0.5cm}

Given that the fluid level is static ($s_n = s_{n-1} = l$) and a scalar, the state evolution equation simplifies.
We identify the state transition matrix $A_n$ and the process noise $u_n$:
\begin{equation}
    s_n = 1 \cdot s_{n-1} + 0
\end{equation}
Thus, identifying terms with the general form $s_n = A_n s_{n-1} + G_n f_n + u_n$:
\begin{itemize}
    \item State variable: $s_n = l$ (scalar).
    \item State transition matrix: $A_n = 1$.
    \item Control input: $G_n = 0$ (no external forcing).
    \item Process noise: $u_n = 0$, which implies the process noise covariance is $Q_n = 0$.
\end{itemize}
For the measurement equation, we relate the pressure observation $x_n$ to the fluid level using the hydrostatic constant $c_h$:
\begin{equation}
    x_n = c_h \cdot s_n + v_n
\end{equation}
Comparing this to $x_n = H_n s_n + w_n$:
\begin{itemize}
    \item Observation matrix: $H_n = c_h$.
    \item Measurement noise: $w_n = v_n$.
    \item Measurement noise covariance: $R_n = \sigma_v^2$ (given variance of $v_k$).
\end{itemize}

\textbf{Question:Derive the Kalman filter equations for this problem, in order to obtain the LMMSE estimate of the state given the observations $x_0$,\ldots,$x_n$ ( i.e., $\hat s_{n|n}$).}
\vspace{0.5cm}

\subsubsection*{Derivation of the Kalman Filter Equations}

Since the system state is a scalar constant ($s_n = s_{n-1}$), we identified that $A_n = 1$, $G_n = 0$, $Q_n = 0$. The measurement follows $x_n = c_h s_n + v_n$, implying $H_n = c_h$ and $R_n = \sigma_v^2$. The Kalman filter recursive steps for $n \ge 0$ are derived as follows:

\begin{enumerate}
    \item \textbf{Prediction ($\hat{s}_{n|n-1}$):}
          Using the state evolution equation with $f_n=0$:
          \begin{equation}
              \hat{s}_{n|n-1} = A_n \hat{s}_{n-1|n-1} = 1 \cdot \hat{s}_{n-1|n-1} = \hat{s}_{n-1|n-1}
          \end{equation}

    \item \textbf{Prediction Error Covariance ($\Sigma_{n|n-1}$):}
          Using the general prediction covariance formula with $Q_n=0$:
          \begin{equation}
              \Sigma_{n|n-1} = A_n \Sigma_{n-1|n-1} A_n^T + Q_n = 1 \cdot \Sigma_{n-1|n-1} \cdot 1 + 0 = \Sigma_{n-1|n-1}
          \end{equation}

    \item \textbf{Kalman Gain ($K_n$):}
          Since $H_n$ and $R_n$ are scalars ($c_h$ and $\sigma_v^2$), the matrix inversion becomes a scalar division:
          \begin{equation}
              K_n = \Sigma_{n|n-1} H_n^T (H_n \Sigma_{n|n-1} H_n^T + R_n)^{-1}
          \end{equation}
          Substituting the known values:
          \begin{equation}
              K_n = \frac{\Sigma_{n|n-1} \cdot c_h}{c_h^2 \Sigma_{n|n-1} + \sigma_v^2}
          \end{equation}

    \item \textbf{Correction ($\hat{s}_{n|n}$):}
          Updating the a priori estimate with the new measurement $x_n$:
          \begin{equation}
              \hat{s}_{n|n} = \hat{s}_{n|n-1} + K_n (x_n - H_n \hat{s}_{n|n-1})
          \end{equation}
          Substituting the prediction from step 1:
          \begin{equation}
              \hat{s}_{n|n} = \hat{s}_{n-1|n-1} + K_n (x_n - c_h \hat{s}_{n-1|n-1})
          \end{equation}

    \item \textbf{Estimation Error Covariance ($\Sigma_{n|n}$):}
          Updating the error covariance. Note that the identity matrix $\mathbf{I}$ becomes the scalar $1$:
          \begin{equation}
              \Sigma_{n|n} = (I - K_n H_n) \Sigma_{n|n-1}
          \end{equation}
          Substituting scalars and the result from step 2:
          \begin{equation}
              \Sigma_{n|n} = (1 - K_n c_h) \Sigma_{n-1|n-1}
          \end{equation}
\end{enumerate}


\textbf{Question: How should we choose the initial estimate $\hat s_{-1|-1}$ and estimation error covariance $\Sigma_{-1|-1}$?}
\vspace{0.5cm}

\begin{enumerate}
    \item \textbf{$\hat{s}_{-1|-1}$:}
          The initial value corresponds to the best best guess \textit{a priori} for the fluid level. This is the mean of the random variable.
          \begin{equation}
              \hat{s}_{-1|-1} = \mu_l
          \end{equation}
    \item \textbf{$\Sigma_{-1|-1}$:}
          \begin{equation}
              \Sigma_{-1|-1} = \sigma_l^2
          \end{equation}
\end{enumerate}

\textbf{Question: Write a Matlab script\footnote{You may use the file {\tt kalman\_dc\_template.m} as starting point.} to simulate this dynamical system and the Kalman filter. Assume that the sensor measures pressure in mbar, fluid level is in cm, and that the fluid is benzene (density $874$ kg/m$^3$ at 25$^\circ$C).
    Our initial guess of the fluid level is $\mu_\ell=250$ cm, and we assign a standard deviation $\sigma_\ell = 11$ cm to reflect our uncertainty about this guess. Measurement errors are uncorrelated, zero-mean Gaussian with standard deviation $\sigma_v = 25$ mbar, and the true fluid level is $340$ cm. Plot the time evolution of the state, the measurements, and the estimate, up to $n=200$, and for two different executions. Also, plot the time evolution of the standard deviation of the fluid level estimation error, in cm. Comment on your results.
}

\subsubsection*{Simulation Results}

The Kalman filter was simulated for the constant fluid level estimation problem using the derived scalar equations. The system parameters were set to $\mu_l = 250$ cm, $\sigma_l = 11$ cm, and the true level $l_{true} = 340$ cm. The measurement noise standard deviation was $\sigma_v = 25$ mbar.

The simulation results for $N=200$ iterations and two different noise realizations are shown below:

\begin{figure}[h!]
    \centering
    % Sustituye 'figura1.png' por el nombre real de tu archivo de imagen
    \includegraphics[width=0.8\textwidth]{img/task1_4_1.png}
    \caption{Time evolution of the true state (black dashed), noisy measurements (dots), and Kalman filter estimates (colored lines) for two executions.}
    \label{fig:kalman_state}
\end{figure}

\begin{figure}[h!]
    \centering
    % Sustituye 'figura2.png' por el nombre real de tu archivo de imagen
    \includegraphics[width=0.8\textwidth]{img/task1_4_2.png}
    \caption{Time evolution of the standard deviation of the estimation error ($\sqrt{\Sigma_{n|n}}$) for two executions.}
    \label{fig:kalman_sigma}
\end{figure}

\subsubsection*{Comments on Results}

Based on the generated plots, we can draw the following conclusions:

\begin{itemize}
    \item \textbf{Convergence of the Estimate:} As seen in Figure \ref{fig:kalman_state}, the estimate starts at the initial guess ($\hat{s}_{-1|-1} = 250$ cm) and rapidly converges towards the true value ($340$ cm). The filter effectively corrects the large initial error (90 cm bias) within the first few iterations.

    \item \textbf{Noise Reduction:} The raw measurements (grey dots) exhibit high variance due to the sensor noise ($\sigma_v \approx 25$ mbar). The Kalman filter estimate behaves as a low-pass filter, smoothing out this noise. As time progresses ($n$ increases), the estimate becomes increasingly stable and closer to the true constant value.

    \item \textbf{Error Covariance Behavior:} Figure \ref{fig:kalman_sigma} shows that the standard deviation of the estimation error ($\sqrt{\Sigma_{n|n}}$) decreases monotonically from the initial uncertainty ($\sigma_l = 11$ cm) towards zero. This asymptotic behavior towards zero is expected because the process noise is zero ($Q=0$); the filter assumes the state is perfectly constant, so it accumulates infinite information over time, theoretically achieving perfect precision as $n \to \infty$.

    \item \textbf{Independence from Realizations:} The error standard deviation curves for both executions are identical and superimposed. This confirms a fundamental property of the Kalman Filter: the covariance evolution ($\Sigma_{n|n}$) depends only on the system matrices ($A, H, Q, R$) and is independent of the specific random values of the measurements ($x_n$). Therefore, the performance metric is deterministic and can be computed offline.
\end{itemize}

\vspace{0.5cm}

\textbf{Question: Using mathematical induction, prove that $\Sigma_{n|n} = \frac{\sigma_\ell^2}{1+(n+1)\alpha c_h^2}$ for all $n\geq 0$. Does this expression look familiar? Evaluate it for $n=200$ with the parameters of the previous point, and check the result of your simulation.  What is the asymptotic value of $\Sigma_{n|n}$?}

\vspace{0.5cm}
