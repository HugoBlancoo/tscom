\appendix
\section{Appendix: MATLAB scripts and data}
\label{app:matlab}

\subsection{Task 1}
\subsubsection{task1\_3.m}


\begin{lstlisting}[language=Matlab]
    A = 1;
    w0 = pi/2;
    omega = linspace(-pi, pi, 2000);

    W_of = @(w) ( (abs(w) <= w0) .* ( A .* (1 - abs(w)/w0) ) );

    W = W_of(omega);

    Lvals = [2, 3];

    figure('Units','normalized','Position',[0.1 0.1 0.6 0.7]);
    subplot(3,1,1);
    plot(omega, W, 'LineWidth', 1.5);
    xlim([-pi pi]); ylim([0 1.05*A]);
    xlabel('$\omega (rad/sample)$','Interpreter','latex');
    ylabel('$W(e^{j\omega})$','Interpreter','latex');
    title('Original spectrum $W(e^{j\omega})$','Interpreter','latex');
    grid on;

    for i=1:length(Lvals)
        L = Lvals(i);
        omegaL = omega * L;
        omegaL_wrapped = mod(omegaL + pi, 2*pi) - pi;
        Wbar = W_of(omegaL_wrapped);

        subplot(3,1,i+1);
        plot(omega, Wbar, 'LineWidth', 1.5);
        xlim([-pi pi]); ylim([0 1.05*A]);
        xlabel('$\omega (rad/sample)$','Interpreter','latex');
        ylabel(['$\bar{W}(e^{j\omega})$, $L=' num2str(L) '$'],'Interpreter','latex');
        title(['Spectrum $\bar{W}(e^{j\omega}) = W(e^{j\omega ' num2str(L) '})$, $L=' num2str(L) '$'],...
            'Interpreter','latex');
        grid on;
    end

\end{lstlisting}

\subsubsection{task1\_4.m}

\begin{lstlisting}[language=Matlab]
    clear; close all; clc;

    Ls = [2 3];
    order = 80;
    window = @hamming;
    nfft = 8192;
    omega = linspace(-pi, pi, nfft);

    figure('Units','normalized','Position',[0.1 0.1 0.7 0.8]);

    for k = 1:length(Ls)
        L = Ls(k);
        Wn = 1/L;
        
        h = fir1(order, Wn, window(order+1));
        
        H = fftshift(fft(h, nfft));
        Hmag = abs(H);
        Hdb  = 20*log10(max(Hmag,1e-12));
        
        subplot(3,1,k);
        plot(omega, Hmag, 'LineWidth', 1.6); hold on;
        yyaxis right;
        plot(omega, Hdb, '--', 'LineWidth', 1);
        yyaxis left;
        
        xline(-pi/L, '--k', 'LineWidth', 1);
        xline( pi/L, '--k', 'LineWidth', 1);
        
        xlabel('$\omega$ (rad/sample)','Interpreter','latex');
        yyaxis left; ylabel('$|G_{TX}(e^{j\omega})|$','Interpreter','latex');
        yyaxis right; ylabel('Magnitude (dB)','Interpreter','latex');
        title(['$G_{TX}(e^{j\omega})$ for $L = ', num2str(L), '$'],'Interpreter','latex');
        xlim([-pi pi]); grid on;
        legend({'Linear magnitude','Magnitude (dB)','$\omega_c = \pm \pi/L$'},...
            'Interpreter','latex','Location','best');
    end

    subplot(3,1,3);
    colors = {'b','r'};
    for k = 1:length(Ls)
        L = Ls(k);
        h = fir1(order, 1/L, window(order+1));
        H = fftshift(fft(h, nfft));
        plot(omega, abs(H), 'LineWidth', 1.8, 'Color', colors{k}); hold on;
        xline(-pi/L, '--', 'Color', colors{k}, 'HandleVisibility','off');
        xline( pi/L, '--', 'Color', colors{k}, 'HandleVisibility','off');
    end
    xlabel('$\omega$ (rad/sample)','Interpreter','latex');
    ylabel('$|G_{TX}(e^{j\omega})|$','Interpreter','latex');
    title('Overlay of $G_{TX}(e^{j\omega})$ for $L=2$ and $L=3$','Interpreter','latex');
    legend({'$L=2$','$L=3$'},'Interpreter','latex','Location','best');
    xlim([-pi pi]); ylim([0 1.1]);
    grid on;
\end{lstlisting}

\subsection{Task 3}
\subsubsection{task3\_1.m}

\begin{lstlisting}[language=Matlab]
    close all;

    N=512;
    delta_c = 31.250e3;
    prefix_redundancy = 0.0655;

    rng(2025);
    M = 4;
    dataSymbols = randi([0 M-1], 10000, 1);
    txSig = pskmod(dataSymbols, M, pi/M);
    scatterplot(awgn(txSig,20))

    OF = 2;
    Lc = round(prefix_redundancy * N);

    data = txSig.';
    [x, u, w] = OFDMmod(data, N, Lc, OF);

    Fs = OF * N * delta_c;

    figure;
    pwelch(x, 512, [], 512, Fs);

    figure;
    pwelch(x, 512, [], 512, Fs, 'centered')

    [Px, f] = pwelch(x, 512, [], 512, Fs, 'centered');
    [Pu, f2] = pwelch(sqrt(OF)*u, 512, [], 512, Fs, 'centered');
    figure;
    plot(f, 10*log10(Px)); hold on;
    plot(f2, 10*log10(Pu));
    grid on;

    xlabel('Frequency (Hz)');
    ylabel('Power/frequency (dB/Hz)');
    legend('PSD of x', 'PSD of sqrt(OF)*u');
    title('PSD comparison on the same axis');

    P = 150;
    gtx = srrc(0, P, OF);
    figure;
    freqz(gtx, 1, 2048, Fs);
    grid on;
\end{lstlisting}

\subsubsection{task3\_2.m}
\begin{lstlisting}[language=Matlab]
    
    close all;

    N=512;
    delta_c = 31.250e3; % subcarrier_spacing in Hz
    prefix_redundancy = 0.0655; % 6.55%

    % random QPSK data symbols to modulate with OFDM
    rng(2025);                                  % Set seed for reproducibility
    M = 4;
    dataSymbols = randi([0 M-1], 10000, 1);     % Generate 10000 random QPSK symbols (0, 1, 2, 3)
    txSig = pskmod(dataSymbols, M, pi/M);        % QPSK modulation
    scatterplot(awgn(txSig,20))

    OF = 3;
    Lc = round(prefix_redundancy * N);

    data = txSig.';
    [x, u, w] = OFDMmod(data, N, Lc, OF);

    Fs = OF * N * delta_c;       % frecuencia de muestreo tras oversampling y filtro

    figure;
    pwelch(x, 512, [], 512, Fs);

    figure;
    pwelch(x, 512, [], 512, Fs, 'centered')

    %% With same X and Y axis
    [Px, f] = pwelch(x, 512, [], 512, Fs, 'centered');
    [Pu, f2] = pwelch(sqrt(OF)*u, 512, [], 512, Fs, 'centered');
    figure;
    plot(f, 10*log10(Px)); hold on;
    plot(f2, 10*log10(Pu));
    grid on;

    xlabel('Frequency (Hz)');
    ylabel('Power/frequency (dB/Hz)');
    legend('PSD of x', 'PSD of sqrt(OF)*u');
    title('PSD comparison on the same axis');

    P = 150;
    gtx = srrc(0, P, OF);
    figure;
    freqz(gtx, 1, 2048, Fs);
    grid on;
\end{lstlisting}

\subsubsection{task3\_3.m}
\begin{lstlisting}[language=Matlab]
    clear; close all;

    N=512;
    delta_c = 31.250e3; % subcarrier_spacing in Hz
    prefix_redundancy = 0.0655; % 6.55%

    % random QPSK data symbols to modulate with OFDM
    rng(2025);                                  % Set seed for reproducibility
    M = 4;
    dataSymbols = randi([0 M-1], 10000, 1);     % Generate 10000 random QPSK symbols (0, 1, 2, 3)
    txSig = pskmod(dataSymbols, M, pi/M);        % QPSK modulation
    OF = 2;
    Lc = round(prefix_redundancy * N);
    data = txSig.';

    Fs = OF * N * delta_c;
    %k_list = [0, 10, 20, 30, 40, 50, 100, 150, 200, 225, 230, 240, 250];
    k_list = [235:240 245:250];

    for k = k_list
        nullpos = [1:k, N-k+1:N]; % null in the edges
        [x, u, w] = OFDMmod(data, N, Lc, OF, nullpos);

        [Px, f] = pwelch(x, 512, [], 512, Fs, 'centered');
        % figure;
        % plot(f, 10*log10(Px));
        % xlabel('Frequency (Hz)');
        % ylabel('Power/frequency (dB/Hz)');

        psd_peak = max(10*log10(Px));
        idx_p7 = find(f>=7e6,1);
        idx_m7 = find(f<=-7e6,1,'last');
        psd_at_p7 = 10*log10(Px(idx_p7));
        psd_at_m7 = 10*log10(Px(idx_m7));
        att_p7 = psd_peak - psd_at_p7;
        att_m7 = psd_peak - psd_at_m7;
        fprintf('k = %d, Attenuation at +7MHz = %6.2f(dB/Hz), at -7MHz = %6.2f(dB/Hz), min = %6.2f\n',k,att_p7,att_m7,min(att_m7,att_p7));
    end
\end{lstlisting}

\subsubsection{OFDMmod.m}

\begin{lstlisting}[language=Matlab]
    function [x, u, w] = OFDMmod(data, N, Lc, OF, nullpos)

    % function [x, u, w] = OFDMmod(data, N, Lc, OF, nullpos)
    %
    % Simulates OFDM modulation
    % Input: 
    %  data    = row vector with (frequency-domain) data to be modulated
    %  N       = IFFT size
    %  Lc      = length of cyclic prefix, in samples
    %  OF      = oversampling factor (sinc pulse shaping)
    %  nullpos = vector with indices (within 1:N) of null subcarriers
    % The data vector will be zero-padded if necessary in order to construct 
    % an integer number of OFDM symbols.
    % Output:
    %  x = filtered   time domain samples ( OF*(N+Lc) samples per OFDM symbol )
    %  u = unfiltered time domain samples ( OF*(N+Lc) samples per OFDM symbol )
    %  w = time domain samples            (    (N+Lc) samples per OFDM symbol )

    if nargin==4
        K = 0; 
        datapos = [1:N];
    else
        K = length(nullpos);                % no. of null subcarriers
        datapos = setdiff([1:N], nullpos);  % indices of data subcarriers
    end

    Nu = N-K;   % No. of useful subcarriers

    %% Format data for IFFT
        % hint: you may want to use 'reshape'
    Nsymbols = ceil( length(data) / Nu );

    s_n = [data, zeros(1, Nsymbols * Nu - length(data))]; % relleno de 0s

    data_blocks = reshape(s_n , Nu, Nsymbols).'; % S to P

    si_n = zeros(Nsymbols, N);
    si_n(:, datapos) = data_blocks;

    %% N-point IFFT operation

    % ifft(Y,[],2) devuelve la transformada de Fourier inversa de cada fila.
    out_IFFT = ifft(si_n, N, 2);

    %% Add Cyclic Prefix

    CP = out_IFFT(:, end-Lc+1:end); % cogemos las ultimas Lc mustras
    wi_n = [CP, out_IFFT];

    %% Parallel to serial
        % hint: you may want to use 'reshape'
    w_n = reshape(wi_n.', 1, []);
    w = w_n;

    %% Upsample
    u = zeros(1,OF*length(w));
    u(1:OF:end) = w;

    P = 150;
    gtx = srrc(0, P, OF);
    x = filter(gtx,1,u);
\end{lstlisting}

\subsection{Task 4}
\subsection{task4\_1.m}
\begin{lstlisting}[language=Matlab]
    close all; clear;

    N=512;
    nullpos = [];
    delta_c = 31.250e3; % subcarrier_spacing in Hz
    prefix_redundancy = 0.0655; % 6.55%
    OF = 2;
    Lc = round(prefix_redundancy * N);

    % random QPSK data symbols to modulate with OFDM
    rng(2025);                              % Set seed for reproducibility
    M = 4;
    dataSymbols = randi([0 M-1], 10000, 1); % Generate 10000 random QPSK symbols (0, 1, 2, 3)
    txSig = pskmod(dataSymbols, M, pi/M); % QPSK modulation
    scatterplot(awgn(txSig,20));
    hold on; grid on;

    data = txSig.';

    [x, u, w] = OFDMmod(data, N, Lc, OF, nullpos);

    dem_data = OFDMdem(x, N, Lc, OF, ones(N,1), nullpos);

    scatter(real(dem_data), imag(dem_data));
    legend('Transmitted QPSK', 'Received QPSK', 'Location', 'best');
    title('QPSK Constellation: Transmitted vs Received');
    xlabel('In-Phase');
    ylabel('Quadrature');
\end{lstlisting}

\subsubsection{task4\_1\_16\_qam.m}
\begin{lstlisting}[language=Matlab]
    close all; clear;

    N=512;
    nullpos = [];
    delta_c = 31.250e3; % subcarrier_spacing in Hz
    prefix_redundancy = 0.0655; % 6.55%
    OF = 2;
    Lc = round(prefix_redundancy * N);

    % random QPSK data symbols to modulate with OFDM
    rng(2025);                              % Set seed for reproducibility
    M = 16;
    dataSymbols = randi([0 M-1], 10000, 1); % Generate 10000 random 16 QAM
    txSig = qammod(dataSymbols, M, 'gray'); % 16 QAM modulation
    scatterplot(awgn(txSig,20));
    hold on; grid on;

    data = txSig.';

    [x, u, w] = OFDMmod(data, N, Lc, OF, nullpos);

    dem_data = OFDMdem(x, N, Lc, OF, ones(N,1), nullpos);

    scatter(real(dem_data), imag(dem_data));
    legend('Transmitted 16 QAM', 'Received 16 QAM', 'Location', 'best');
    title('16-QAM Constellation: Transmitted vs Received');
    xlabel('In-Phase');
    ylabel('Quadrature');
\end{lstlisting}

\subsubsection{task4\_2.m}
\begin{lstlisting}[language=Matlab]
    close all; clear;

    N=512;
    delta_c = 31.250e3; % subcarrier_spacing in Hz
    prefix_redundancy = 0.0655; % 6.55%
    OF = 2;
    Lc = round(prefix_redundancy * N);

    % random QPSK data symbols to modulate with OFDM
    rng(2025);                              % Set seed for reproducibility
    M = 4;
    dataSymbols = randi([0 M-1], 10000, 1); % Generate 10000 random QPSK symbols (0, 1, 2, 3)
    txSig = pskmod(dataSymbols, M, pi/M); % QPSK modulation
    scatterplot(awgn(txSig,20));
    grid on;

    data = txSig.';

    Fs = OF * N * delta_c;
    k_list = [0 5 20 40];% 60 80 100];

    figure;
    for i = 1:length(k_list)
        k = k_list(i);
        nullpos = [1:k, N-k+1:N]; % null in the edges

        [x, u, w] = OFDMmod(data, N, Lc, OF, nullpos);
        dem_data = OFDMdem(x, N, Lc, OF, ones(N,1), nullpos);

        subplot(2,2,i);
        %figure;
        scatter(real(dem_data), imag(dem_data));
        grid on;
        axis equal;
        title(sprintf('k = %d', k));
    end

\end{lstlisting}

\subsubsection{OFDMdem.m}
\begin{lstlisting}[language=Matlab]
    function data = OFDMdem(r, N, Lc, OF, H, nullpos)

    % function [data, H] = OFDMdem(r, N, Lc, OF, H, nullpos)
    %
    % Simulates OFDM modulation
    % Input: 
    %  r       = row vector with received signal samples 
    %               ( OF*(N+Lc) samples/OFDM symbol )
    %  N       = IFFT size
    %  Lc      = length of cyclic prefix, in samples
    %  OF      = oversampling factor
    %  H       = column vector with frequency response of channel, to be used in FEQ
    %  nullpos = vector with indices (within 1:N) of null subcarriers
    % If the number of received samples (after taking into account the delays 
    % of the pulse-shaping and matched filters) is not an integer multiple of 
    % N+Lc, the last samples will be discarded. 
    % Output:
    %  data = row vector with demodulated data (data in null subcarriers is discarded)

    if nargin==5
        K = 0; 
        datapos = [1:N];
    else
        K = length(nullpos);                % no. of null subcarriers
        datapos = setdiff([1:N], nullpos);  % indices of data subcarriers
    end

    %% Filtering and downsampling
    P = 150;
    pulse = srrc(0, P, OF);
    rxsig = filter(pulse,1,r);
    % throw away initial samples due to filter delay
    rxdec = rxsig(2*P*OF+1:OF:end);

    %% Format data for processing
    Lsymb = N + Lc;
    Nsymbols = floor(length(rxdec)/Lsymb);
    rxdec = rxdec(1:Nsymbols * Lsymb);              % Descarta muestras incompletas
    rxmat = reshape(rxdec, Lsymb, Nsymbols).';

    rxmat_noCP = rxmat(:, Lc+1:end);

    Si_n = fft(rxmat_noCP, N, 2);

    Si_eq = Si_n ./ H.';

    %% Discard null subcarriers
    Si_data = Si_eq(:,datapos);

    %% Parallel to serial
    % Recupera el vector original de datos en fila

    data = reshape(Si_data.', 1, []);

    end
\end{lstlisting}

\subsection{Task 5}

\subsubsection{task5\_1.m}
\begin{lstlisting}[language=Matlab]
    close all; clear;

N=64;
T = 0.25e-6;
nullpos = [29, 30, 31, 32, 33, 34] + 1; %matlab indexa desde 1
prefix_redundancy = 0.09375; % 9.375%
Lc = round(prefix_redundancy * N);

OF = 10;
Fs = OF / T;

% random QPSK data symbols to modulate with OFDM
rng(2025);                              % Set seed for reproducibility
M = 16;
dataSymbols = randi([0 M-1], 10000, 1); % Generate 10000 random 16 QAM symbols
txSig = qammod(dataSymbols, M, 'UnitAveragePower', true); % 16 QAM modulation
scatterplot(awgn(txSig,20));
hold on; grid on;

data = txSig.';

[x, u, w] = OFDMmod(data, N, Lc, OF, nullpos);

% Define multipath channel htilde(t)
% h_tilde(t) = sum of Dirac deltas at specified delays
tau = [0, 2*T, 3*T, 4*T];              % Delays in seconds
alpha = [1, 0.7, 0.4, 0.5];            % Amplitudes
phi = [0, -pi/2, pi/4, pi/2];          % Phases

% Convert to complex channel taps
h_tilde = alpha .* exp(1j * phi);

% Convert delays to sample indices at rate Fs
delay_samples = round(tau * Fs);

% Create discrete-time channel impulse response
max_delay = max(delay_samples);
h_tilde_discrete = zeros(1, max_delay + 1);
h_tilde_discrete(delay_samples + 1) = h_tilde;

% Generate transmit and receive filters
P = 150;
gtx = srrc(0, P, OF);            % Transmit filter
grx = srrc(0, P, OF);            % Receive (matched) filter

%% Compute equivalent channel h(t) = gtx * h_tilde * grx
h_aux = conv(gtx, h_tilde_discrete);
heq = conv(h_aux, grx);

% Compute the FFT Heq = fft(heq,8192)
Nfft = 8192;
Heq = fft(heq,8192);

% Plot Heq magnitude
figure;
plot(linspace(-20, 20, 8192), 20*log10(abs(fftshift(Heq))));
grid on;
xlabel('Frequency (MHz)');
ylabel('Magnitude (dB)');
title('Equivalent Channel Frequency Response');
xlim([-20 20]);

%% Zoom in on the OFDM signal bandwidth
% Calculate OFDM bandwidth
delta_c = 1 / (N * T);           % Subcarrier spacing
BW_ofdm = N * delta_c / 1e6;     % OFDM bandwidth in MHz

fprintf('Subcarrier spacing: %.3f kHz\n', delta_c/1e3);
fprintf('OFDM bandwidth: %.3f MHz\n', BW_ofdm);

figure;
plot(linspace(-20, 20, 8192), 20*log10(abs(fftshift(Heq))), 'LineWidth', 1.5);
grid on;
xlabel('Frequency (MHz)');
ylabel('Magnitude (dB)');
title('Channel Response - OFDM Passband');
xlim([-BW_ofdm/2, BW_ofdm/2]);
ylim([min(20*log10(abs(fftshift(Heq))))-5, max(20*log10(abs(fftshift(Heq))))+5]);

%% largest difference (in dB) between passband points of the channel transfer function's magnitude
% Frequency axis in MHz
freq_axis = linspace(-Fs/2, Fs/2, Nfft) / 1e6;

% Analyze flatness within passband
idx_passband = find(abs(freq_axis) <= BW_ofdm/2);
Heq_passband_dB = 20*log10(abs(fftshift(Heq)));
Heq_passband_dB = Heq_passband_dB(idx_passband);

% Passband Analysis
max_dB = max(Heq_passband_dB);
min_dB = min(Heq_passband_dB);
diff_dB = max_dB - min_dB;
fprintf('Maximum magnitude: %.2f dB\n', max_dB);
fprintf('Minimum magnitude: %.2f dB\n', min_dB);
fprintf('Peak-to-peak variation: %.2f dB\n', diff_dB);
\end{lstlisting}

\subsubsection{task5\_2.m}
\begin{lstlisting}[language=Matlab]
    close all; clear;

N=64;
T = 0.25e-6;
nullpos = [29, 30, 31, 32, 33, 34] + 1; %matlab indexa desde 1
prefix_redundancy = 0.09375; % 9.375%
Lc = round(prefix_redundancy * N);

OF = 10; % sampling rate of 10/T Hz
Fs = OF / T;

% random QPSK data symbols to modulate with OFDM
rng(2025);                              % Set seed for reproducibility
M = 16;
dataSymbols = randi([0 M-1], 10000, 1); % Generate 10000 random 16 QAM symbols
txSig = qammod(dataSymbols, M, 'UnitAveragePower', true); % 16 QAM modulation
scatterplot(awgn(txSig,20));
hold on; grid on;

data = txSig.';

[x, u, w] = OFDMmod(data, N, Lc, OF, nullpos);

% Define multipath channel htilde(t)
% h_tilde(t) = sum of Dirac deltas at specified delays
tau = [0, 2*T, 3*T, 4*T];              % Delays in seconds
alpha = [1, 0.7, 0.4, 0.5];            % Amplitudes
phi = [0, -pi/2, pi/4, pi/2];          % Phases

% Convert to complex channel taps
h_tilde = alpha .* exp(1j * phi);

% Convert delays to sample indices at rate Fs
delay_samples = round(tau * Fs);

% Create discrete-time channel impulse response
max_delay = max(delay_samples);
h_tilde_discrete = zeros(1, max_delay + 1);
h_tilde_discrete(delay_samples + 1) = h_tilde;

%% Generate received signal z(t) at receiver (no noise)
% z(t) = x(t) * h_tilde(t)

z = conv(x, h_tilde_discrete);
length(x)
length(h_tilde_discrete)
length(z)

\end{lstlisting}

\subsubsection{task5\_3.m}
\begin{lstlisting}[language=Matlab]
    close all; clear;

N=64;
T = 0.25e-6;
nullpos = [29, 30, 31, 32, 33, 34] + 1; %matlab indexa desde 1
prefix_redundancy = 0.09375; % 9.375%
Lc = round(prefix_redundancy * N);

OF = 10; % sampling rate of 10/T Hz
Fs = OF / T;

% random QPSK data symbols to modulate with OFDM
rng(2025);                              % Set seed for reproducibility
M = 16;
dataSymbols = randi([0 M-1], 10000, 1); % Generate 10000 random 16 QAM symbols
txSig = qammod(dataSymbols, M, 'UnitAveragePower', true); % 16 QAM modulation
%scatterplot(awgn(txSig,20));hold on; grid on;
data = txSig.';

[x, u, w] = OFDMmod(data, N, Lc, OF, nullpos);

% Define multipath channel htilde(t)
% h_tilde(t) = sum of Dirac deltas at specified delays
tau = [0, 2*T, 3*T, 4*T];              % Delays in seconds
alpha = [1, 0.7, 0.4, 0.5];            % Amplitudes
phi = [0, -pi/2, pi/4, pi/2];          % Phases

% Convert to complex channel taps
h_tilde = alpha .* exp(1j * phi);

% Convert delays to sample indices at rate Fs
delay_samples = round(tau * Fs);

% Create discrete-time channel impulse response
max_delay = max(delay_samples);
h_tilde_discrete = zeros(1, max_delay + 1);
h_tilde_discrete(delay_samples + 1) = h_tilde;

z = conv(x, h_tilde_discrete);

% Pasamos un vector de unos, por lo que el ecualizador no hace nada.
dem_data = OFDMdem(z, N, Lc, OF, ones(N, 1), nullpos);

%% Plot demodulated data (No Equalization)
figure;
scatter(real(dem_data), imag(dem_data), 10, 'b', 'filled', 'MarkerFaceAlpha', 0.5);
grid on;
xlabel('In-Phase');
ylabel('Quadrature');
title('Received 16-QAM (No Equalization Assumed)');
axis equal;

hold on;
% IMPORTANTE: Usar 'UnitAveragePower' true para que la escala coincida
ideal_qam = qammod(0:15, M, 'UnitAveragePower', true);
scatter(real(ideal_qam), imag(ideal_qam), 100, 'r', 'x', 'LineWidth', 2);
legend('Received (Unequalized)', 'Ideal 16-QAM', 'Location', 'best');
\end{lstlisting}

\subsubsection{task5\_4.m}
\begin{lstlisting}[language=Matlab]
    close all; clear;

N=64;
T = 0.25e-6;
nullpos = [29, 30, 31, 32, 33, 34] + 1; %matlab indexa desde 1
prefix_redundancy = 0.09375; % 9.375%
Lc = round(prefix_redundancy * N);

OF = 10; % sampling rate of 10/T Hz
Fs = OF / T;

% random QPSK data symbols to modulate with OFDM
rng(2025);                              % Set seed for reproducibility
M = 16;
dataSymbols = randi([0 M-1], 10000, 1); % Generate 10000 random 16 QAM symbols
txSig = qammod(dataSymbols, M, 'UnitAveragePower', true); % 16 QAM modulation
%scatterplot(awgn(txSig,20));hold on; grid on;

data = txSig.';

[x, u, w] = OFDMmod(data, N, Lc, OF, nullpos);

% Define multipath channel htilde(t)
% h_tilde(t) = sum of Dirac deltas at specified delays
tau = [0, 2*T, 3*T, 4*T];              % Delays in seconds
alpha = [1, 0.7, 0.4, 0.5];            % Amplitudes
phi = [0, -pi/2, pi/4, pi/2];          % Phases

% Convert to complex channel taps
h_tilde = alpha .* exp(1j * phi);

% Convert delays to sample indices at rate Fs
delay_samples = round(tau * Fs);

% Create discrete-time channel impulse response
max_delay = max(delay_samples);
h_tilde_discrete = zeros(1, max_delay + 1);
h_tilde_discrete(delay_samples + 1) = h_tilde;

z = conv(x, h_tilde_discrete);

dem_data = OFDMdem(z, N, Lc, OF, ones(N, 1), nullpos);

%% Plot demodulated data for specific subcarriers (no equalization)
k1 = 10;
idx_k10 = k1:N:length(dem_data);
symbols_k10 = dem_data(idx_k10);

k2 = 46;
idx_k46 = k2:N:length(dem_data);
symbols_k46 = dem_data(idx_k46);

figure;
subplot(1,2,1);
scatter(real(symbols_k10), imag(symbols_k10), 10, 'g', 'filled', 'MarkerFaceAlpha', 0.5);
grid on;
xlabel('In-Phase');
ylabel('Quadrature');
title(sprintf('Subcarrier k=%d', k1));
axis equal;

subplot(1,2,2);
scatter(real(symbols_k46), imag(symbols_k46), 10, 'y', 'filled', 'MarkerFaceAlpha', 0.5);
grid on;
xlabel('In-Phase');
ylabel('Quadrature');
title(sprintf('Subcarrier k=%d', k2));
axis equal;

\end{lstlisting}

\subsubsection{task5\_5.m}
\begin{lstlisting}[language=Matlab]
    close all; clear;

N=64;
T = 0.25e-6;
nullpos = [29, 30, 31, 32, 33, 34] + 1; %matlab indexa desde 1
prefix_redundancy = 0.09375; % 9.375%
Lc = round(prefix_redundancy * N);

OF = 10; % sampling rate of 10/T Hz
Fs = OF / T;

% random QPSK data symbols to modulate with OFDM
rng(2025);                              % Set seed for reproducibility
M = 16;
dataSymbols = randi([0 M-1], 10000, 1); % Generate 10000 random 16 QAM symbols
txSig = qammod(dataSymbols, M, 'UnitAveragePower', true); % 16 QAM modulation
%scatterplot(awgn(txSig,20));hold on; grid on;
data = txSig.';

[x, u, w] = OFDMmod(data, N, Lc, OF, nullpos);

% Define multipath channel htilde(t)
% h_tilde(t) = sum of Dirac deltas at specified delays
tau = [0, 2*T, 3*T, 4*T];              % Delays in seconds
alpha = [1, 0.7, 0.4, 0.5];            % Amplitudes
phi = [0, -pi/2, pi/4, pi/2];          % Phases

% Convert to complex channel taps
h_tilde = alpha .* exp(1j * phi);

% Convert delays to sample indices at rate Fs (Oversampled)
delay_samples_Fs = round(tau * Fs);

% Create discrete-time channel impulse response (High Res)
max_delay_Fs = max(delay_samples_Fs);
h_physical = zeros(1, max_delay_Fs + 1);
h_physical(delay_samples_Fs + 1) = h_tilde;
z = conv(x, h_physical);

delay_samples_symbol = round(tau / T); % [0, 2, 3, 4]
h_eff = zeros(1, max(delay_samples_symbol) + 1);
h_eff(delay_samples_symbol + 1) = h_tilde;

% Compute correct H[k] for the equalizer
H = fft(h_eff, N).';

dem_data_correct = OFDMdem(z, N, Lc, OF, H, nullpos);

%% Plot received constellation WITH equalization
figure;
scatter(real(dem_data_correct), imag(dem_data_correct), 10, 'b', 'filled', 'MarkerFaceAlpha', 0.5);
grid on;
xlabel('In-Phase');
ylabel('Quadrature');
title('Received 16-QAM Constellation - With Perfect Equalization');
axis equal;
\end{lstlisting}

\subsubsection{task5\_6.m}
\begin{lstlisting}[language=Matlab]
    close all; clear;

N=64;
T = 0.25e-6;
nullpos = [29, 30, 31, 32, 33, 34] + 1; %matlab indexa desde 1
prefix_redundancy = 0.09375; % 9.375%
Lc = round(prefix_redundancy * N);

OF = 10; % sampling rate of 10/T Hz
Fs = OF / T;

% random QPSK data symbols to modulate with OFDM
rng(2025);                              % Set seed for reproducibility
M = 16;
dataSymbols = randi([0 M-1], 10000, 1); % Generate 10000 random 16 QAM symbols
txSig = qammod(dataSymbols, M, 'UnitAveragePower', true); % 16 QAM modulation
%scatterplot(awgn(txSig,20));hold on; grid on;
data = txSig.';

[x, u, w] = OFDMmod(data, N, Lc, OF, nullpos);

%% \tilde h(t) = \delta(t) -0.8\delta(t-5T) +0.7\delta(t-10T).
tau_symbol_rate = [0, 5, 10]; % Porque son 0, 5T, 10T
alpha = [1, -0.8, 0.7];

h_eff(tau_symbol_rate + 1) = alpha;

H = fft(h_eff, N).'; 

delay_samples_Fs = tau_symbol_rate * OF;
h_physical = zeros(1, max(delay_samples_Fs) + 1);
h_physical(delay_samples_Fs + 1) = alpha;

% Received signal
z = conv(x, h_physical);

% Demodulate with equalization
dem_data = OFDMdem(z, N, Lc, OF, H, nullpos);

%% Plot constellation
figure;
scatter(real(dem_data), imag(dem_data), 10, 'b', 'filled', 'MarkerFaceAlpha', 0.5);
grid on;
xlabel('In-Phase');
ylabel('Quadrature');
title('Received 16-QAM Constellation - Simplified Channel');
axis equal;

% Ideal constellation for reference
hold on;
ideal_qam = qammod(0:15, 16, 'UnitAveragePower', true);
scatter(real(ideal_qam), imag(ideal_qam), 100, 'r', 'x', 'LineWidth', 2);
legend('Received (with EQ)', 'Ideal 16-QAM', 'Location', 'best');
\end{lstlisting}