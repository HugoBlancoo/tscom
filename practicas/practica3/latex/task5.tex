Suppose now that fluid can be added to the tank, and that the tank is not 100\% tight, so that there may be some leakage. We have no control regarding the incoming or outcoming fluid flow, so we just model it as a random variable, which will become part of the unknown state vector. Thus, $\bm s_n$ is now a two-dimensional vector whose first entry is the fluid level at time $nT$, and whose second entry is the fluid flow (in e.g. cm$^3/$s) at time $nT$. If we assume\footnote{Note that this approximation is accurate if we sample significantly faster than the rate of change of the fluid flow.} that this flow remains approximately constant from $t=(n-1)T$ to $t=nT$, we can write
\begin{eqnarray}
    \mbox{[fluid level at time $nT$]} &=& \mbox{[fluid level at time $(n-1)T$]} \nonumber \\
    & & {}+ \, \frac{T}{A} \,\times \, \mbox{[fluid flow at time $(n-1)T$]},
\end{eqnarray}
where $A$ is the area of the tank cross-section (assumed constant with height).

Our measurements, as before, are given by the hydrostatic pressure at the bottom of the tank.

%%%%%%%%%%%%%%%%%%%%%%%%%%%%%%%%%%%%%%%%%%%%%%%%%%%%%%%%%%%%%%%%%%%%%%%%%%%%%%%%%%%%%%%%%%%%%%%%%%%%%
%%%%%%%%%%%%%%%%%%%%%%%%%%%%%%%%%%%%%%%%%%%%%%%%%%%%%%%%%%%%%%%%%%%%%%%%%%%%%%%%%%%%%%%%%%%%%%%%%%%%%
\textbf{Question: Find a ``state evolution'' and ``measurement'' description for the system in the form of (4), assuming that the fluid flow remains constant with time. (Assume that the tank is sufficiently big so that it never overflows, and that the initial fluid level is sufficiently high so that the tank never empties out).}
\vspace{0.5cm}

\begin{itemize}
    \item State vector:

          Now we have two unknowns which change over time:
          \subitem The fluid level ($l_n$) in cm.
          \subitem The fluid flow ($q_n$) in $cm^3/s$
          Therefore our state is now a 2x1 vector:
          $$\bm s_n = \begin{bmatrix} l_n \\ q_n \end{bmatrix}$$
    \item Transition matrix:

          We are looking for a matrix $A_n$ that tells us how to calculate the current state from the previous one. $s_n = A_n s_{n-1}$.

          We know that $l_n = l_{n-1} + \frac{T}{A_{tank}} \cdot q_{n-1}$.

          We also know that $q_n$ is constant, so $q_n = q_{n-1}$

          We can write it as a lineal system and extract the matrix:
          $$\begin{cases}
                  l_n = \mathbf{1} \cdot l_{n-1} + \mathbf{\frac{T}{A_{tank}}} \cdot q_{n-1} \\
                  q_n = \mathbf{0} \cdot l_{n-1} + \mathbf{1} \cdot q_{n-1}
              \end{cases}$$
          $$\bm A_n = \begin{bmatrix} 1 & \frac{T}{A_{tank}} \\ 0 & 1 \end{bmatrix}$$
    \item Observation Matrix:

          $x_n = c_h \cdot l_n + \text{noise}$
          $$x_n = \begin{bmatrix} c_h & 0 \end{bmatrix} \begin{bmatrix} l_n \\ q_n \end{bmatrix} + v_n$$
          This means that $\bm H_n = \begin{bmatrix} c_h & 0 \end{bmatrix}$

\end{itemize}

%%%%%%%%%%%%%%%%%%%%%%%%%%%%%%%%%%%%%%%%%%%%%%%%%%%%%%%%%%%%%%%%%%%%%%%%%%%%%%%%%%%%%%%%%%%%%%%%%%%%%
%%%%%%%%%%%%%%%%%%%%%%%%%%%%%%%%%%%%%%%%%%%%%%%%%%%%%%%%%%%%%%%%%%%%%%%%%%%%%%%%%%%%%%%%%%%%%%%%%%%%%
\textbf{Question: Write a Matlab script {\tt kalman\_flow.m} to simulate this dynamical system and the corresponding Kalman filter to estimate the fluid level and the fluid flow. Assume that the sensor works correctly (i.e., $p=0$ in \eqref{eq:xno}), measures pressure in mbar, taking one measurement every 5 seconds, and that the cross-section of the tank is circular with diameter 22 cm. Fluid level is to be expressed in cm, as before, whereas fluid flow is to be expressed in cm$^3/$s.}
\vspace{0.5cm}

COMPLETAR

The initial true fluid level is 340 cm; our guess is 250 cm, with a standard deviation of 11 cm. The true fluid flow is constant with time and equal to 33 cm$^3$/s; our guess is 0 cm$^3/$s, with standard deviation of 10 cm$^3/$s. Measurement errors are zero-mean uncorrelated Gaussian, with standard deviation of 25 mbar

%%%%%%%%%%%%%%%%%%%%%%%%%%%%%%%%%%%%%%%%%%%%%%%%%%%%%%%%%%%%%%%%%%%%%%%%%%%%%%%%%%%%%%%%%%%%%%%%%%%%%
%%%%%%%%%%%%%%%%%%%%%%%%%%%%%%%%%%%%%%%%%%%%%%%%%%%%%%%%%%%%%%%%%%%%%%%%%%%%%%%%%%%%%%%%%%%%%%%%%%%%%
\textbf{Question: Plot the time evolution of the fluid level and fluid flow, the measurements, and the estimates, for up to $60$ minutes, and for two different executions. Comment on your results.}
\vspace{0.5cm}

COMPLETAR

%%%%%%%%%%%%%%%%%%%%%%%%%%%%%%%%%%%%%%%%%%%%%%%%%%%%%%%%%%%%%%%%%%%%%%%%%%%%%%%%%%%%%%%%%%%%%%%%%%%%%
%%%%%%%%%%%%%%%%%%%%%%%%%%%%%%%%%%%%%%%%%%%%%%%%%%%%%%%%%%%%%%%%%%%%%%%%%%%%%%%%%%%%%%%%%%%%%%%%%%%%%
\textbf{Question: Plot the time evolution of the standard deviation of the estimation error for both the fluid level (in cm) and the fluid flow (in cm$^3/$s), also for two different realizations. Comment on your results.}
\vspace{0.5cm}

COMPLETAR

%%%%%%%%%%%%%%%%%%%%%%%%%%%%%%%%%%%%%%%%%%%%%%%%%%%%%%%%%%%%%%%%%%%%%%%%%%%%%%%%%%%%%%%%%%%%%%%%%%%%%
%%%%%%%%%%%%%%%%%%%%%%%%%%%%%%%%%%%%%%%%%%%%%%%%%%%%%%%%%%%%%%%%%%%%%%%%%%%%%%%%%%%%%%%%%%%%%%%%%%%%%
\textbf{Question:  Repeat the above two points if we change the standard deviation for our guess of the flow to 2 cm$^3/$s, and comment on your results.}
\vspace{0.5cm}